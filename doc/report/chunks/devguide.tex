% ==============================================================================
%
%                      P R O J E C T   S T R U C T U R E
%
% ==============================================================================
\chapter{Project Structure} % <<< -------------------------------------------- %
\label{ch:devguide:project_structure}
% ---------------------------------------------------------------------------- %

This chapter contains answers the question of ``What do I find where if I want
to  do $\langle  X  \rangle$?'' It guides  through  the top-level  directories
of  the  repository  and  presents  some information  on  the  more  important
subdirectories. It  is not  meant as  a  replacement to  reading the  READMEs,
the  technical  documentation  for  the respective  components,  or  toolchain
documentation  by third  parties. Rather, it  is indented  as a  guide to  get
started and  point the developer in  the right direction.

The  repository tree  with its  major  components is  shown on  the next  page
in  Figure~\ref{fig:project_tree},  along   with  some  explanatory  remarks
for  respective  nodes  of  the  tree.   Throughout  this  chapter,  the  root
directory \code{/}  is understood  to be  the top-level  repository directory.
\emph{NOTE:} Only  the  more important  parts  of  the project  structure  are
mentioned here; less crucial components have been omitted.

The rest  of the  developer guide explains  the setup and  usage of  the major
components listed  in the above  directory trees. For the most  up-to-date and
complete documentation, consulting the READMEs and source code is advised.

%\vspace{6ex}
%
%\noindent\begin{minipage}[t]{0.33\textwidth}
%    \noindent\dirtree{%
%        .1 /.
%        .2 design/.
%        .3 /filter/.
%        .3 /fpga/.
%        .2 doc/.
%        .3 report/.
%        .3 verfication/.
%        .3 vivado-install/\vspace{7.75ex}.
%        .2 env/\vspace{6ex}.
%        .2 (continued on next page).
%    }
%\end{minipage}
%\hfill
%\begin{minipage}[t]{0.64\textwidth}
%    \begin{description}\tightlist
%        \item[\code{design/}]       contains      the       filter      design
%            toolchain     \code{filter/}     which     is     described     in
%            Chapter~\ref{ch:devguide:filter_toolchain}. Additionally,  it  has
%            a  subdirectory   for  running   FPGA  tests  with   the  designed
%            filters, used  to create the  filter resource usage  analysis from
%            Appendix~\ref{sec:fir_filter_resouce_usage}.
%
%        \item[\code{doc/}] is where the project documentation and associated
%            tools      are      located.     \code{report/}      is      where
%            the     directory      containing     this      document.      The
%            \code{verification/}     directory     contains    the     scripts
%            and    results   on    which   Chapter~\ref{ch:verification}    is
%            based. A   pictorial   guide   to    the   Vivado   install   (see
%            Section~\ref{sec:devguide:fpga_toolchain:vivado}) can  be found in
%            \code{vivado-install/}.
%
%        \item[\code{env/}] has the  components which are needed to  set up the
%            build  box. See  Section~\ref{sec:devguide:fpga_toolchain:buildbox}
%            for the instructions on how to accomplish this.
%    \end{description}
%\end{minipage}
%
%\newpage
%
%\noindent\begin{minipage}[t]{0.33\textwidth}
%    \noindent\dirtree{%
%        .1 / (continued).
%        .2 firmware/.
%        .3 arm/.
%        .4 patches/.
%        .4 scripts/.
%        .4 server/.
%        .3 bin/.
%        .3 fpga/\vspace{16ex}.
%        .2 scope/\vspace{8ex}.
%        .2 resources/.
%    }
%\end{minipage}
%\hfill
%\begin{minipage}[t]{0.64\textwidth}
%    \vspace{1.33ex}
%    \begin{description}\tightlist
%        \item[\code{firmware/}]  comprises all  necessary components  to build
%            the Linux image for the STEMlab and the bitstream. \code{patches/}
%            is a  set of  patches which  are needed  to get  Ubuntu to  run on
%            the  STEMlab,  \code{scripts/} are  some  helper  scripts to  ease
%            development and deployment, and \code{server/} contains the server
%            and its needed dependencies.
%
%            \code{bin/} is a collection of helper scripts and binary blobs
%                related to the creation of the board firmware.
%
%            \code{fpga/} consists  of the  FPGA toolchain, which  is primarily
%                the collection of  Tcl scripts and Makefiles  used to generate
%                the Vivado projects, block designs, and the bitstream.
%
%            \item[\code{scope/}]   encompasses   the    scope   project   (see
%                Chapter~\ref{ch:gui}). Note   that  this   is  located   in  a
%                dedicated  Git repository  and is  included via  the mechanism
%                offered by Git submodules.
%
%            \item[\code{resources/}]  are  documents   and  links  which  have
%                accumulated through the  course of the project  and which have
%                proven  more or  less  useful.  This  primarily includes  data
%                sheets and user manuals from third parties.
%    \end{description}
%\end{minipage}
%
%\vspace{6ex}

% Alternative: everything on a single page
\newpage
\vspace*{20ex}
\noindent\begin{minipage}[t]{0.33\textwidth}
    \noindent\dirtree{%
        .1 /.
        .2 design/.
        .3 /filter/.
        .3 /fpga/.
        .2 doc/.
        .3 report/.
        .3 verfication/.
        .3 vivado-install/\vspace{7.75ex}.
        .2 env/\vspace{5.5ex}.
        .2 firmware/.
        .3 arm/.
        .4 patches/.
        .4 scripts/.
        .4 server/\vspace{4.5ex}.
        .3 bin/\vspace{2.5ex}.
        .3 fpga/\vspace{8.5ex}.
        .2 scope/\vspace{8ex}.
        .2 resources/.
    }
\end{minipage}
\hfill
\begin{minipage}[t]{0.64\textwidth}
    \begin{description}\tightlist
        \item[\code{design/}]       contains      the       filter      design
            toolchain     \code{filter/}     which     is     described     in
            Chapter~\ref{ch:devguide:filter_toolchain}. Additionally,  it  has
            a  subdirectory   for  running   FPGA  tests  with   the  designed
            filters, used  to create the  filter resource usage  analysis from
            Appendix~\ref{sec:fir_filter_resouce_usage}.

        \item[\code{doc/}] is where the project documentation and associated
            tools      are      located.     \code{report/}      is      where
            the     directory      containing     this      document.      The
            \code{verification/}     directory     contains    the     scripts
            and    results   on    which   Chapter~\ref{ch:verification}    is
            based. A   pictorial   guide   to    the   Vivado   install   (see
            Section~\ref{sec:devguide:fpga_toolchain:vivado}) can  be found in
            \code{vivado-install/}.

        \item[\code{env/}] has the  components which are needed to  set up the
            build  box. See  Section~\ref{sec:devguide:fpga_toolchain:buildbox}
            for the instructions on how to accomplish this.

        \item[\code{firmware/}]  comprises all  necessary components  to build
            the Linux image for the STEMlab and the bitstream. \code{patches/}
            is a  set of  patches which  are needed  to get  Ubuntu to  run on
            the  STEMlab,  \code{scripts/} are  some  helper  scripts to  ease
            development and deployment, and \code{server/} contains the server
            and its needed dependencies.

            \textbf{\code{bin/}} is a collection  of helper scripts and binary
            blobs related to the creation of the board firmware.

            \textbf{\code{fpga/}}  consists of  the FPGA  toolchain, which  is
            primarily  the collection  of Tcl  scripts and  Makefiles used  to
            generate the Vivado projects, block designs, and the bitstream.

        \item[\code{scope/}]    encompasses    the    scope    project    (see
            Chapter~\ref{ch:gui}). Note that  this is  located in  a dedicated
            Git repository  and is included  via the mechanism offered  by Git
            submodules.

        \item[\code{resources/}]   are   documents   and  links   which   have
            accumulated  through the  course  of the  project  and which  have
            proven more or  less useful.  This primarily  includes data sheets
            and user manuals from third parties.
    \end{description}
\end{minipage}
\vspace*{2ex}
\figcaption[Project Structure Tree]{%
    Project structure with major directories and subdirectories%
    \label{fig:project_tree}
}



%>>>
% ==============================================================================
%
%                          S O C   T O O L C H A I N
%
% ==============================================================================
\chapter{FPGA/SoC Toolchain} % <<< ------------------------------------------- %
\label{ch:devguide:fpga_toolchain}
% ---------------------------------------------------------------------------- %

Because the FPGA/SoC requires a lot of different utilities and environments, it
is advisable to  have a fixed development environment as  hardware tooling often
breaks at the slightest change.
For this reason we  provide a build box\footnote{%
    A virtual machine image for the purposes of development.
}
which contains  an Ubuntu Linux. To set  up the build box, Vagrant  and ansible
are required.  The former pulls the base Linux box from a global repository at
HashiCorp\footnote{The corporation  behind Vagrant}.  ansible is then  used to
provision the box  to install all the necessary tooling. Once  that is set up,
the user has to perform a graphical install of the Vivado toolchain.

This chapter  describes how to  set up  the build box  and Vivado, and  how to
generate a Linux image which can be  flashed onto the STEMlab's SD card to get
the device up and running.

\emph{Note:} The following indicates a code snippet which has to be entered on
the command line:
\begin{commandshell}
    enter commands here
\end{commandshell}

\section{Setting Up the Build Box} % ----------------------------------------- %
\label{sec:devguide:fpga_toolchain:buildbox}
% ---------------------------------------------------------------------------- %

\paragraph{The following  prerequisites} need  to be  installed onto  the host
machine first:
\begin{itemize}\tightlist
    \item Vagrant
    \item ansible
    \item VirtualBox
\end{itemize}
If you are on a Linux or OSX  and use a package manager to install VirtualBox.
It is also necessary to install the Virtualbox Guest Additions packages.  They
are  required in  later  stages during  the  setup  of the  box.   It is  also
advisible to perform a
\begin{commandshell}
    sudo vboxreload
\end{commandshell}
\noindent Otherwise a reboot of the host system might be necessary.

After all the  tools on the host have  been installed, we move to  the root of
the project  repository. For all the further  steps we assume that  we operate
from that root directory.

To do an initial setup of the box, enter
\begin{commandshell}
    cd env
    vagrant up
\end{commandshell}
\noindent This  will boot the  box and  provision it using  ansible. This step
requires a  stable internet connection. If  anything fails during  the initial
setup (red messages in the shell), you can run and retry the provisioning with
\begin{commandshell}
    vagrant provision
\end{commandshell}

Once the provisioning  has finished, the build box should  be rebootet because
some  changes  (for  example  the installed  desktop  environment)  require  a
reboot. Do this by running
\begin{commandshell}
    vagrant halt
    vagrant up
\end{commandshell}
\noindent If you  want to make changes  to the default box setup,  have a look
at  the  file \newline\mbox{\code{/env/roles/common/tasks/main.yml}}  and  the
official ansible documentation~\cite{ansible-docs}.

The  fully   configured  build   box  should   contain  two   shared  folders:
\code{/vagrant/}  points  to  the  \code{/env/} directory  on  the  host,  and
\code{/repo/} points to \code{/} on the host.

\paragraph{IMPORTANT:} The   password  of   the   default   vagrant  user   is
\textbf{\code{vagrant}} and has \code{sudo} privileges.


\section{Setting up Vivado} % ------------------------------------------------ %
\label{sec:devguide:fpga_toolchain:vivado}
% ---------------------------------------------------------------------------- %

After having  completed the basic  build box  setup as outlined  above, Vivado
needs to  be installed.   For this  Section, we assume  that all  commands are
executed on the VirtualBox (\emph{guest} for the rest of this manual).  Enter
\begin{commandshell}
    /repo/Xilinx/Downloads/Xilinx_Vivado_SDK_2016.2_0605_1_Lin64.bin
\end{commandshell}
\noindent into a  shell to start the Vivado install.   The graphical installer
will guide through the process. A complete  pictorial guide for this is beyond
the scope of this document, but  an illustrated guide with screenshots for all
steps  is available  under~\cite{vivado-install-guide}.  For  documentation on
Vivado  itself,  it  is  recommended  to have  a  look  at  the  documentation
portal from  Xilinx~\cite{xilinx:documentation-portal}; it  contains extensive
documentation on many topics.

\section{Building a Linux} % ------------------------------------------------- %
\label{sec:devguide:fpga_toolchain:linux}
% ---------------------------------------------------------------------------- %

With the fully set up build box it is possible to build an image with only two
commands. While it  is possible to only  perform the build steps  once changes
have been made to the FPGA project  or the server application, it is advisable
to perform  an initial  build. This allows  to check if  everything is  set up
correctly and works as intended.

First  we  copy  the  repo  to  a  \textbf{non-shared}  folder  on  the  guest
machine,  because uboot  requires  \code{mmap()}, which  cannot handle  shared
folders. After  that,  building the  image  is  a single-step  process.   This
includes building the Linux, the bitstream, the required firststage bootloader
and  the  board  support  package,   the  logger  kernel  module,  the  server
application, and the scope application.  After  that, a bash script is used to
create an  image with all  the components, mount  it and provision  the ubuntu
environment which is to be run on the STEMlab. To start this process, enter
\begin{commandshell}
        cp -r /repo ~/local_folder
        cd ~/local_folder
        make init
\end{commandshell}

After the build process has finished, the image can be created using
\begin{commandshell}
        cd ~/local_folder/firmware/arm
        sudo sh scripts/image.sh scripts/ubuntu.sh red-pitaya-ubuntu.img 1024
\end{commandshell}
\noindent These scripts were initially  created by the Red Pitaya corporation,
altered by  Pavel Demin~\cite{pita:github:pitaya-notes},  and have  again been
adapted to the requirements of this project.
%>>>
% ==============================================================================
%
%                       F I L T E R   T O O L C H A I N
%
% ==============================================================================
\chapter{Filter Toolchain} % <<< --------------------------------------------- %
\label{ch:devguide:filter_toolchain}
% ---------------------------------------------------------------------------- %

Designing   the  filters   according  to   the  desired   specifications  (see
Chapter~\ref{ch:filter_design})  is   performed  through   a  set   of  Matlab
scripts. This chapter describes the overall  structure of the script suite and
gives a basic usage example. All  functions have a \code{help} available which
describes their usage, particularly their respective interfaces. 

\section{Toolchain Structure} % ---------------------------------------------- %
\label{sec:devguide:filter_toolchain:structure}
% ---------------------------------------------------------------------------- %

This  section explains  which files  constitute the  toolchain and  what their
purpose is. The  entire toolchain can  be used either from  Matlab's graphical
front-end, or  from its commandline  mode. The following files are  present in
the filter design directory\footnote{%
    relative to global project root%
}:
\vspace{2ex}
\noindent\dirtree{%
    .1 design/filter/.
    .2 cliDispatcher.m.
    .2 guiWrapper.m.
    .2 generators/.
    .3 decCIC.m.
    .3 decFIR.m.
    .3 halfbandFIR.m.
    .3 compCIC.m.
    .3 cascador.m.
    .3 parcascador.m.
    .3 pardecFIR.m.
    .3 parhalfbandFIR.m.
    .2 plotData.
    .2 coefData.
    .2 Makefile.
    .2 README.md.
}

\paragraph{\code{cliDispatcher.m}} is  the main  script from where  the design
functions are initiated.   If you wish to  design a new filter  chain, this is
where its specifications are located, and from where the filter design scripts
are then called with those paramters.

\paragraph{\code{guiWrapper.m}} is  a convenience layer which  makes it easier
to work with \code{cliDispatcher.m} from Matlab's graphical front-end. It sets
some  of  the  configuration  parameters  for  calling  \code{cliDispatcher.m}
which are  otherwise set  in Matlab's commandline  interface when  calling the
dispatcher from there.

\paragraph{\code{generators/}}  is  the  directory  where  the  filter  design
scripts are located. These are split  into two primary groups: Functions which
design filters, and functions which  combine them into cascades. Note that all
generators support  generating multiple filters  at once for a  combination of
various paramters and  will pass back a cell array  with the resulting filters
and the paramters used in their specification.

This  allows to  iteratively generate  a large  set of  filters in  an initial
step  to assess  resource  usage or  other characteristics  for  a wide  range
of  parameters. The parameters  for  FIR  filters are  the  ones  laid out  in
Section~\ref{subsec:FIR_filters}. For half-band and CIC filters, they slightly
differ.  The filter design functions are:
\begin{description}\tightlist
    \item[\code{decCIC.m}] designs a CIC filter.
    \item[\code{decFIR.m}] designs a FIR filter.
    \item[\code{halfbandFIR.m}] designs a halfband filter.
    \item[\code{compCIC.m}] designs a compensator for a CIC filter.
\end{description}
Note that all of these have version which iterate in parallel over a given set
of  specifications,  prefixed  with  \code{par}  (e.g.  \code{pardecFIR}). The
interface for all parallel versions is identical. They can be used if Matlab's
parallel processing toolbox is available. If only a small set of filters is to
be designed,  it is  recommended to  use the  regular, serial  versions, since
starting up  a parallel processing  pool in Matlab is  a slow process  and its
overhead is  usually not  worth it in  those cases. In case  Matlab is  run in
commandline  mode, make  sure \emph{not}  to start  it with  the \code{-nojvm}
switch, since the parallel processing pool requires the Java Virtual Machine.

The functions  \code{cascador.m} and  \code{parcascador.m} create  cascades of
filter cell arrays  passed to them. Note that in order  to achieve the desired
iteration  result,  some  manual  intervention in  re-structuring  the  filter
objects  passed  to  the  cascade  functions  might  on  occasion  be  needed,
particularly in the case of cascading  other filters with CIC filters. This is
because the set of parameters used to design CIC filters is different from the
set of coefficients used to design FIR filters, which means the objects passed
back from the CIC and FIR filter functions might not always match as needed in
their structure to cascade them in  all possible manners. This is not an issue
when only cascading single CIC filters with other filters.

\paragraph{\code{coefData} and \code{plotData}} are  two directories which are
created  by the  functions  (if they  do  not yet  exist)  for storing  filter
property  data.  \code{coefData} contains  the  filter  coefficients for  each
filter which  has coefficients in a  Vivado-compliant \code{.coe} format. This
allows a direct import of Matlab's results into Vivado's FIR compiler.

\code{plotData} contains frequency  responses for each filter, as  well as any
potential cascade, in \code{.csv} format,  as given by Matlab's \code{freqz()}
function.  This  is used to  generate the  frequency response plots  from this
report, for examlpe.

\paragraph{The  \code{Makefile}} can  be used  to call  \code{cliDispatcher.m}
from the command line. It has  various targets for designing different filters
or filter  chains. Matlab will be called  in commandline mode, and  its output
redirected  to  a  log file,  so  not  output  will  generally be  visible  on
screen. This  is primarily  intended to  be used  when all  filters have  been
specified and fixed, and  the needed files are to be generated  for use on the
FPGA side.

\paragraph{The  \code{README.md}}  contains  additional information  which  is
beyond the scope  of this documentation. In general, it  is highly recommended
to  consult  both  the  \code{README}  and the  \code{help}  of  the  provided
functions  in  case  of questions,  as  well  as  the  code itself,  which  is
extensively commented (particularly \code{cliDispatcher.m}).


\section{Usage} % ------------------------------------------------------------ %
\label{sec:devguide:filter_toolchain:usage}
% ---------------------------------------------------------------------------- %

Now that we know  the location and purpose of the main  components, it is time
for a basic usage example. For this, it is useful to know how to use the basic
FIR filter  design function (Listing~\ref{lst:devguide:fdesign:fir}),  the CIC
design function  (Listing~\ref{lst:devguide:fdesign:cic}), and how  to cascade
filters (Listing~\ref{lst:devguide:fdesign:casc}). All  of this code is  to be
put  into  the \code{cliDispatcher.m}  file,  and  a \code{case}  created  for
it. This is shown in Listing~\ref{lst:devguide:fdesign:cliDispatchter}.

\begin{tcolorbox}[
    title={
        \refstepcounter{listing}
        \textbf{Listing \thelisting:} Using \code{cliDispatcher.m}
        \label{lst:devguide:fdesign:cliDispatchter}
        \addcontentsline{lol}{listing}{\protect\numberline{\thelisting}Using \code{cliDispatcher.m}}
    }
]
\inputminted[
    linenos,
    numbersep=4pt,
    style=solarizedlight,
]{matlab}{./code/fdesign/cliDispatcher.m}
\end{tcolorbox}

To execute \code{cliDispatcher.m}, Matlab's commandline interface can be used:
\begin{tcolorbox}[
    title={
        \refstepcounter{listing}
        \textbf{Listing \thelisting:} Calling \code{cliDispatcher.m} from Matlab's Commandline Interface
        \label{lst:devguide:fdesign:fir_cell}
        \addcontentsline{lol}{listing}{\protect\numberline{\thelisting}Calling \code{cliDispatcher.m} from Matlab's Commandline Interface}
    }
]
\begin{minted}{matlab}
>> filtertype='EXAMPLE_FILTER_AND_OR_FILTER_CHAIN';
>> cliDispatcher
\end{minted}
\end{tcolorbox}
\noindent Alternatively, one  may create an entry  in \code{guiWrapper.m} when
using Matlab's graphical interface for added convenience. An entry can be created
as show
\begin{tcolorbox}[
    title={
        \refstepcounter{listing}
        \textbf{Listing \thelisting:} Creating an Entry for \code{cliDispatcher.m} in \code{guiWrapper.m}
        \label{lst:devguide:fdesign:fir_cell}
        \addcontentsline{lol}{listing}{\protect\numberline{\thelisting}Creating an Entry for \code{cliDispatcher.m} in \code{guiWrapper.m}}
    }
]
\begin{minted}{matlab}
%% EXECUTE EXAMPLE_FILTER_AND_OR_FILTER_CHAIN
clear all;close all;clc;
filtertype = 'EXAMPLE_FILTER_AND_OR_FILTER_CHAIN';
disp('Designing Chain for R = EXAMPLE')
run cliDispatcher;
\end{minted}
\end{tcolorbox}

Besides  knowing  how  to  initiate   the  filter  design  toolbox,  one  must
obviously   also  know   how  to   actually  design   filtes.   The   code  in
Listing~\ref{lst:devguide:fdesign:fir}  designs \emph{two}  FIR filters,  with
two  different stopband  edge  frequencies (\code{Fst}).   The resulting  cell
array  \code{Hd} from  Listing~\ref{lst:devguide:fdesign:fir_cell} has  in its
first column the designed filter system  objects, and in the remaining columns
the design parameters used to specify the filter.

\begin{tcolorbox}[
    title={
        \refstepcounter{listing}
        \textbf{Listing \thelisting:} Designing two FIR Filtes 
        \label{lst:devguide:fdesign:fir}
        \addcontentsline{lol}{listing}{\protect\numberline{\thelisting}Designing two FIR Filters}
    }
]
\inputminted[
    linenos,
    numbersep=4pt,
    style=solarizedlight,
]{matlab}{./code/fdesign/fir.m}
\end{tcolorbox}
\begin{tcolorbox}[
    title={
        \refstepcounter{listing}
        \textbf{Listing \thelisting:} Cell Array with Two FIR Filters
        \label{lst:devguide:fdesign:fir_cell}
        \addcontentsline{lol}{listing}{\protect\numberline{\thelisting}Cell Array with Two FIR Filters}
    }
]
\begin{minted}{text}
>> Hd

Hd =

  2×6 cell array

    [1×1 dsp.FIRDecimator]    [0.2000]    [0.2500]    [0.2100]    [60]    [5]
    [1×1 dsp.FIRDecimator]    [0.2000]    [0.2500]    [0.2250]    [60]    [5]
\end{minted}
\end{tcolorbox}

Designing  a   CIC  filter   uses  slightly   different  parameters,   but  is
otherwise  similar. One  point of  note  is  that the  compensator's  passband
edge  is  specified relative  to  the  CIC  filter's incoming  sampling  rate,
not  the  sampling rate  at  which  the compensator  runs,  as  is common  for
FIR  filters  otherwise. Hence  \code{FpComp  =   1/R2}  on  line  \num{8}  of
Listing~\ref{lst:devguide:fdesign:cic}.

The \code{HdComp} cell array will look similar  to the cell array from the FIR
design  function above. However,  it will  have both  the cascade  of the  CIC
filter and its  compensator, as well as  each filter by itself,  in a separate
column entry.

\begin{tcolorbox}[
    title={
        \refstepcounter{listing}
        \textbf{Listing \thelisting:} Designing a CIC Filter and Its Compensator
        \label{lst:devguide:fdesign:cic}
        \addcontentsline{lol}{listing}{\protect\numberline{\thelisting}Designing a CIC Filter and Its Compensator}
    }
]
\inputminted[
    linenos,
    numbersep=4pt,
    style=solarizedlight,
]{matlab}{./code/fdesign/cic.m}
\end{tcolorbox}

Cascading two filters with the \code{cascador} or \code{parcascador} functions
is  shown in  Listing~\ref{lst:devguide:fdesign:casc}. It accepts  filter cell
arrays  as  returned  by  the  filter design  functions  (so,  \code{Hd}  from
\mbox{Listing~\ref{lst:devguide:fdesign:fir}},  for example). However,  before
being  given to  the \code{cascador}  function, the  filters which  are to  be
cascaded must  be packaged into a  single cell array, called  \code{stages} in
\mbox{Listing~\ref{lst:devguide:fdesign:casc}}. \code{Hd1}  and \code{Hd2} are
presumed to be of the form of \code{Hd} from above.

No special care  needs to be taken when handling  \code{HdComp} cell arrays to
\code{cascador}\footnote{%
    There is  a limitation  to the numbers  of filters Matlab  can chain  in a
    single  cascade. This limitation  might  be relevant  when designing  long
    filter chains. However,  multiple cascades can themselves  be cascaded, so
    this limitation can be worked around if necessary.
},
despite  its first  column  being  a cascade  filter  object
instead of  a single  filter. \code{cascador} will  notice the  difference and
unpack the cascade\footnote{
    Because cascade objects  can themselves be cascaded, this  is not strictly
    necessary.   But for  the sake  of easier  understanding and  elegance, we
    unpack cascades when possible.%
}
The other parameters  handed to \code{cascador}, such  as \code{R}, \code{Fst}
etc. should correspond  to the overall properties of the  cascade, and not the
individual stages. Since it is not possible to automatically determine some of
these properties before actually cascading  the filter, the user must manually
set these to the correct  values. Particularly when cascading cell arrays with
multiple filters, some care must be taken  in order to ensure that all desired
permutations are  produced. This is because the  \code{cascador} iterates over
these paramters  when cascading the  filters. They are  also used to  name the
resulting plot files for the cascade.

\begin{tcolorbox}[
    title={
        \refstepcounter{listing}
        \textbf{Listing \thelisting:} Cascading Two Filter Cell Arrays
        \label{lst:devguide:fdesign:casc}
        \addcontentsline{lol}{listing}{\protect\numberline{\thelisting}Cascading Two Filter Cell Arrays}
    }
]
\inputminted[
    linenos,
    numbersep=4pt,
    style=solarizedlight,
]{matlab}{./code/fdesign/casc.m}
\end{tcolorbox}

While the above remarks cover the most essential information, they by no means
constitute  a comprehensive  guide. When  designing filters,  it is  therefore
highly  recommended to  consult  the \code{README}s,  the  \code{help} of  the
functions,  the  code  and  its  comments, as  well  as  the  official  Matlab
documentation.

%>>>
% ==============================================================================
%
%                                 S E R V E R
%
% ==============================================================================
\chapter{Server} % <<< ------------------------------------------------------- %
\label{ch:devguide:server}
% ---------------------------------------------------------------------------- %

The server application runs on the ARM  Linux and is responsible for moving
data from the FPGA to the network, and vice versa. This chapter explains how
it is to be compiled and how to expand its functionality, if needed.

\section{Building the Server} % ---------------------------------------------- %
\label{sec:devguide:server:build}
% ---------------------------------------------------------------------------- %

The  server can  only be  built using  the build  environment as  explained in
Chapter~\ref{ch:devguide:fpga_toolchain}.  A simple
\begin{commandshell}
    cd /repo/firmware/arm/server
    make
\end{commandshell}
\noindent should suffice to build all the external dependencies and the binary
for the ARM core. These externals can be rebuilt using
\begin{commandshell}
    cd /repo/firmware/arm/server
    make external
\end{commandshell}
\noindent The server  application can be rebuilt after changes  have been made
by issuing
\begin{commandshell}
    cd /repo/firmware/arm/server
    make arm
\end{commandshell}

The   server  application   is   a  one-file   application   and  depends   on
libuWebSockets~\cite{uws:github} and a  headerfile called \code{json.hpp} from
Niels Lohmann's \emph{JSON for Modern C++} project \cite{lohmann:github:json},
which contains the entire JSON library.   There is two important functions for
extending the  server application: \code{onHttpRequest}  and \code{onMessage},
which are explained in the following sections.

\section{Manually Starting the Server} % ---------------------------------------------- %
\label{sec:devguide:server:build}
% ---------------------------------------------------------------------------- %

TODO: Raphi read!
The image for the entire system is built such that it contains a Sys-V init script and the necessary entries to start on boot.
The Sys-V init script then in turn calls the server startup script.

This  script loads  the kernel  module and  sets the  \code{LD\_LIBRARY\_PATH}
variable such that  it can load the shared libraries  required. The script can
be executed with

\begin{commandshell}
    /opt/server/init_server.sh
\end{commandshell}

If the Sys-V init daemon already started an instance, it can be stopped using

\begin{commandshell}
    service server stop
\end{commandshell}

For the exact commands, consult the code of the \code{init\_server.sh} script.

\subsection{onHttpRequest} % ------------------------------------------------- %
\label{subsec:devguide:server:onhttprequest}
% ---------------------------------------------------------------------------- %

This callback  is executed  when the user  makes an HTTP  request wich  is not
an  \code{UPGRADE} request. Currently  it  simply serves  the  files from  the
filesystem, but  could easily be  extended to  outline data or  similar tasks.
Returning a  correct HTTP  Response has to  be done manually  and is  not well
documented in the library itself.

If the response should outline a \code{200 OK} status,
\begin{tcolorbox}
    \begin{minted}[autogobble]{C++}
        res->end(const char*, size_t);
    \end{minted}
\end{tcolorbox}
\noindent can be used with a string  and a size. The library then detects that
no header  is attached  to the  response and attaches  a proper  \code{200 OK}
header.

If a  custom header should be  attached, it first  has to be written  into the
answer, after which the actual content can be written into it as well:
\begin{tcolorbox}
    \begin{minted}[autogobble]{C++}
        std::string mime;
        char header[128];

        mime = std::string("text/css");
        std::string content = std::string("Hello World");
        int header_length = std::sprintf(
            header,
            "HTTP/1.1 200 OK\r\nContent-Length: %u\r\nContent-Type: %s\r\n\r\n",
            str.size(),
            mime.c_str()
        );
        res->write(header, header_length);
        res->end(str.c_str(), str.size());
    \end{minted}
\end{tcolorbox}

%I am not entirely sure if better handling of this will follow in the future or
%if it is  a performance thing, as  the library itself is  awesome in structure
%and tidyness. But docs sadly are very sparse


\subsection{onMessage} % ----------------------------------------------------- %
\label{subsec:devguide:server:onmessage}
% ---------------------------------------------------------------------------- %

This  callback  is triggered  when  a  new  message  is received  through  the
WebSocket.  For  now, this  call only  handles incoming  text messages  in our
project  as  those  contain  the instructions.   Binary  messages  are  simply
discarded but could be used at a later time to interface with the DAC.

TODO: Raphi read! (Evtl chönnt mer de satz au weg lah)
As a reference how exactly a command can be added, the code of the server application can be inspected

\section{Instruction Set} % -------------------------------------------------- %
\label{sec:devguide:server:instruction}
% ---------------------------------------------------------------------------- %

The        instruction        set        contains        eight        commands
which     are      listed     in     Listing~\ref{lst:devguide:server:instr:1}
through~\ref{lst:devguide:server:instr:8}. The   following  sections   explain
their basic purpose and functionality.

\subsection{Forcing a New Trigger Event} % ----------------------------------- %
\label{subsec:devguide:server:forcing_trigger}
% ---------------------------------------------------------------------------- %

This  command  forces  the  logger  to  finish  its  current  frame. It  still
repects the  set \code{pre}  and \code{suf} conditions.   The server  does not
automatically send the recorded frame. It has to be requested separately.  The
argument \code{forceTrigger} always has to be set to \code{true}.

\begin{tcolorbox}[
    title={
        \refstepcounter{listing}
        \textbf{Listing \thelisting:} Forcing a New Trigger Event
        \label{lst:devguide:server:instr:1}
        \addcontentsline{lol}{listing}{\protect\numberline{\thelisting}Forcing a New Trigger Event}
    }
]
\inputminted[
    linenos,
    numbersep=4pt,
    style=solarizedlight,
]{javascript}{./code/serverinstructions/forceTrigger.js}
\end{tcolorbox}

\subsection{Configuring the Frame Sent by the Server } % --------------------- %
\label{subsec:devguide:server:config_frame}
% ---------------------------------------------------------------------------- %

This command tells the server how big the frame should be and how many samples
have to be recorded before and after  the trigger. All arguments have to be of
the numerical type in the JSON format, not strings.

\begin{tcolorbox}[
    title={
        \refstepcounter{listing}
        \textbf{Listing \thelisting:} Configuring the Frame Sent by the Server
        \label{lst:devguide:server:instr:2}
        \addcontentsline{lol}{listing}{\protect\numberline{\thelisting}Configuring the Frame Sent by the Server}
    }
]
\inputminted[
    linenos,
    numbersep=4pt,
    style=solarizedlight,
]{javascript}{./code/serverinstructions/frameconfig.js}
\end{tcolorbox}

\subsection{Setting the Number of Logged Channels} % ------------------------- %
\label{subsec:devguide:server:no_of_logged_channels}
% ---------------------------------------------------------------------------- %

This command tells the server how  many channels are being logged. This should
always be two as the STEMlab does not support more channels. The logger itself
would support up  to eight channels.  The  argument has to be a  number in the
JSON format, not strings.

\begin{tcolorbox}[
    title={
        \refstepcounter{listing}
        \textbf{Listing \thelisting:} Setting the Number of Logged Channels
        \label{lst:devguide:server:instr:3}
        \addcontentsline{lol}{listing}{\protect\numberline{\thelisting}Setting the Number of Logged Channels}
    }
]
\inputminted[
    linenos,
    numbersep=4pt,
    style=solarizedlight,
]{javascript}{./code/serverinstructions/numberofchannels.js}
\end{tcolorbox}

\subsection{Reading the Currently Stored Frame} % ---------------------------- %
\label{subsec:devguide:server:currently_stored_frame}
% ---------------------------------------------------------------------------- %

This command  forces the server  to send the  currently stored frame  over the
binary channel as soon as the current  frame is finished. If the logger is not
recording currently the frame will be sent immediately.
The  \code{channel} argument  has to  be a  number in  the JSON  format, not
strings.

\begin{tcolorbox}[
    title={
        \refstepcounter{listing}
        \code{Listing \thelisting:} Reading the Currently Stored Frame
        \label{lst:devguide:server:instr:4}
        \addcontentsline{lol}{listing}{\protect\numberline{\thelisting}Reading the Currently Stored Frame}
    }
]
\inputminted[
    linenos,
    numbersep=4pt,
    style=solarizedlight,
]{javascript}{./code/serverinstructions/readframe.js}
\end{tcolorbox}

\subsection{Requesting a New Frame and Reading It When It Is Ready} % -------- %
\label{subsec:devguide:server:request_new_frame}
% ---------------------------------------------------------------------------- %

This command  forces the  server to  start a new  frame and  send it  over the
binary channel as soon as the  frame is finished.  The \code{channel} argument
has to be a number in the JSON format, not strings.

\begin{tcolorbox}[
    title={
        \refstepcounter{listing}
        \textbf{Listing \thelisting:} Requesting a New Frame and Reading It When It Is Ready
        \label{lst:devguide:server:instr:5}
        \addcontentsline{lol}{listing}{\protect\numberline{\thelisting}Requesting a New Frame and Read It When It Is Ready}
    }
]
\inputminted[
    linenos,
    numbersep=4pt,
    style=solarizedlight,
]{javascript}{./code/serverinstructions/requestframe.js}
\end{tcolorbox}

\subsection{Setting the Sampling Rate} % ------------------------------------- %
\label{subsec:devguide:server:set_sampling_rate}
% ---------------------------------------------------------------------------- %

This command sets the  sampling rate.  The argument has to be  a number in the
JSON format and has to be the samplingrate in \si{\Hz}

\begin{tcolorbox}[
    title={
        \refstepcounter{listing}
        \textbf{Listing \thelisting:} Setting the Sampling Rate
        \label{lst:devguide:server:instr:6}
        \addcontentsline{lol}{listing}{\protect\numberline{\thelisting}Setting the sampling Rate}
    }
]
\inputminted[
    linenos,
    numbersep=4pt,
    style=solarizedlight,
]{javascript}{./code/serverinstructions/samplingrate.js}
\end{tcolorbox}

\subsection{Polling the Status of the Logger} % ------------------------------ %
\label{subsec:devguide:server:polling_logger}
% ---------------------------------------------------------------------------- %

This command  requests the current  logger status.  The response  contains all
the information the logger currently holds.

\begin{tcolorbox}[
    title={
        \refstepcounter{listing}
        \textbf{Listing \thelisting:} Polling the Status of the Logger
        \label{lst:devguide:server:instr:7}
        \addcontentsline{lol}{listing}{\protect\numberline{\thelisting}Polling the Status of the Logger}
    }
]
\inputminted[
    linenos,
    numbersep=4pt,
    style=solarizedlight,
]{javascript}{./code/serverinstructions/status.js}
\end{tcolorbox}

\subsection{Configuring the trigger} % --------------------------------------- %
\label{subsec:devguide:server:configuring_trigger}
% ---------------------------------------------------------------------------- %

This command  configures the currently  active trigger.  For now,  only rising
edge triggers  are supported  on the server  side. The logger  itself supports
more trigger types. For  more information on the logger's  capabilities, it is
recommended to peruse its code at~\cite{pita:github:huesser:zynq-logger}.

\code{channel} is the channel on which the trigger should be active. The level
at which the trigger shot fire is set with \code{level}, while \code{slope} is
the  minimum slope  the  curve needs  to have  to  trigger. The hysteresis  is
configured for  all triggers at  once and makes  sure there is  no accidential
trigger. It is the variance the signal can have until a trigger is armed.

\begin{tcolorbox}[
    title={
        \refstepcounter{listing}
        \textbf{Listing \thelisting:} Configuring the Trigger
        \label{lst:devguide:server:instr:8}
        \addcontentsline{lol}{listing}{\protect\numberline{\thelisting}Configuring the Trigger}
    }
]
\inputminted[
    linenos,
    numbersep=4pt,
    style=solarizedlight,
]{javascript}{./code/serverinstructions/trigger.js}
\end{tcolorbox}

%>>>
% ==============================================================================
%
%                                  S C O P E
%
% ==============================================================================
\chapter{Scope} % <<< -------------------------------------------------------- %
\label{ch:devguide:scope}
% ---------------------------------------------------------------------------- %

%\section{Build environment}
%\label{sec:devguide:scope:build_environment}

To set up the scope project, the build box is not needed. Still, the build box
does  have the  necessary  tools installed,  should the  need  arise for  some
reason.  For development, a local install is advised, however.

This  requires installing  yarn  and  nodejs version  8.0  or newer.   Package
managers for  Unix systems usually  already have  packages for these  tools in
their repositories. For  Windows, the  official installers  can be  used.  For
Once yarn is set up, the project dependencies can be installed using
\begin{commandshell}
    cd ~/repo/scope/
    yarn install
\end{commandshell}
\noindent The yarn  project comes with two main  build configurations; one for
deployment and one for running a debugging webserver. Building is done with
\begin{commandshell}
    cd ~/repo/scope/
    yarn build
\end{commandshell}

This leaves all the built files in \code{~/repo/scope/build}. They can then be
copied to any directory  served by a webserver, to be provided  to a client by
said server.

Running the debugging webserver brings the benefit of automatic rebuilding and
reloading in  the browser  after rebuilding. The webserver  can be  started in
debugging mode with
\begin{commandshell}
    cd ~/repo/scope/
    yarn watch
\end{commandshell}
The webpage  can now  be reached on  \code{http://localhost:8080} and  will be
automatically reloaded when yarn detecs a change and rebuilds the bundle.
%>>>

%^^A vim: foldenable foldcolumn=4 foldmethod=marker foldmarker=<<<,>>>
