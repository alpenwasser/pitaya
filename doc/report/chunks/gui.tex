To view measured data a graphical user interface (GUI in further text) was created. It can receive the recorded samples over the network and display them on a canvas. Furthermore it manages triggers and does a lot of math to get more specific metrics of a signal.
In this section the requirements for this piece of software, the design choices and the implementation details are discussed.

\subsection{Requirements}

The requirements for the GUI were given by the scope of Prof. Gut's 'Spektrum Analyzer' written in Java.
The task description required the new GUI to have the same features as the old one plus as many more as possible.

The requirements were as follows

\begin{itemize}
    \item Receive data over the network.
    \item Display received data in time as well as fequency space.
    \item Calculate the RMS power density in the signal.
    \item Calculate the THD ratio of the signal.
\end{itemize}

\subsection{Design Choices}

There is a wealth of programming languages to choose from. And there is as many libraries helping with graphics and networking as well for most of those languages.

\begin{table}
\begin{centering}
\setlength{\extrarowheight}{2pt}
\begin{tabular}{*{10}{c|}}

    \multicolumn{2}{c}{}        & \multicolumn{2}{c}{}\\\cline{3-10}%
    %% Header row
    \multicolumn{1}{c}{} &      & \parbox[t]{2mm}{\rotatebox[origin=c]{90}{Open Standard}}%
                                & \parbox[t]{2mm}{\rotatebox[origin=c]{90}{Networking}}%
                                & \parbox[t]{2mm}{\rotatebox[origin=c]{90}{Graphics}}%
                                & \parbox[t]{2mm}{\rotatebox[origin=c]{90}{Widespread}}%
                                & \parbox[t]{2mm}{\rotatebox[origin=c]{90}{User-Friendly}}%
                                & \parbox[t]{2mm}{\rotatebox[origin=c]{90}{Easy To Use}}%
                                & \parbox[t]{2mm}{\rotatebox[origin=c]{90}{Familiarity With The Language}}%
                                & \parbox[t]{2mm}{\rotatebox[origin=c]{90}{Total}} \\\cline{2-10}
                    & Rust      & 6 & 6 & 2 & 3 & 5 & 3 & 3 & 28\\\cline{2-10}
                    & C++       & 6 & 6 & 5 & 6 & 5 & 4 & 4 & 36\\\cline{2-10}
                    & Java      & 1 & 6 & 5 & 6 & 5 & 5 & 3 & 31\\\cline{2-10}
                    & Python    & 6 & 6 & 5 & 5 & 5 & 6 & 6 & 39\\\cline{2-10}
                    & JavaScript& 6 & 4 & 6 & 6 & 6 & 6 & 6 & 40\\\cline{2-10}
\end{tabular}
\caption{Descision matrix to choose a programming language}
\end{centering}
\end{table}