% ==============================================================================
%
%                             U S E R   G U I D E
%
% ==============================================================================

Initializing the system and using  the oscilloscope is mostly straightforward,
even for  beginners. A few notes  on the initial  setup and the  GUI's overall
concept are outlined here to get the reader started. The oscilloscope has been
designed with high discoverability of its functionality in mind, so a complete
guide on every single  last button and menu is considered  beyond the scope of
this document.

% ==============================================================================
%
%                                  S E T U P
%
% ==============================================================================
\chapter{Setup} % <<< -------------------------------------------------------- %
\label{ch:userguide:setup}
% ---------------------------------------------------------------------------- %

Getting the system  up and running requires two  basic steps: Initializing the
STEMlab  itself,  and  connecting  to  it  from  a  web  browser  to  run  the
oscilloscope.  Assuming the  STEMlab board at hand has been  delivered with an
SD  card containing  the  correct Linux  image from  this  project, the procedure
for this is as follows:
\begin{enumerate}
    \item 
        Insert the SD card coming with the board (NOTE: Check that the SD card
        has the right side up to make pin contact).
    \item 
        Connect the board physically to the network.
    \item 
        Connect the board to the power supply.
    \item 
        Call    the    board's    IP    address    and    the    right    port
        (eg.    \code{https://10.84.130.54:50090})    in    a    browser    of
        your   preference   (for   best    results   use   Chrome   61.0   and
        above). Figure~\ref{fig:userguide:url} shows  an example  how to
        do this.
    \item   A  notice   that  a   popup   has  been   blocked  should   appear
        (Figure~\ref{fig:userguide:popup:warn}). Select   \emph{Always   allow
        popups from  this application.} or  similar and reload  the previously
        called   page. Figure~\ref{fig:userguide:popup:yes}  illustrates   the
        necessary setting.
    \item A popup  tab should now contain the scope  and automatically connect
        to the STEMlab's webserver. Figure~\ref{fig:userguide:running} shows a
        running scope.
\end{enumerate}

If it  is unknown  wether the  SD card  was delivered  with a  prebuilt image,
it  can  be  assumed  that  it  was  and  the  board  can  be  powered  on. If
the  LED  farthest away  from  the  ports (RJ45,  SD  card,  \ldots )  flashes
orange  with \SI{2}{\Hz},  the SD  card  is fine.   If  the SD  card does  not
contain a  prebuilt image,  Section~\ref{sec:devguide:fpga_toolchain:linux} in
the Developer's Guide explains how one can be acquired or built.

\begin{figure}
    \centering
    \includegraphics[width=0.75\textwidth]{images/userguide/url}
    \caption[Entering the URL]{%
        Using a browser and the correct URL to run the scope application on
        the STEMlab.
    }
    \label{fig:userguide:url}
\end{figure}

\begin{figure}
    \centering
    \includegraphics[width=0.75\textwidth]{images/userguide/popup_warn}
    \caption[The popup warn popup]{%
        The browser warns about a popup the site has tried to open.
    }
    \label{fig:userguide:popup:warn}
\end{figure}

\begin{figure}
    \centering
    \includegraphics[width=0.75\textwidth]{images/userguide/popup_yes}
    \caption[Accept popups in the future]{%
        Let the browser accept popups in the future.
    }
    \label{fig:userguide:popup:yes}
\end{figure}

\begin{figure}
    \centering
    \includegraphics[width=0.75\textwidth]{images/userguide/running}
    \caption[Running the scope]{%
        The running scope after everything has been set up properly.
    }
    \label{fig:userguide:running}
\end{figure}

%>>>
% ==============================================================================
%
%                              O P E R A T I O N
%
% ==============================================================================
\chapter{Operation} % <<< ---------------------------------------------------- %
\label{ch:userguide:operation}
% ---------------------------------------------------------------------------- %

The UI to operate the measuring device is presented as seen in Figure~\ref{fig:userguide:screenshot}.

\section*{Zooming and Paning}

To zoom use the vertical and horizontal scrolling function of the mouse or trackpad.
To pan, click and drag the signal. If you are dragging a timetrace it will automatically move the trigger location.

The two arrow buttons in the general pref pane resize the signal to the visible are.

\section*{Triggering}

To set the trigger level, move the drawn trigger level by clicking and dragging it or enter the level in numbers on the pref pane of the triggering time trace.

\section*{Markers}

There is four markers by default. Two mark the start and stop of the area over which power is integrated and two mark the ara that contain the signal for the automatic SNR calculation.
Markers can be dragged and moved by click and drag. The number at the bottom of each marker shows the frequency it is at.

The plus button in the general pref pane adds a new marker which can be used to mark certain frequencies.

To move the SNR markers, the SNR mode has to be set from auto to manual mode in the corresponding fft trace pref pane.

\section*{Sampling Rate}
To configure the sampling rate, use the select at the top of the general pref pane.

\section*{Line Width}
To improve readability, some people will like a thicker line width. To set the linewidth use the input at the top of the general pref pane.

\begin{figure}
    \centering
    \includegraphics[width=1\textwidth]{images/gui/scope}
    \caption[The scope application]{%
        The scope application in it's current state, displaying time and FFT data.%
    }
    \label{fig:usegruide:screenshot}
\end{figure}

%>>>

%^^A vim: foldenable foldcolumn=4 foldmethod=marker foldmarker=<<<,>>>
