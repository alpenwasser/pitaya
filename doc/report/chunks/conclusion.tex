% ==============================================================================
%
%                            C O N C L U S I O N S
%
% ==============================================================================
\chapter{Conclusions} % ------------------------------------------------------ %
\label{ch:conclusions}
% ---------------------------------------------------------------------------- %
\enlargethispage{6ex}

The project has been a success. We  can present six chains with sampling rates
from  \SI{25}{\mega\Hz}  to  \SI{50}{\kilo\Hz}. While  not  perfect  in  every
aspect,  their passbands  are mostly  linear,  and aliasing  attenuation is  a
minimum of \SI{60}{\dB}.

A toolchain to design filters is available; it can quickly specify a number of
filters to a wide range of specifications.  The results then allow to converge
on  the  desired implementation. This  enables  even  users without  extensive
filter designing experience to devise filter chains.  Implementing the filters
is accomplished  through a  custom FPGA  toolchain which  allows for  easy and
reliable creation of a bitstream by way of Tcl scripts.

To read data from  the FPGA and provide it to the user  via Ethernet, a server
application running on an embedded GNU/Linux  has been developed, along with a
toolchain for compilation of the operating system.  Visualizing and analyzing
the  data  is  accomplished  with a  newly  developed  JavaScript  application
which  can run  in  any  modern browser,  ensuring  compatibility across  many
platforms. Data  access through  other  programs like  Matlab  is also  easily
possible.

There were a few notable  challenges during development: Soon after launch, it
was discovered  that the existing  code base from the  STEMlab was not  at the
time a  viable route; this necessitated  the expansion of the  project's scope
far  beyond  initial  plans. Secondly,  ensuring that  the  correct  bits  are
propagated through  the filter chains  turned out  to be a  highly non-trivial
task, requiring  many hours of simulations  and verification. Lastly, although
the  math behind  calibrating the  oscilloscope is  relatively straightforward
in  theory, correctly  applying  it in  practice  revealed some  unanticipated
subtleties which need to be accounted for to obtain correct results.

If  this  project  were  to  be   continued,  we  would  issue  the  following
recommendations: Firstly, performing  more and improved measurements  in order
to  tune the  device's  performance,  particularly for  the  sampling rate  of
\SI{25}{\MHz}.  Secondly, about \SI{15}{\percent}  of the FPGA's resources are
are  currently not  being utilized.   This was  done to  have some  leeway for
adding more features  or in case of resource problems  during development, but
those resources could be exploited  for adding functionality and/or increasing
performance. And  thirdly,  the  oscilloscope  can be  readily  expanded  with
additional functionality with little effort.

Overall, we are very happy the  results. We thank all our supporters for their
help,  both personal  and  technical, and  to whomever  may  find our  results
useful, we say:
\begin{quote}
\centering
\emph{%
    May  your harmonics  be undistorted,  your  noise floor  minute, and  your
    aliasing effects attenuated.%
}
\end{quote}
%^^A vim: foldenable foldcolumn=4 foldmethod=marker foldmarker=<<<,>>>
