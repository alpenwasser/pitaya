% ==============================================================================
%
%                                   F P G A
%
% ==============================================================================
\chapter{FPGA} % ------------------------------------------------------------- %
\label{ch:fpga}
% ---------------------------------------------------------------------------- %
% ==============================================================================
%
%                               O V E R V I E W
%
% ==============================================================================
%<<<
Broadly speaking,  the FPGA  firmware (the bitstream)  consists of  three main
components:  The ADC control logic, a  data acquisition core which writes data
to  RAM  and  is  responsible  for  triggering,  and,  most  importantly,  the
filter chains  that connect the two.   Figure~\ref{fig:fpga:structure} shows a
schematic of  how these components fit  together, and how they  are related to
the overall STEMlab system. This chapter first presents a rough outline of the
FPGA toolchain,  and then provides  more specific  information on each  of the
three  FPGA subsystems  in Sections~\ref{sec:fpga:adc},  \ref{sec:fpga:logger}
and~\ref{sec:fpga:chains}, respectively.

\begin{figure}
    \centering
    \tikzsetnextfilename{intro-system-overview}
\begin{tikzpicture}[
    rounded corners=1mm,
    node distance=5mm,
    fpgaComponent/.style={
        draw,
        minimum height=12ex,
        fill=q4,
    },
    SoCComponent/.style={
        minimum height=8ex,
        minimum width=6em,
        fill=ct2,
        draw=sqC,
        text=br2,
        font=\bfseries,
    },
]
    \small
    \sffamily

    \node[SoCComponent] at (0,0) (fpga) {FPGA};
    \node[SoCComponent,right=of fpga] (server) {Server};

    \begin{scope}[on background layer]
        \node[
            draw=black,
            fit=(fpga) (server),
            inner sep=2mm,
            align=left,
            text height=11ex,
            minimum height=16ex,
            fill=br1,
            text=black,
            font=\bfseries,
        ] (board) {Board};
    \end{scope}

    \node[
        right=10mm of board,
        minimum width=12ex,
        minimum height=8ex,
        fill=ct2,
        align=center,
        text=br2,
    ] (pcmonitor) {\bfseries Scope\\\\};

    \begin{scope}[on background layer]
        \node[
            draw=black,
            fit=(pcmonitor),
            minimum height=8ex,
            fill=black!60!white,
        ] (pcmonitorframe) {};
    \end{scope}

    \fill[br2] ($(pcmonitorframe.south) + (-1ex,0)$) --+ (0,-1ex) --+ (2ex,-1ex) --+ (2ex,0) -- cycle;
    \draw ($(pcmonitorframe.south) + (-1ex,0)$) --+ (0,-1ex);
    \draw ($(pcmonitorframe.south) + (1ex,0)$)--+ (0,-1ex);
    \draw ($(pcmonitorframe.south) + (1ex,0)$)--+ (0,-1ex);
    \draw[fill=br2] ($(pcmonitorframe.south) + (-6ex,-1ex)$) rectangle
          ($(pcmonitorframe.south) + (+6ex,-3ex)$);

    \draw[thick,-latex] (-18.0mm,0mm) -- ($(board.west) + (-1mm,0mm)$);
    \draw[thick,-latex] ($(board.east) + (1mm,0mm)$) -- ($(pcmonitorframe.west) + (-1mm,0mm)$);
    \draw[thick,-latex] ($(fpga.east) + (0.5mm,0mm)$) -- ($(server.west) + (-0.5mm,0mm)$);

    \begin{axis}[
        anchor=east,
        at={(-20mm,0mm)},
        width=20mm,
        height=10mm,
        axis line style={draw=none},
        tick style={draw=none},
        grid=none,
        ticks = none,
        xmin=0,
        xmax=360,
        ymin=-1.1,
        ymax=1.1,
        samples=500,
    ]
        \addplot[line cap=round,very thick,q1,domain=0:360,-]{sin(x)};
    \end{axis}

    \begin{axis}[
        anchor=south,
        yshift=1mm,
        at={(pcmonitor.south)},
        width=12mm,
        height=6mm,
        axis line style={draw=none},
        tick style={draw=none},
        grid=none,
        ticks = none,
        xmin=0,
        xmax=360,
        ymin=-1.3,
        ymax=1.3,
        samples=500,
    ]
        \addplot[very thick,q4,domain=0:360]{sin(x)};
    \end{axis}

    \node[font=\bfseries\footnotesize,text=black] at (-29mm,-8.95mm) {Signal};
\end{tikzpicture}

    \caption[System Schematic]{%
        Schematic  of  the  STEMlab  with   the  three  main  FPGA  subsystems
        highlighted in yellow.%
    }
    \label{fig:fpga:structure}
\end{figure}
%>>>
% ==============================================================================
%
%                              T O O L C H A I N
%
% ==============================================================================
\section{The Xilinx Toolchain} % <<< ----------------------------------------- %
\label{sec:fpga:toolchain}
% ---------------------------------------------------------------------------- %

The bitstream is compiled  using Vivado, Xilinx's own IDE, a  tool that can do
everything around  Xilinx FPGAs. It does  the crucial  parts right and  can be
interfaced  with using  Tcl. This  is very  convenient, as  a  project can  be
replicated idempotently\footnote{%
    Idempotence [\ldots] is the property  of certain operations in mathematics
    and computer science, that can  be applied multiple times without changing
    the result beyond the initial application~\cite{wiki:idempotence}.%
}%
whenever a rebuild is needed.

Whilst Vivado offers  a GUI to build  block designs, this process can be a bit
frustrating to the user due to various ``eccentricities'' of the application.
Therefore, we choose to use its Tcl API
to write scripts that create a new project and apply and connect all necessary
blocks. This avoids a lot of errors as  a bug in Vivado's user interface won't tamper
with the project.
It  also enables  us  to use  version control tools for  the  project as  Vivado
projects  create  a  lot  of  files which  often  clash  in  very  simple
versioning operations. Using Tcl scripts which create and configure the project,
and leaving everything else out of the repository, avoids this hassle.
Tcl also allows  the creation of sub-blocks: One can group  blocks together and
insert  them multiple  times  with  little effort;  a  feature which  Vivado's
graphical front-end apparently does not offer.
More    on    the    Tcl    API     and    Tcl    itself    can    be    found
in~\cite{xilinx:vivado-tcl-command-reference-guide},
\cite{xilinx:vivado-design-suit-user-guide:using-tcl-scripting},
and~\cite{tcl-exchange}.

The final advantage of the Tcl API to  be mentioned here is that it allows the
creation of  the entire project, the  block design, perform the  synthesis and
implementation, build  a bitstream, as well  as the board support  package and
first stage  bootloader, all in a  single sequence of automated  tasks without
the need  for manual  intervention. Since this  tends to take  quite a  lot of
time, that is a significant advantage.

%>>>
% ==============================================================================
%
%                               A D C   C O R E
%
% ==============================================================================
\section{The ADC Core} % <<< ------------------------------------------------- %
\label{sec:fpga:adc}
% ---------------------------------------------------------------------------- %

The ADC  core is a simple  piece of logic  that interfaces with the  FPGA pins
which  are  connected  to  the  STEMlab's ADC. It  reads  the  ADC's  unsigned
\SI{14}{\bit}  values and  converts  them to  \SI{16}{\bit}  signed format  by
adding an offset of $2^{13}$ and performing a \SI{2}{\bit} sign extension. The
resulting numbers are then provided over an AXI Stream bus interface, which is
also  used by  all the  filters.  The  core is  used from  the git  repository
provided by Pavel Demin~\cite{pita:github:pitaya-notes} More on his repository
and project can be read in Section~\ref{subsec:concept:fpga_components}.

%>>>
% ==============================================================================
%
%                            L O G G E R   C O R E
%
% ==============================================================================
\section{The Logger Core} % <<< ---------------------------------------------- %
\label{sec:fpga:logger}
% ---------------------------------------------------------------------------- %

The logger core (logger  in further text) is a piece of  VHDL code that stores
samples it  gets from a  source in  a ringbuffer in  the RAM. It is packaged  as a
Vivado IP core and can be seamlessly integrated into the project.
The logger originated  from an earlier project\cite{huess-schnid}. 
In addition  to logging data  to RAM, it can  also be programmed  with various
triggers. It reads  instructions from  a BRAM  on the  FPGA and  iterates over
them. Having reached  the last one, it  issues an IRQ signal,  signalizing the
end of the transcription.

The logger's original  implementation features eight channels with  a width of
\SI{14}{\bit}, padded to \SI{16}{\bit} in  order to simplify data transmission
in byte-sized chunks. Each two channels require one clock cycle to store
a sample.

In order to take advantage of the  fact that additional bits can be ``won'' by
oversampling a signal, this projects implements a new configuration, which can
process full \SI{16}{\bit}  values, a gain in two bits  over the ADC's output.
The penalty for this is one additional  clock cycle of delay, since the adders
and  comparators cannot  match  the timing  requirements  with two  additional
carries.  Since this project aims to optimize for lower-frequency signals, the
resulting additional delay of \SI{8}{\nano\second}  is acceptable and will not
be an issue in practice.

The  logger  core comes  with  a  kernel  module  that provides  a  convenient
interface from the ARM core. This avoids having to manually program the logger
core.

%>>>
% ==============================================================================
%
%                          F I L T E R   C H A I N S
%
% ==============================================================================
\section{The Filter Chains} % <<< -------------------------------------------- %
\label{sec:fpga:chains}
% ---------------------------------------------------------------------------- %
% ==============================================================================
%
%                               O V E R V I E W
%
% ==============================================================================
%<<<
The filter chains are  the most crucial part of the project  and also the most
delicate  one  as  simple  mistakes  can cost  several  decibels  of  SNR  and
make  the signal  leaving  the filter  worse  than it  came  into the  filter,
instead  of improving  it.  The  logical structure  of the  chains can  be seen
in  Figure~\ref{fig:fdesign:chain_concept}  and  the rationale  behind  it  is
explained  in the  respective Section~\ref{sec:fdesign:filter_specifications}.
For the  detailed implementation,  it is  advised to look  at the  project in
Vivado  itself, as  the block  design is  impractically large  to be  put onto
paper, thus it is omitted in this report.

This section first gives a few notes on the two most important building blocks
in the FPGA design: The CIC and FIR  compilers by Xilinx.  After that, two key
points which are particularly challenging  when implementing filter chains are
elaborated  upon: Propagating  the  correct  bits  through  the  cascade,  and
adjusting  the gain  correctly  in order  to exploit  the  available bits  for
maximum dynamic range.

%>>>
% ==============================================================================
%
%                       F I L T E R   C O M P I L E R S
%
% ==============================================================================
\subsection{Filter Compilers} % <<< ------------------------------------------ %
\label{subsec:fpga:filter_compilers}
% ---------------------------------------------------------------------------- %

The  basic building  blocks of  the  filter chains  are FIR  and CIC  filters,
which  are  based on  ready-to-use  blocks  by  Xilinx. Vivado's CIC  and  FIR
compilers natively  utilize the  DSP slices  to a maximum  extent and  make it
very  easy  (at  least  in  theory)  to  implement  a  Matlab-designed  filter
in  hardware. Those  IPs  are  described   very  thoroughly  in  the  official
documentation~\cite{xilinx:fir-compiler},~\cite{xilinx:cic-compiler}.

The  FIR filter  compilers  are  configured using  a  set  of coefficients  in
\code{double} format (exported from Matlab, or any other filter design tool of
choice). The compiler quantizes the  coefficients with maximum precision using
a \SI{16}{\bit} fixed  point number (this can be changed  from the default but
should be  left to the compiler  for best results). It does  so by determining
the index of the MSB required to  represent the biggest coefficient in the set
using \num{16}\,bits downwards.

As an example, take the biggest coefficient to be $c_\mathrm{max} = 0.23$. The
bit at  index \num{-2}\footnote{The bit at  index $-n$ is the  $2^{-n}$ valued
bit.} becomes  the sign  as its  value (\num{0.25}) is  not needed  to display
$c_\mathrm{max}$. Thus the  number is  said to have  \num{16}\si{bits} overall
and \num{17} fractional bits. While this  might seem a bit counterintuitive at
first, it simply means  that the LSB is the one at index  \num{-17} and it has
\num{16}\,bits, meaning the sign is at the bit index \num{-2}.

The  compiler  then also  takes  the  specified  input bit  configuration  and
determines the  required output  configuration to  guarantee no  overflows and
achieve maximum  performance.  What is important  here is that the  output bit
width  is  the same  that  is  needed to  guarantee  no  overflows inside  the
filter. This means that  the user has to  be aware of the maximum  gain of the
designed filter  and determine on  their own which  bits are important  at the
output.

%>>>
% ==============================================================================
%
%                       B I T   P R O P A G A T I O N
%
% ==============================================================================
\subsection{Bit Propagation Through the Filter Chains} % <<< ----------------- %
\label{subsec:fpga:bit_propagation}
% ---------------------------------------------------------------------------- %

For  this  application, only  \SI{16}{\bit}  values  are stored. However,  the
filters generate far wider numbers at  their outputs.  This means that many of
them are  discarded at the  output.  To make sure  that no important  bits are
truncated, the chosen  input format for the filters  is \code{17.7}, resulting
in \num{24} total bits. The MSB should always remain just a sign extend of the
sign actually  residing at bit  \num{15}. The \num{7} fractional bits  are cut
off at the end of the filter chains but are still important for more precision
so less  rounding and/or  truncation errors are  introduced inside  the filter
chain.

As values can use up significantly  more than \num{24}\,bits inside the filter
due to bit growth, it has to be ensured that no overflows happen, resulting in
greater  bit widths at  the output  of the  filter.  It  is important  that the
location  of the  decimal point  is always  tracked and  remains in  its right
place.  Figure~\ref{fig:fpga:bitflow} depicts the flow through an example chain
but represents the general case in out block design.

\begin{figure}
    \centering
    \tikzsetnextfilename{bitflow}
\begin{tikzpicture}[
    % rounded corners=1mm,
    every node/.append style={
        minimum height=12ex,
        scale=0.6
    },
    scale=0.6,
    node distance=5mm,
]
    \small
    \sffamily

    % \draw[step=1cm,gray,very thin] (-0.9,-12.9) grid (26.9,18.9);
    \draw[q2,ultra thick,dashed] (0,8) rectangle (2,8.5);
    \draw[q2,ultra thick,dashed] (0,7.5) rectangle (2,8);
    \draw[q2,ultra thick,dashed] (0,7) rectangle (2,7.5);
    \fill[q2] (0,6.5) rectangle (2,7);
    \draw [->,ultra thick] (0.3,6.75) -- (0.3,8.25);
    \fill[q1] (0,0) rectangle (2,6.5);
    % \fill[white] (1,0) circle (0.25cm);
    \node[] at (1,-7.5) (adc) {ADC};
    \node[] at (1,8.25) {1 bit};
    \node[] at (1,7.75) {1 bit};
    \node[] at (1,7.25) {1 bit};
    \node[white] at (1,6.75) {1 bit};
    \node[white] at (1,3.25) {13 bits};

    \fill[q4] (4,0) rectangle (6,-3.5);
    % \fill[white] (5,0) circle (0.25cm);
    \node[black] at (5,-7.5) (zero) {ZERO};
    \node[white] at (5,-1.75) {7 bits};

    \fill[q2] (8,6.5) rectangle (10,8.5);
    \fill[q1] (8,0) rectangle (10,6.5);
    \fill[q4] (8,0) rectangle (10,-3.5);
    % \fill[white] (9,0) circle (0.25cm);
    \node[black] at (9,-7.5) (fir1in) {FIR 1[in]};
    \node[white] at (9,7.5) {4 bits};
    \node[white] at (9,3.25) {13 bits};
    \node[white] at (9,-1.75) {7 bits};

    \fill[q1] (10,0) rectangle (12,9.5);
    \fill[q4] (10,0) rectangle (12,-6.5);
    % \fill[white] (11,0) circle (0.25cm);
    \node[black] at (11,-7.5) (fir1out) {FIR 1[out]};
    \node[white] at (11,4.75) {19 bits};
    \node[white] at (11,-3.25) {13 bits};
    
    \fill[q1] (14,0) rectangle (16,8.5);
    \fill[q4] (14,0) rectangle (16,-3.5);
    % \fill[white] (15,0) circle (0.25cm);
    \node[black] at (15,-7.5) (fir2in) {FIR 2[in]};
    \node[white] at (15,3.25) {17 bits};
    \node[white] at (15,-1.75) {7 bits};

    \fill[q1] (16,0) rectangle (18,9.5);
    \fill[q4] (16,0) rectangle (18,-6.5);
    % \fill[white] (17,0) circle (0.25cm);
    \node[black] at (17,-7.5) (fir2out) {FIR 2[out]};
    \node[white] at (17,4.75) {19 bits};
    \node[white] at (17,-3.25) {13 bits};

    \fill[q1] (20,0) rectangle (22,7);
    \fill[q4] (20,0) rectangle (22,-1);
    % \fill[white] (21,-1) circle (0.25cm);
    \node[black] at (21,-7.5) (logger) {Final Value};
    \node[white] at (21,3) {14 bits};
    \node[white] at (21,-0.5) {2 bits};

    \draw[->] (adc) -- (zero);
    \draw[->] (zero) -- (fir1in);
    \draw[->] (fir1out) -- (fir2in);
    \draw[->] (fir2out) -- (logger);

    % ----------- L E G E N D -------------
    \draw[black,dashed] (2,10.25) -- (4,10.25);
    \node[black] at (5,10.25) (cut) {Cuttoff};
    \fill[q1] (6,10) rectangle (8,10.5);
    \node[black] at (9,10.25) (full) {Full bits};
    \fill[q4] (10,10) rectangle (12,10.5);
    \node[black] at (13,10.25) (fractionals) {Fractional bits};
    \fill[q2] (14,10) rectangle (16,10.5);
    \node[black] at (17,10.25) (extends) {Sign extension bits};

    \draw [white,dashed,ultra thick] (0,0) -- (22,0);
    \draw [black,dashed] (10,-3.5) -- (16,-3.5);
    \draw [black,dashed] (16,-1) -- (22,-1);
    \draw [black,dashed] (16,7) -- (22,7);
    \draw [black,dashed] (10,8.5) -- (16,8.5);
\end{tikzpicture}

    \caption[Bit Flow in Filter Chain]{%
        The flow of  the bits in a filter chain. The  bits are shifted through
        horizontally. Every bit that does not fit into the numerical width of
        the next stage will be cut off, starting with those furthest away from
        the sign. At the  sign extend, the bit is replicated  \num{4} times and
        handed through to the next  stage. The ZERO stage simply holds \num{7}
        \code{'0'} bits to pad the number coming from the ADC on its lower end
        before it goes  into the first filter. Inside the filter  the bits can
        grow, so those are not shifted  and cut off but rather resized towards
        the output.%
    }
    \label{fig:fpga:bitflow}
\end{figure}

%>>>
% ==============================================================================
%
%                 M A X I M I Z E   D Y N A M I C   R A N G E
%
% ==============================================================================
\subsection{Ensuring Maximum Dynamic Range} % <<< ---------------------------- %
\label{subsec:fpga:maximize_dynamic_range}
% ---------------------------------------------------------------------------- %

Of course it is important not to discard any MSBs (or signs) because otherwise
the signal (a sine wave in our example) will clip or, even worse, overflow and
wrap around. The same applies  to the case where too few  bits are cut off. If
the bit count is increased by one  to guarantee no overflow inside the filter,
but that bit is  not set at the output due to unity  gain, it will effectively
be lost  as it  will never  be used.  This  reduces the  maximum theoretically
achievable SNR by \SI{6}{\dB}, which is obviously highly undesirable.

To make  sure neither  of these faults  happens, it is  important to  have the
highest possible  filter gain at $G  \leq 1$. Furthermore, at the  end of each
filter chain,  the additional bits  must not be  carried over, but  rather the
initial \num{14}\,bits  before the  decimal point,  along with  two fractional
bits after the decimal point. This  yields the desired \SI{16}{\bit} value and
ensure no ``empty'' bits.

The FIR  compiler can avoid  those empty  bits by normalizing  the coefficient
such  that the  highest  gain (i.e.  the  top peaks  of  its passband  ripple)
is  at  exactly   \num{1}.   This  is  called   \emph{maximizing  the  dynamic
range}. Figure~\ref{fig:fpga:dynamicrange}  illustrates the  issue of  loosing
one bit.

One can  observe that  the sine in  Case \num{1} (top  plot) uses  the dynamic
range to a perfect extent as the full-scale sine has an amplitude of \num{31},
the maximum  value a \SI{6}{\bit}  \code{int} can hold. Case  \num{2} (middle
plot) shows  a sine  that has  been scaled  to \num{34}  and thus  requires an
additional bit.  But  because a \SI{7}{\bit} \code{int} can hold  values up to
\num{63}, most of the time the MSB  ends up not being used. With \SI{7}{\bit},
the highest $SNR_\mathrm{max}$ possible would be
\begin{align}
    %\mathrm{ENOB}             &= \log_2(130) \nonumber\\
    %\mathrm{SNR}_\mathrm{max} &= \SI{1.76}{\dB} + \mathrm{ENOB} \cdot \SI{6.02}{\dB} = 44.03
    \mathrm{ENOB}             &= \log_2(34) = 5.09 \nonumber \\
    \mathrm{SNR}_\mathrm{max} &= \SI{1.76}{\dB} + \mathrm{ENOB} \cdot \SI{6.02}{\dB} = 32.39
\end{align}
In   Case    \num{3},   the   dynamic    range   is   used   well    and   the
$\mathrm{SNR}_\mathrm{max}$ with a \SI{6}{\bit} \code{int} is
\begin{align}
    %\mathrm{ENOB}             &= \log_2(126) \nonumber\\
    %\mathrm{SNR}_\mathrm{max} &= \SI{1.76}{\dB} + ENOB \cdot \SI{6.02}{\dB} = 43.76
    \mathrm{ENOB}             &= \log_2(30)  = 4.91 \nonumber\\
    \mathrm{SNR}_\mathrm{max} &= \SI{1.76}{\dB} + \mathrm{ENOB} \cdot \SI{6.02}{\dB} = 31.30
\end{align}
%This yields a difference of only \SI{0.27}{\dB}. This example shows that it is
This yields a difference of only \SI{1.09}{\dB}\footnote{%
    While \SI{1.09}{\dB} might still seem rather large, keep in mind that with
    the  wider numbers  running in  the actual  filter chains,  the result  is
    significantly better than in this illustrative example.%
}.
This example shows that it is well  advised to scale the coefficient such that
they  don't ripple  around  \num{1}, but  rather that  the  maximum ripple  is
exactly \num{1} and not more.

Because if  it can be ensured  that no additional  MSB is used which  is empty
most of the  time, it is possible  use an additional LSB which  is always well
used and to effectively win a bit. So  in most cases when the coefficients are
designed to have a unity gain, close to \SI{6.02}{\dB} can be won by rescaling
the coefficients.

\begin{figure}
    \centering
    % \tikzsetnextfilename{dynamicrange}
\pgfplotsset{
    axis line style={black},
    every axis label/.append style={black},
    every tick label/.append style={black}  
  }
\begin{tikzpicture}[
    scale=1,
]
    \small
    \sffamily
    \pgfplotsset{every axis/.style={
            height=45mm,
            width=\textwidth,
            samples=300,
            %ymin=0,ymax=256,
            ymin=0,ymax=64,
            xmin=0,xmax=180,
            xtick={},
            xticklabels={},
            const plot mark mid,
            scaled ticks=false,
            xlabel={$t$},
            ylabel={$y$},
            ytick={0,16,32,64,128,256},
            yticklabels={0,16,32,64,128,256},
        },
    }

    \begin{axis}[
        title={Case 1: Full Scale Sine Input},
    ]
        %\addplot[thick,q1][domain=0:180]{126*sin(x)} node[right]{x(t)};
        %\draw[dashed,line width=.4pt] (0,128) -- (180,128);
        \addplot[thick,q1][domain=0:180]{31*sin(x)} node[right]{x(t)};
        \draw[dashed,line width=.4pt] (0,32) -- (180,32);
    \end{axis}
    \begin{axis}[
        title={Case 2: Sine Output After a Filter With Bad Dynamic Range Usage},
        at={(0cm,-60mm)},
    ]
        %\addplot[thick,q1][domain=0:180]{130*sin(x)} node[right]{x(t)};
        %\draw[dashed,line width=.4pt] (0,128) -- (180,128);
        \addplot[thick,q1][domain=0:180]{34*sin(x)} node[right]{x(t)};
        \draw[dashed,line width=.4pt] (0,32) -- (180,32);
    \end{axis}
    \begin{axis}[
        title={Case 3: Sine Output After a Filter With Good Dynamic Range Usage},
        at={(0cm,-120mm)},
    ]
        %\addplot[thick,q1][domain=0:180]{124*sin(x)} node[right]{x(t)};
        %\draw[dashed,line width=.4pt] (0,128) -- (180,128);
        \addplot[thick,q1][domain=0:180]{30*sin(x)} node[right]{x(t)};
        \draw[dashed,line width=.4pt] (0,32) -- (180,32);
    \end{axis}
\end{tikzpicture}

    \caption{An illustration of good and bad use of dynamic range.}
    \label{fig:fpga:dynamicrange}
\end{figure}

%>>>
% ==============================================================================
%
%  M E A N   E R R O R   A N D   V A R I A N C E   I N   C I C   F I L T E R
%
% ==============================================================================
\subsection{Errors Due to Truncation in the CIC Filter} % <<< ---------------- %
\label{subsec:fpga:errors_in_cic_filter}
% ---------------------------------------------------------------------------- %

As detailed in Section~\ref{subsubsec:cic:register_growth},  the high gain of
CIC filters generally requires discarding  bits at the filter's output. In our
implementation,  we discard  \SI{17}{\bit} at  the  output of  the $R=25$  CIC
filter, and \SI{26}{\bit}  at the output of the $R=125$  filter, in both cases
through  truncation. No bits  are discarded  at  the filter's  input, and  the
CIC  compiler ensures  that  its  internal widths  are  always sufficient  for
full  precision~\cite{xilinx:cic-compiler}. Therefore,  only  the  last  stage
introduces an error.

Hogenauer's formulas are used to determine the mean error and variance of both
CIC  filters. Interestingly, the  result  is identical  for both  filters. But
given that the output precision is  \SI{16}{\bit} in both cases, this actually
does make sense. The results are:
\begin{align}
    \mu_\mathrm{CIC25}       &= 0.5 \label{eq:fpga:cic:mu:cic25}  \\
    \sigma^2_\mathrm{CIC25}  &= 0.289\label{eq:fpga:cic:sigmasq:cic25} \\
    \mu_\mathrm{CIC125}      &= 0.5 \label{eq:fpga:cic:mu:cic125} \\
    \sigma^2_\mathrm{CIC125} &= 0.289\label{eq:fpga:cic:sigmasq:cic125}
\end{align}
The  calculations are  performed  by a  Matlab script  and  are therefore  not
further elaborated upon at this point.

%>>>
%>>>
%^^A vim: foldenable foldcolumn=4 foldmethod=marker foldmarker=<<<,>>>
