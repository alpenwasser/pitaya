\chapter{Theoretical Background} % <<< --------------------------------------- %
\label{ch:analog-to-digital_data_aquisition}
% ---------------------------------------------------------------------------- %

This chapter  will present a  brief synopsis on  some aspects of  digital data
acquisition from  an analog source  and the processing  of that data,  and how
those issues pertain to our project. It  is not intended to be a comprehensive
treatise on the subject but shall serve  as a short refresher. At its end, the
reader should  have sufficient insight  to understand the basic  motivation of
our project from a theoretical point of view.

TODO: references to more comprehensive literature.

\section{The Digital Signal Processing Chain}% <<< --------------------------- %
\label{sec:dsp_chain}

Digitally acquiring a signal generally requires at least the following steps:
\begin{itemize}\tightlist
        \item
            Passing the signal through an analog low-pass filter.
        \item
            Sampling and quantizing the filtered signal.
\end{itemize}

The   resulting   sequence  of   values   can   then  be   further   digitally
processed. The necessary  building blocks  for this  process are  portrayed in
Figure~\ref{fig:dspChain:blocks}.

\begin{figure}
    \centering
    \tikzsetnextfilename{dspChain}
\begin{tikzpicture}[
    dspBlock/.style={
        draw=sqC,
        rounded corners=1mm,
        fill=sq3,
        minimum height=3.5ex,
        minimum width=4em,
    },
    signalPath/.style={
        draw=q4,,
        fill=q4!50!white,
        circle,
        inner sep=0.3mm,
    }
]
    \coordinate (in) at (-1.5,0);
    \coordinate (out) at (7.5,0);

    \node[dspBlock] (LP)  at (0,0) {LP};
    \node[dspBlock] (ADC) at (3,0) {ADC};
    \node[dspBlock] (DSP) at (6,0) {DSP};

    \draw[-latex] (in)  -- (LP);
    \draw[-latex] (LP)  -- (ADC);
    \draw[-latex] (ADC) -- (DSP);
    \draw[-latex] (DSP) -- (out);

    \node[above      =1ex] (a1a) at (in)       {A};
    \node[above right=1ex] (a2a) at (LP.east)  {A};
    \node[above right=1ex] (d1a) at (ADC.east) {D};
    \node[above      =1ex] (d2a) at (out)      {D};

    \node[signalPath,below=4ex] (a1b) at (a1a) {\footnotesize 1};
    \node[signalPath,below=4ex] (a2b) at (a2a) {\footnotesize 2};
    \node[signalPath,below=4ex] (d1b) at (d1a) {\footnotesize 3};
\end{tikzpicture}

    \caption[The DSP Chain]{%
        The basic  building blocks of the  DSP chain from its  analog input to
        its digitally processed output.\protect\newline
        From  left  to  right: The  analog low-pass  filter  (\emph{LP}),  the
        analog-to-digital  converter (\emph{ADC}),  and  an arbitrary  digital
        signal processing  system for further  processing of the  ADC's output
        (\emph{DSP}).%
    }
    \label{fig:dspChain:blocks}
\end{figure}

\begin{figure}
    \centering
    \tikzsetnextfilename{dspChainTimeDomain}
\begin{tikzpicture}[
    signalPath/.style={
        draw=q4,,
        fill=q4!50!white,
        circle,
        inner sep=0.3mm,
    }
]
     \pgfplotsset{every axis/.style={
            width=30mm,
            height=25mm,
            grid=none,
            axis line style={draw=none},
            tick style={draw=none},
            ticks = none,
        }
    }
    \begin{axis}[
        at = {(0,0)},
    ]
        \addplot[thick,q1,-] 
            table[x=t,y=y,col sep=comma] {images/dspChain/noisySine.csv};
    \end{axis}

    \begin{axis}[
        at = {(40mm,0)},
    ]
        \addplot[thick,q1,-,smooth] 
            table[x=t,y=y,col sep=comma] {images/dspChain/smoothSine.csv};
    \end{axis}

    \begin{axis}[
        at = {(80mm,0)},
    ]
        \addplot[q1,ycomb,mark=o,mark size=1.0]
            table[x=t,y=y,col sep=comma] {images/dspChain/sampledSine.csv};
    \end{axis}
    \draw[-latex] (2.9,1.25) -- (3.95,1.25);
    \draw[-latex] (6.9,1.25) -- (7.95,1.25);

    \node[signalPath] at (1.50,2.75) {\footnotesize 1};
    \node[signalPath] at (5.50,2.75) {\footnotesize 2};
    \node[signalPath] at (9.50,2.75) {\footnotesize 3};
\end{tikzpicture}

    \begin{tikzpicture}
     \pgfplotsset{every axis/.style={
            height=30mm,
            width=40mm,
            grid=none,
            axis line style={draw=none},
            tick style={draw=none},
            ticks = none,
        }
    }
    \begin{axis}[
        at = {(0,0)},
    ]
        \addplot[-] table[x=t,y=y,col sep=comma] {images/dspChain/spectrumFlat.csv};
    \end{axis}

    \begin{axis}[
        at = {(40mm,0)},
    ]
        \addplot[-,smooth] table[x=t,y=y,col sep=comma] {images/dspChain/spectrumLP.csv};
    \end{axis}

    \begin{axis}[
        at = {(80mm,0)},
    ]
        \addplot[-,smooth] table[x=t,y=y,col sep=comma] {images/dspChain/spectrumSampled.csv};
    \end{axis}
    \draw[-latex] (2.5,0.75) -- (3.85,0.75);
    \draw[-latex] (6.5,0.75) -- (7.85,0.75);

    \node[draw,inner sep=0.3mm,circle] at (1.25,1.75) {\footnotesize 1};
    \node[draw,inner sep=0.3mm,circle] at (5.25,1.75) {\footnotesize 2};
    \node[draw,inner sep=0.3mm,circle] at (9.25,1.75) {\footnotesize 3};
\end{tikzpicture}

    \caption[Signals Passing Through the DSP Chain (Simplified)]{%
        Simplified  time-domain  (top)   and  frequency-domain  (bottom)  view
        of  the  signal  at  different  stages on  its  way  through  the  DSP
        chain. The circled  numbers correspond  to the  stages as  outlined in
        Figure~\ref{fig:dspChain:blocks}.\protect\newline
        Stage  1 is  the  signal  before passing  through  the input  low-pass
        filter,  with  a  significant   amount  of  high-frequency  noise. The
        low-pass filter removes any  frequency components above ${f_s}/{2}$ in
        an ideal scenario  (in reality, it merely attenuates them,  as we will
        see later), resulting in the signal at stage 2.\protect\newline
        After having been filtered, the  ADC samples and quantizes the signal,
        yielding  a sequence  of  values, as  schematically  portrayed in  the
        rightmost picture for stage 3.\protect\newline
        Note that  due to the sampling  process, the spectrum of  the filtered
        signal is  repeated at intervals of  $f_s$. This is the source  of the
        issue of \emph{aliasing}.%
    }
    \label{fig:dspChain:signals}
\end{figure}

Of particular  interest for our  application is  what happens in  the ADC. The
quantization process converts a  value-continuous signal into a value-discreet
one,  with its  resolution being  a specification  of the  ADC which  is being
used. As an example, the ADC in our system has a resolution of \num{14}\,bits,
meaning it can divite its valid  input range into \num{16384} values. Given an
input range of \SI{2}{\volt} $V_{PP}$, that equates to a resolution of roughly
\SI{122}{\micro\volt} (in theory). This quantization  process is the source of
what is generally  known as \emph{quantization noise}. For more  on the topic,
see \code{TODO}.

TODO: Check numbers. Give references for further reading.

Besides the quantization,  the other step happening in the  ADC is sampling; a
time-continuous signal is converted into a series of time-discreet values. The
time between those values is known as \emph{sampling time}, its inverse is the
\emph{sampling  frequency}. Note that  usually  these are  constant, at  least
during the  time where the signal  is measured. This need not  strictly be the
case  in theory  though. In our  system, this  sampling frequency  is a  fixed
property of the ADC, and is \SI{125}{\mega\hertz}.

The sampling  step lies  at the  core of  the problem  our project  intends to
address:  \emph{aliasing}. Therefore, we  will take  a  closer look  at a  few
consequences  of the  sampling  process, and  how they  are  relevant to  this
project.

Descriptively, the sampling  process can be thought of as  looking at a signal
at  specific points  in  time and  capturing  its value. Mathematically,  this
amounts to multipliying the  signal with a series of Dirac  pulses in the time
domain,  and  convolving with  a  series  of  Dirac  pulses in  the  frequency
domain\footnotemark.
\footnotetext{%
    \emph{Pro memoria}: A series of Dirac pulses  in the time domain has as its
    spectrum a series of Dirac pulses as well.%
}.
This  convolution  in   the  frequency  domain  lies  at  the   heart  of  the
problem   of  aliasing,   because  it   results  in   the  incoming   signal's
spectrum  being  repeated  at  intervals  of  $f_s$  (see  also:  stage  3  in
Figure~\ref{fig:dspChain:signals}). This is no problem as long as the spectrum
of the incoming signal fits within  the boundaries set by this repetition. But
if  the  spectrum   of  the  incoming  singal  is  too   broad,  two  or  more
recurrences  of  the spectrum  will  overlap. This  effect is  highlighted  in
Figure~\ref{fig:aliasing:band}.

\begin{figure}
    \centering
    \begin{tikzpicture}
     \pgfplotsset{every axis/.style={
            height=40mm,
            width=120mm,
            grid=none,
            axis x line=bottom,
            axis y line=middle,
            xtick={-400,-200,0,200,400},
            xticklabel style={font=\footnotesize},
            xticklabels={
                $-f_s$,
                $-\frac{f_s}{2}$,
                $0$,
                $\frac{f_s}{2}$,
                $f_s$,
            },
            yticklabels={},
            ytick style={draw=none},
            xlabel={$f$},
            xlabel style={
                at={(ticklabel* cs:1.05)},
                anchor=east,
            },
            ymin=0,
            ymax=1,
        }
    }
    \begin{axis}[
        at = {(0mm,35mm)},
            xmin=-800,
            xmax=800,
            ymin=0,
            ymax=1.1,
    ]
        \addplot[smooth,-] table[x=f,y=Y,col sep=comma] {images/aliasing/bandNoAliasing.csv};
        \addplot[ycomb,mark=none,gray,style={dotted,thick}] coordinates {
            (-400,1.1)
            (-200,1.1)
            ( 200,1.1)
            ( 400,1.1)
        };
        \addplot[ycomb,mark=none,gray,style={dotted,thick}] coordinates {
            (-400,1.1)
            (-200,1.1)
            ( 200,1.1)
            ( 400,1.1)
        };
    \end{axis}

    \begin{axis}[
        at = {(0mm,0mm)},
        xmin=-800,
        xmax=+800,
        ymin=0,
        ymax=1.1,
    ]
        \addplot[name path=zero,black,very thin] coordinates {
            (-800,0) 
             (800,0)
        };

        \addplot[smooth,       -] table[x=f,y=Y,col sep=comma] {images/aliasing/bandAliased1.csv};
        \addplot[smooth,name path=uno,-] table[x=f,y=Y,col sep=comma] {images/aliasing/bandOverlap1.csv};
        \addplot[smooth,       -] table[x=f,y=Y,col sep=comma] {images/aliasing/bandAliased2.csv};
        \addplot[smooth,name path=due,-] table[x=f,y=Y,col sep=comma] {images/aliasing/bandOverlap2.csv};
        \addplot[smooth,       -]            table[x=f,y=Y,col sep=comma] {images/aliasing/bandAliased3.csv};

        \addplot[q2] fill between[
            of = uno and zero,
        ];
        \addplot[q2] fill between[
            of = due and zero,
        ];
        \addplot[ycomb,mark=none,gray,style={dotted,thick}] coordinates {
            (-400,1.1)
            (-200,1.1)
            ( 200,1.1)
            ( 400,1.1)
        };
    \end{axis}
\end{tikzpicture}

    \caption[Aliasing Illustrated via Signal Frequency Band]{%
        Simplified view  of a signal  which does not produce  aliasing between
        its recurrences  in the  frequency spectrum  (top), contrasted  with a
        signal whose  frequency band  has components  above half  the sampling
        frequency,  resulting   in  aliasing;  its  spectral   copies  overlap
        (highlighted areas in the bottom plot).%
    }
    \label{fig:aliasing:band}
\end{figure}

This overlap results in two primary problems:
\begin{itemize}\tightlist
    \item
        The digital  signal may not  be unambiguously reconstructable  into an
        analog signal, if that is intended.
    \item
        Frequencies  may occur  in the  digital  signal stream  which are  not
        actually  present  in  the  original  signal. This  problem  is  often
        referred to  as the  \emph{folding back} of  frequency components. See
        Figure~\ref{fig:aliasing:dirac} for an illustration  of how this might
        look.

        This problem is of particular interest  to our application, as we will
        see later.
\end{itemize}

\begin{figure}
    \centering
    \tikzsetnextfilename{aliasingDirac}
\begin{tikzpicture}
     \pgfplotsset{every axis/.style={
            height=30mm,
            width=\textwidth,
            grid=none,
            axis x line=bottom,
            axis y line=middle,
            %axis line style={draw=none},
            %tick style={draw=none},
            %ticks = none,
            yticklabels={},
            ytick style={draw=none},
            ymin=0,
            ymax=1.33,
            xlabel={$f$},
            xlabel style={
                at={(ticklabel* cs:1.05)},
                anchor=east,
            },
            xmin=-8,
            xmax=8,
            ylabel={$\delta(f)$},
            legend columns=3,
        },
        every axis legend/.append style={
            %at={(1.05,1.1)}, % if attached to top plot
            at={(0.5,2.6)},  % if attached to bottom plot
            anchor=south,
            font=\footnotesize,
            cells={anchor=east},
        },
    }
    % fs = 4
    % fs/2 = 2
    
    % Dirac at f=1
    \begin{axis}[
        at = {(0,0)},
        xtick={-6,-4,-2,-1,0,1,2,4,6},
        xticklabels={
            $-\frac{3f_s}{2}$,
            $-f_s$,
            $-\frac{f_s}{2}$,
            \raisebox{-6ex}{$-f_{\mathrm{sig}}$},
            0,
            \raisebox{-6ex}{$f_{\mathrm{sig}}$},
            $\frac{f_s}{2}$,
            $f_s$,
            $\frac{3f_s}{2}$},
        xticklabel style={font=\footnotesize},
    ]
        % Centered around 0
        \addplot[q3,thick,ycomb,mark=triangle*,mark options={scale=2.0}] coordinates {
            (1,1)
            (-1,1)
        };
        % Centered around fs = 4
        \addplot[thick,q1,ycomb,mark=triangle*,mark options={scale=2.0}] coordinates {
            (3,1)
            (5,1)
        };
        % Centered around -fs = -4
        \addplot[thick,q5,ycomb,mark=triangle*,mark options={scale=2.0}] coordinates {
            (-3,1)
            (-5,1)
        };

        % Centered around +-2fs = +-8
        \addplot[thick,br0,ycomb,mark=triangle*,mark options={scale=2.0}] coordinates {
            (7,1)
        };
        \addplot[thick,br1,ycomb,mark=triangle*,mark options={scale=2.0}] coordinates {
            (-7,1)
        };

        % Sampling Frequency and its half
        \addplot[thick,gray,ycomb,mark=none,style={dashed,thick}] coordinates {
            (2,0.75)
            (-2,0.75)
            (6,0.75)
            (-6,0.75)
        };
        \addplot[thick,gray,ycomb,mark=none,style={dashed,thick}] coordinates {
            (4,1.33)
            (-4,1.33)
        };
    \end{axis}

    % Dirac at f=3
    \begin{axis}[
        at = {(0mm,-43mm)},
        xtick={-6,-4,-3,-2,0,2,3,4,6},
        xticklabels={
            $-\frac{3f_s}{2}$,
            $-f_s$,
            \raisebox{-6ex}{$-f_{\mathrm{sig}}$},
            $-\frac{f_s}{2}$,
            0,
            $\frac{f_s}{2}$,
            \raisebox{-6ex}{$f_{\mathrm{sig}}$},
            $f_s$,
            $\frac{3f_s}{2}$},
        xticklabel style={font=\footnotesize},
    ]
        % Centered around 0
        \addplot[thick,q3,ycomb,mark=triangle*,mark options={scale=2}] coordinates {
            (3,1)
            (-3,1)
        };
        \addlegendentry{copy centered around $f=0$}

        % Centered around fs = 4
        \addplot[thick,q1,ycomb,mark=triangle*,mark options={scale=2}] coordinates {
            (1,1)
            (7,1)
        };
        \addlegendentry{copy centered around $f=f_s$}
        % Centered around -fs = -4
        \addplot[thick,q5,ycomb,mark=triangle*,mark options={scale=2}] coordinates {
            (-1,1)
            (-7,1)
        };
        \addlegendentry{copy centered around $f=-f_s$}

        % Centered around +-2fs = +-8
        \addplot[thick,br0,ycomb,mark=triangle*,mark options={scale=2}] coordinates {
            (5,1)
        };
        \addlegendentry{copy centered around $f=2f_s$}
        \addplot[thick,br1,ycomb,mark=triangle*,mark options={scale=2}] coordinates {
            (-5,1)
        };
        \addlegendentry{copy centered around $f=-2f_s$}

        % Sampling Frequency and its half
        \addplot[thick,gray,ycomb,mark=none,style={dashed,thick}] coordinates {
            (2,0.75)
            (-2,0.75)
            (6,0.75)
            (-6,0.75)
        };
        \addplot[thick,gray,ycomb,mark=none,style={dashed,thick}] coordinates {
            (4,1.33)
            (-4,1.33)
        };
    \end{axis}
\end{tikzpicture}

    \caption[Aliasing With Harmonic Signals]{%
        Example  of  two harmonic  signals  being  sampled. In the  top  plot,
        the  signal's  frequency is  below  half  the sampling  frequency  and
        there  is  no  aliasing. The   signal  can  be  reconstructed  without
        error.\protect\newline
        In the bottom plot, the signal's  frequency is above half the sampling
        frequency. Consequently, the copies of the signal's frequency spectrum
        centered around  the sampling  frequency and  its negative  alias back
        into  the  band  between  $-f_s/2$  and  $f_s/2$. If  this  signal  is
        reconstructed, the resulting  signal would have a frequency  of $f_s -
        f_{\mathrm{sig}}$ instead of $f_{\mathrm{sig}}$.%
    }
    \label{fig:aliasing:dirac}
\end{figure}


Once a signal  has left the ADC and  is handed down the DSP  chain for further
processing,  the primary  problem becomes  one of  resources, particularly  in
real-time applications. In  most systems,  the available  hardware is  a fixed
constraint, and depending on what sort of processing is to be conducted on the
digital data stream, the available resources may or may not suffice.

If available resources are found to be insufficient for real-time processing of
the data stream, one may choose to
\begin{itemize}\tightlist
    \item
        not process the data in real time,
    \item
        reduce the complexity of the computations, or
    \item
        reduce the amount of data to be processed through \emph{downsampling}
        of the signal.
\end{itemize}
The  last case  is the  route  which is  chosen in  our application. The  main
constraint on the Red Pitaya is that  the data being generated cannot be moved
off the device in  real time, and the device itself  does not offer sufficient
storage for capturing a meaningful amount of data which can then be moved onto
another device for  further processing at a later  point. Therefore the amount
of data must  be reduced before it can  be moved off the device  to a computer
for viewing or further processing.

TODO: Amount of data being generated on the PITA.

Because downsampling a signal is in  essence nothing more than the sampling of
a signal which has already been sampled, a lot of the considerations which are
valid for the  step from an analog  to a digital signal as  outlined above are
either very  similar or even identical. Specifically,  the same considerations
for  aliasing still  apply: If  the  signal which  is  to  be downsampled  has
frequency components  above $f_{s,  downsampled}/2$, aliasing  will occur. And
since the  signal coming out of  the ADC has the  original signal's (filtered)
spectrum recurring at intervals of its  sampling frequency, this is always the
case.\code{TODO: Correct?}

Therefore,  the sampled  signal must  be  filtered through  a low-pass  filter
before  being downsampled,  just as  the original  analog signal  was low-pass
filtered  before being  passed into  the  ADC. In light  of the  signal to  be
downsampled  being a  \emph{digital} signal  instead  of an  analog one,  that
low-pass filter must  naturally be a digital filter as  well. Designing such a
digital low-pass filter is the core mission of this project.

The key properites of such a filter which are relevant to our application are
\begin{itemize}\tightlist
    \item
        its transition band width (filter steepness), and
    \item
        its aliasing attenuation.
\end{itemize}
The aliasing attenuation refers  to the fact that when a  filter is being used
for  downsampling,  copies  of  its  frequency response  will  be  created  at
intervals  of the  lower  sampling  rate (analogous  to  the sampling  process
producing spectral copies of a signal when sampling an analog signal).

The stopband  components of these  copies overlap with the  intended passband,
leading to aliasing  (it should be noted that this  phenomenon is also present
in the  case of the  analog intput filter for  the DSP chain). This  effect is
portrayed  in  Figure~\ref{fig:aliasing:iirCopies}. The  top  plot  shows  the
filter's frequency response  along with four copies to  illustrate the overlap
effect. The bottom plot shows the aliasing effect more clearly by removing the
spectral copies and retaining the aliased regions.

The overlapping parts  of the spetrum are composed of  spectral copies both to
the right  and left side of  the original. Therefore, the aliased  regions are
alternately flipped around the vertical axis. This creates in essence the same
effect  as if  the paper  were folded  along multiples  of the  lower sampling
rate  over the  frequency  range of  the  central  copy (in  the  case of  our
example: $0.2f_s$, $0.4f_s$, $0.6f_s$ and $0.8f_s$) like an accordion. This is
where the term \emph{folding back} originates.

\begin{figure}
    \centering
    %\tikzsetnextfilename{iirCopies}
\newcommand*\freqzFileCICA{images/iirCopies/iirAliasingZero.csv}
\newcommand*\freqzFileCICB{images/iirCopies/iirAliasingP2.csv}
\newcommand*\freqzFileCICC{images/iirCopies/iirAliasingN2.csv}
\newcommand*\freqzFileCICD{images/iirCopies/iirAliasingP4.csv}
\newcommand*\freqzFileCICE{images/iirCopies/iirAliasingN4.csv}
\newcommand*\freqzFileCICF{images/iirCopies/iirAliasingCleanZero.csv}
\newcommand*\freqzFileCICG{images/iirCopies/iirAliasingCleanP2.csv}
\newcommand*\freqzFileCICH{images/iirCopies/iirAliasingCleanN2.csv}
\newcommand*\freqzFileCICI{images/iirCopies/iirAliasingCleanP4.csv}
\newcommand*\freqzFileCICJ{images/iirCopies/iirAliasingCleanN4.csv}
\pgfplotstableread[col sep=comma]{\freqzFileCICA}\freqzTableCICA
\pgfplotstableread[col sep=comma]{\freqzFileCICB}\freqzTableCICB
\pgfplotstableread[col sep=comma]{\freqzFileCICC}\freqzTableCICC
\pgfplotstableread[col sep=comma]{\freqzFileCICD}\freqzTableCICD
\pgfplotstableread[col sep=comma]{\freqzFileCICE}\freqzTableCICE
\pgfplotstableread[col sep=comma]{\freqzFileCICF}\freqzTableCICF
\pgfplotstableread[col sep=comma]{\freqzFileCICG}\freqzTableCICG
\pgfplotstableread[col sep=comma]{\freqzFileCICH}\freqzTableCICH
\pgfplotstableread[col sep=comma]{\freqzFileCICI}\freqzTableCICI
\pgfplotstableread[col sep=comma]{\freqzFileCICJ}\freqzTableCICJ
\begin{tikzpicture}[
%    trim axis left,
%    trim axis right,
]
     \pgfplotsset{every axis/.style={
            width=\textwidth,
            grid=none,
            y filter/.code={\pgfmathparse{20*log10(\pgfmathresult))}},
            x filter/.code={\pgfmathparse{\pgfmathresult / 3.141592654}},
            xlabel=Normalized Frequency,
            ylabel=Magnitude,
            x unit=\times\,\pi\,\si{\radian}/\si{\sample},
            y unit=\si{dB},
        },
    }
    \begin{axis}[
            %title={Folding Back Due to Spectral Copies{,} $f_\mathrm{s,low} = \frac{f_\mathrm{s,high}}{5}$},
            at = {(0,0)},
            height=50mm,
            xmin=-1.8,
            xmax=1.8,
            ymin=-120,
            ymax=5,
            xtick={
                -1,
                %-0.8,
                %-0.4,
                0,
                %0.4,
                %0.8,
                1%
            },
            xticklabels={%
                $-f_\mathrm{s,high}/2$,
                %$-2 \cdot f_\mathrm{s,low}$,
                %$-1 \cdot f_\mathrm{s,low}$,
                $0$,
                %$1 \cdot f_\mathrm{s,low}$,
                %$2 \cdot f_\mathrm{s,low}$,
                $f_\mathrm{s,high}/2$%
            },
            %xticklabel style={rotate=90},
        ]
        \fill[q2!20!white] (0.2,5) rectangle (0.4,-120);
        \fill[q0!20!white] (0.4,5) rectangle (0.6,-120);
        \fill[q5!20!white] (0.6,5) rectangle (0.8,-120);
        \fill[q7!20!white] (0.8,5) rectangle (1.0,-120);

        \draw[black] (0,5) -- (0,-120);

        \addplot[q2,-]            table[x=w, y=abs(H)] \freqzTableCICB;
        \addplot[q0,-]            table[x=w, y=abs(H)] \freqzTableCICC;
        \addplot[q5,-]            table[x=w, y=abs(H)] \freqzTableCICD;
        \addplot[q7,-]            table[x=w, y=abs(H)] \freqzTableCICE;
        \addplot[very thick,q1,-] table[x=w, y=abs(H)] \freqzTableCICA;

        % The rectangles draw over the axis lines
        \draw (0.2,   5) -- (1,   5);
        \draw (0.2,-120) -- (1,-120);
    \end{axis}

    %\begin{axis}[
    %        title={Folding Back Due to Spectral Copies: Detail},
    %        at = {(0,-125mm)},
    %        height=90mm,
    %        xmin=0,
    %        xmax=1,
    %        ymin=-120,
    %        ymax=5,
    %        xtick={
    %            0,
    %            0.2,
    %            0.4,
    %            0.6,
    %            0.8,
    %            1%
    %        },
    %        xticklabels={%
    %            $0$,
    %            $1 \cdot f_\mathrm{s,low}$,
    %            $2 \cdot f_\mathrm{s,low}$,
    %            $3 \cdot f_\mathrm{s,low}$,
    %            $4 \cdot f_\mathrm{s,low}$,
    %            $f_\mathrm{s,high}/2$%
    %        },
    %        xticklabel style={rotate=90},
    %    ]
    %    \fill[q2!20!white] (0.2,5) rectangle (0.4,-120);
    %    \fill[q0!20!white] (0.4,5) rectangle (0.6,-120);
    %    \fill[q5!20!white] (0.6,5) rectangle (0.8,-120);
    %    \fill[q7!20!white] (0.8,5) rectangle (1.0,-120);

    %    \addplot[thick,q2,-] table[x=w, y=abs(H)] \freqzTableCICG;
    %    \addplot[thick,q0,-] table[x=w, y=abs(H)] \freqzTableCICH;
    %    \addplot[thick,q5,-] table[x=w, y=abs(H)] \freqzTableCICI;
    %    \addplot[thick,q7,-] table[x=w, y=abs(H)] \freqzTableCICJ;
    %    \addplot[very thick,q1,-] table[x=w, y=abs(H)] \freqzTableCICF;

    %    % The rectangles draw over the axis lines
    %    \draw (0.2,   5) -- (1,   5);
    %    \draw (0.2,-120) -- (1,-120);
    %    \draw (1,     5) -- (1,-120);
    %\end{axis}
\end{tikzpicture}

    \caption[Folding Back of Stopband Components Into Passband]{%
        The phenomenon  of folding back  when downsampling, illustrated  for a
        lowpass IIR filter with with a  cutoff frequency of $0.2\cdot f_s$ for
        a  downsampling  ratio  of $R=5$.\protect\newline
        The  downsampling process  produces copies  of the  filter's frequency
        response at intervals of the  lower sampling frequency, visible in the
        top  plot.   The stopbands  of  these  copies  then overlap  with  the
        intended passband.\protect\newline
        The bottom plot  shows an uncluttered version with  the copies removed
        and the aliasing components retained.\protect\newline
        As can be seen, the folding back within the filter's edge results in
        an aliasing component with potentially significant magnitude.%
    }
    \label{fig:aliasing:iirCopies}
\end{figure}

%>>>

\section{Digital Filters}% <<< ----------------------------------------------- %
\label{sec:digital_filters}

Digital filters can  be distinguished by several  characteristics; common ways
to  categorize them  are by  topology,  impulse response  and their  frequency
response. There  are  two commonly  used  types  of digital  filters: Infinite
impulse   response   (IIR)  filters   and   finite   impulse  response   (FIR)
filters. Another important class of filters are cascaded integrator-comb (CIC)
filters,  however, in  the strictest  sense they  are a  special class  of FIR
filters rather than an entirely new  type of LTI system \cite{1163535}.  While
our system uses FIR and CIC filters,  a brief overview of IIR filters is still
presented here, for the sake of completeness.

\subsection{IIR Filters} %<<< ------------------------------------------------ %
\label{subsec:iir_filters}

Infinite impulse response  filters are so named because  their imulse response
continues  into perpetuity,  never  reaching zero. In  practice, the  response
usually comes  sufficiently close to  zero at a certain  point that it  can be
considered zero for most intents and purposes.

IIR filters have feedback paths, resulting  in a filter response equation with
non-trivial  denominator components. Their  basic  building  blocks are  delay
elements, multipliers and adders.

\begin{equation}
    \label{eq:iir_filter}
    H(z) = \frac{%
            \sum_{k=0}^N b_k \cdot z^{-k}}{%
            1 + \sum_{i=0}^M a_i \cdot z^{-i}}
\end{equation}

TODO: Move sum limits above and below sigma sign.

IIR filters generally require a lower order (and therefore fewer resources) to
approximate a  certain frequency  response specification  than FIR  filters do
(particularly the constraint of a narrow transition band), but this comes at a
cost:  IIR  filters have a  non-linear phase response;  linear-phase responses
can only  be approximated. Furthermore, IIR  filters are not guaranteed  to be
BIBO stable due to their feedback paths.

\begin{figure}
    \centering
    % https://tex.stackexchange.com/a/183092/131649
\begin{tikzpicture}[
    triangle/.style = {draw,regular polygon, regular polygon sides=3 },
    node rotated/.style   = {rotate=180},
    border rotatedA/.style = {shape border rotate=-90},
    border rotatedB/.style = {shape border rotate=90},
]    
    \coordinate (in)  at (0,0);
    \coordinate (out) at (8,0);

    % Delay elements
    \node[draw] (d1) at  (2,-1) {$z^{-1}$};
    \node[draw] (d2) at  (2,-3) {$z^{-1}$};
    \node[draw] (d3) at  (6,-1) {$z^{-1}$};
    \node[draw] (d4) at  (6,-3) {$z^{-1}$};

    % Multipliers
    \node[triangle, border rotatedA] (m1) at  (3,0) {};
    \node[triangle, border rotatedA] (m2) at  (3,-2) {};
    \node[triangle, border rotatedA] (m3) at  (3,-4) {};
    \node[triangle, border rotatedA] (m4) at  (5,0) {};
    \node[triangle, border rotatedB] (m5) at  (5,-2) {};
    \node[triangle, border rotatedB] (m6) at  (5,-4) {};

    %% Adders
    \node[draw,circle, inner sep=0.3mm] (a1) at  (4,0) {$+$};
    \node[draw,circle, inner sep=0.3mm] (a2) at  (4,-2) {$+$};
    \node[draw,circle, inner sep=0.3mm] (a3) at  (4,-4) {$+$};

    %% Lines
    \draw[-latex] (in) -- (m1);
    \draw[-latex] (in) -| (d1);
    \draw[-latex] (d1) -- (d2);
    \draw[-latex] (d1) |- (m2);
    \draw[-latex] (d2) |- (m3);
    \draw[-latex] (m1) -- (a1);
    \draw[-latex] (m2) -- (a2);
    \draw[-latex] (m3) -- (a3);
    \draw[-latex] (a1) -- (m4);
    \draw[-latex] (m5) -- (a2);
    \draw[-latex] (m6) -- (a3);
    \draw[-latex] (d3) |- (m5);
    \draw[-latex] (d3) -- (d4);
    \draw[-latex] (d4) |- (m6);
    \draw[-latex] (m4) -| (d3);
    \draw[-latex] (m4) -- (out);
\end{tikzpicture}

    \caption[IIR Filter: Biquad]{Example of an IIR filter topology for a biquad}
    \label{fig:filtertopologies:iir}
\end{figure}

Some of the generally used types of IIR filters are:

\begin{itemize}\tightlist
    \item
        Butterworth  filter: Named after  the British  engineer and  physicist
        Stephen  Butterworth  (1885  --  1958),  who  first  described  it  in
        1930. Characterized by a very flat passband (no passband ripple).
    \item
        Chebyshev filter  (type I  and II): Named after  Russian mathematician
        Pafnuty Chebyshev  (1821 --  1894). They are steeper  than Butterworth
        filters, at the cost of suffering from ripple in the passband (type I)
        or stopband (type II).
    \item
        Bessel  filter: Named for  the German  mathematician Friedrich  Bessel
        (1784 -- 1846). Optimized to have a maximally linear phase response in
        order  to  minimize the  distortion  of  signals passing  through  the
        filter.
    \item
        Elliptical  filters: Also known  as  Cauer filters,  after the  German
        mathematician Wilhelm Cauer (1900 -- 1945), or Zolotarev filter, after
        Russian mathematician Yegor Zolotarev (1847 -- 1878). Characterized by
        equiripple in the  bassband and stopband and a  very narrow transition
        band compared to other filters of the same order.
\end{itemize}

TODO: Check correct dashes for years.

Digital IIR filters are often designed by way of the bilinear transform.

%>>>

\subsection{FIR Filters} %<<< ------------------------------------------------ %
\label{subsec:FIR_filters}

FIR filters are  characterized by an impulse response which  decays to zero in
finite time (see  Figure~\ref{fig:filter_specs:coefs}, unlike IIR filters. The
filter response is characterized by Equation~\ref{eq:fir_filter}:

\begin{equation}
    \label{eq:fir_filter}
    H(z) = \sum_{k=0}^{N} b_k \cdot z^{-k}
\end{equation}

FIR filters have several advantages:

\begin{itemize}\tightlist
    \item
        They are inherently BIBO stable because they lack feedback paths.
    \item
        They  can  be  easily  designed  to  have  a  linear  phase  response,
        preventing signal distortion due to  different group delays for signal
        components of different frequencies.
    \item
        The shape of  their frequency response can be  very finely tuned. This
        makes them ideally  suited for certain purposes,  such as compensation
        filters (see Section~\ref{subsec:CIC_filters}).
    \item
        Implemenation is usually rather straightforward.
\end{itemize}
TODO: Correct?

Their  main  disadvantage   is  that  due  to  the  lack   of  feedback,  they
generally require comparatively high filter  orders for narrow transition band
widths. Illustratively, this can be understood by the following considerations:
\begin{itemize}\tightlist
    \item
        The frequency response of an ideal low-pass filter is the brick wall
        filter, i.e. a rectangle.
    \item
        The inverse  Fourier transform  of a rectangle  is a  $sinc$ function,
        which is infinitely long.
    \item
        Therefore, the impulse  response of the ideal brick  wall filter would
        have an infinite number of taps.
    \item
        Truncation of the number of taps  leads to a deviation of the filter's
        frequency response from the brick wall  filter.  As the number of taps
        (and  therefore the  FIR filter's  impulse response)  is reduced,  its
        frequency response deviates more and  more from the brick wall filter,
        resulting  in  a  flatter  transition between  the  passband  and  the
        stopband as well as the introduction of ripple.
\end{itemize}

This     process     is     illustrated      in     simplified     form     in
Figure~\ref{fig:brick_wall_vs_FIR}.

The  FIR filter's  transition band  width  is particularly  important for  our
application in order to reduce aliasing effects, as will be shown later {TODO:
actually show later}.

\begin{figure}
    \centering
    \begin{tikzpicture}
     \pgfplotsset{every axis/.style={
            height=40mm,
            width=60mm,
            grid=none,
            axis x line=middle,
            axis y line=middle,
            ytick style={draw=none},
            xtick style={draw=none},
        }
    }
    \begin{axis}[
        at = {(0,30mm)},
        xmin=-10,
        xmax=10,
        xtick={},
        xticklabels={},
        ytick={},
        yticklabels={},
        ymax=1.1,
        xlabel=$t$,
    ]
        \addplot[thick,q1,smooth,-] table[x=t,y=y,col sep=comma] {images/brickwallVsFIR/sinc.csv};
        \node[anchor=east] at (-7.9,0.04) {\footnotesize\color{q1}\ldots};
        \node[anchor=west] at ( 8.1,0.04) {\footnotesize\color{q1}\ldots};
    \end{axis}

    \begin{axis}[
        at = {(50mm,30mm)},
        xmin=-10,
        xmax=10,
        xtick={},
        xticklabels={},
        ytick={},
        yticklabels={},
        ymax=1.1,
        xlabel=$t$,
    ]
        \addplot[thick,q1,smooth,-] table[x=t,y=y,col sep=comma] {images/brickwallVsFIR/fir.csv};
    \end{axis}

    \begin{axis}[
        at = {(0,0)},
        xmin=0,
        xmax=3,
        xtick={},
        xticklabels={},
        ytick={},
        yticklabels={},
        ymax=1.1,
        xlabel=$f$,
    ]
        \addplot[thick,q1,-] coordinates{
            (0,1)
            (1,1)
            (1,0)
            (3,0)
        };
    \end{axis}

    \begin{axis}[
        at = {(50mm,0)},
        xmin=0,
        xmax=3,
        xtick={},
        xticklabels={},
        ytick={},
        yticklabels={},
        ymin=0,
        ymax=1.1,
        xlabel=$f$,
    ]
        \addplot[thick,q1,-] coordinates{
            (0,1)
            (1,1)
            (1.45,0)
            (3,0)
        };
        \addplot[thick,q1,domain=1.45:1.98,samples=100] {0.1*1/x*abs(sin(1000*x))};
    \end{axis}

\end{tikzpicture}

    \caption[Brick Wall Filter vs. FIR Filter (simplified)]{%
        The effect of  truncating a $sinc$ function in the  time domain on its
        spectrum (simplified)%
    }
    \label{fig:brick_wall_vs_FIR}
\end{figure}

TODO: Equation for order estimation.

Designing    FIR    filters    is     usually    performed    by    specifying
certain     desired    characteristics     of    the     filter's    frequency
response. Figure~\ref{fig:filter_specs:freqResponse} shows one possible way of
doing this for FIR filters by specifying four constraint parameters:
\begin{itemize}\tightlist
    \item
        pass band ripple: $A_P$
    \item
        stop band attenuation: $A_{St}$
    \item
        pass band edge frequency: $F_P$
    \item
        stop band edge frequency: $F_{St}$
\end{itemize}

The resulting  transition band  width $F_{Tb}$ is  the difference  between the
pass band  edge frequency and  the stop band edge  frequency, and serves  as a
useful indicator of how many coefficients (i.e. resources) the filter will end
up using. Narrower transition bands tend to require a higher filter order, and
therefore  more  resources. Coefficient  counts  of several  hundred  are  not
uncommon for steep FIR filters.

Other sets of  constraint parameters can be used to  design filters, but these
are the ones used in this project, therefore the emphasis on them.

Figure~\ref{fig:filter_specs:coefs}  shows  the   resulting  impulse  response
(coefficient  set)  for  a  FIR  filter  designed  by  using  the  four  above
mentioned parameters, with  values given by Equations~\ref{eq:filter_specs:ap}
through \ref{eq:filter_specs:fst} handed to one  of Matlab's FIR filter design
algorithms.

\begin{align}
    A_P    &= \SI{2}{\dB}   \label{eq:filter_specs:ap}\\
    A_{St} &= \SI{60}{\dB}  \label{eq:filter_specs:ast}\\
    F_P    &= 0.3 \cdot f_s \label{eq:filter_specs:fp}\\
    F_{St} &= 0.4 \cdot f_s \label{eq:filter_specs:fst}
\end{align}

\begin{figure}
    \centering
    \tikzsetnextfilename{FIRDesignSpecsExample}
\newcommand*\freqzFile{images/filterSpecs/freqResponse.csv}
\pgfplotstableread[col sep=comma]{\freqzFile}\freqzTable
\begin{tikzpicture}[
    trim axis left,
    trim axis right,
]
     \pgfplotsset{every axis/.style={
            height=45mm,
            width=\textwidth,
            grid=none,
            y filter/.code={\pgfmathparse{20*log10(\pgfmathresult))}},
            x filter/.code={\pgfmathparse{\pgfmathresult / 3.141592654}},
            xlabel=Normalized Frequency,
            ylabel=Magnitude,
            x unit=\times\,\pi\,\si{\radian}/\si{\sample},
            y unit=\si{dB},
        },
        set layers,
    }
    \begin{axis}[
            xmin=0,
            xmax=1,
            ymax=5,
            ymin=-100,
            xtick={0,0.3,0.4,1},
            ytick={0,-60,-100},
            xticklabels={0,$f_P$,$f_{St}$,$f_s/2$},
        ]
        \addplot[thick,q1,-] table[x=W, y=abs(H)] \freqzTable;

        % Passband Edge and Stopband Edge Frequency, Transition Band Width
        \draw[gray,dashed] (rel axis cs:0.3,0) -- (rel axis cs:0.3,1);
        \draw[gray,dashed] (rel axis cs:0.4,0) -- (rel axis cs:0.4,1);
        \draw[latex-latex,gray,dashed] (rel axis cs:0.3,0.1) -- (rel axis cs:0.4,0.1);
        \node at (rel axis cs:0.35,0.15) {$f_{Tb}$};

        % Stop band attenuation
        \draw[gray,dashed] (0,-60) -- (1,-60);
        \draw[gray,dashed] (0.7,0) -- (1,0);
        \draw[latex-latex,gray,dashed] (0.77,0) -- (0.77,-60);
        \node at (0.815,-30) {$A_{St}$};

        % Passband Ripple
        \draw[gray,dashed] (0, 2) -- (0.5,2);
        \draw[gray,dashed] (0,-2) -- (0.5,-2);
        \draw[gray,dashed,-latex] (0.45, 10) -- (0.45,2);
        \draw[gray,dashed,-latex] (0.45,-10) -- (0.45,-2);
        \node at (0.5,-10) {$A_{P}$};
    \end{axis}
\end{tikzpicture}

    \caption[Specifying FIR Filter Constraints]{
        Specifications in the frequency domain and the resulting filter's
        frequency response as designed by Matlab.%
    }
    \label{fig:filter_specs:freqResponse}
\end{figure}

\begin{figure}
    \centering
    \begin{tikzpicture}
     \pgfplotsset{every axis/.style={
            height=50mm,
            width=100mm,
            grid=none,
            % TODO: fix x axis unit
        }
    }
    \begin{axis}[
        at = {(0,0)},
        xmin=-22,
        xmax=22,
        xtick={-19,-10,0,10,19},
    ]
        \addplot[ycomb,mark=*,dv+5] table[x=x,y=y,col sep=comma] {images/filterSpecs/coefs.csv};

        \draw[gray] (-22,0) -- (22,0);
    \end{axis}
\end{tikzpicture}

    \caption[Impulse Response of a FIR Filter]{
        Impulse    response    (coefficients)     for    the    filter    from
        Figure~\ref{fig:filter_specs:freqResponse}    with   the    parameters
        as     given     by     Equations~\ref{eq:filter_specs:ap}     through
        \ref{eq:filter_specs:fst}    passed   to    one   of    Matlab's   FIR
        filter   design   algorithms,  resulting   in   a   set  of   \num{39}
        coefficients.\protect\newline
        Note that the  coefficients to the left and right  of these values are
        zero, hence \emph{finite} impulse response filters.%
    }
    \label{fig:filter_specs:coefs}
\end{figure}


Figure~\ref{fig:filtertopologies:fir}   shows   one  possible   topology   for
implementing a  FIR filter,  the so-called  direct form. As  can be  seen, the
basic building  blocks of  a FIR  filter are  delay elements,  multipliers and
adders, same as for IIR filters.

\begin{figure}
    \centering
    \tikzsetnextfilename{firTopology}
% https://tex.stackexchange.com/a/183092/131649
\begin{tikzpicture}[
    triangle/.style = {draw,regular polygon, regular polygon sides=3 },
    node rotated/.style = {rotate=180},
    border rotated/.style = {shape border rotate=180}
]    
    \coordinate (in) at   (0,0);
    \coordinate (out) at (12,-2);

    % Delay elements
    \node[draw] (d1) at  (2,0) {$z^{-1}$};
    \node[draw] (d2) at  (4,0) {$z^{-1}$};
    \node[draw] (d3) at  (6,0) {$z^{-1}$};
    \node       (d4) at  (8,0) {\ldots};
    \node[draw] (d5) at (10,0) {$z^{-1}$};

    % Multipliers
    \node[triangle, border rotated] (m1) at  (1,-1) {};
    \node[triangle, border rotated] (m2) at  (3,-1) {};
    \node[triangle, border rotated] (m3) at  (5,-1) {};
    \node[triangle, border rotated] (m4) at  (7,-1) {};
    \node[triangle, border rotated] (m5) at (11,-1) {};

    % Adders
    \node[draw,circle, inner sep=0.3mm] (a1) at  (3,-2) {$+$};
    \node[draw,circle, inner sep=0.3mm] (a2) at  (5,-2) {$+$};
    \node[draw,circle, inner sep=0.3mm] (a3) at  (7,-2) {$+$};
    \node                               (a4) at  (9,-2) {\ldots};
    \node[draw,circle, inner sep=0.3mm] (a5) at (11,-2) {$+$};

    % Lines
    \draw[-latex] (in) -- (d1);
    \draw[-latex] (in) -| (m1);
    \draw[-latex] (d1) -- (d2);
    \draw[-latex] (d1) -| (m2);
    \draw[-latex] (d2) -- (d3);
    \draw[-latex] (d2) -| (m3);
    \draw[-latex] (d3) -- (d4);
    \draw[-latex] (d3) -| (m4);
    \draw[-latex] (d4) -- (d5);
    \draw[-latex] (d5) -| (m5);
    \draw[-latex] (m1) |- (a1);
    \draw[-latex] (m2) -- (a1);
    \draw[-latex] (a1) -- (a2);
    \draw[-latex] (m3) -- (a2);
    \draw[-latex] (a2) -- (a3);
    \draw[-latex] (m4) -- (a3);
    \draw[-latex] (a3) -- (a4);
    \draw[-latex] (a4) -- (a5);
    \draw[-latex] (m5) -- (a5);
    \draw[-latex] (a5) -- (out);
\end{tikzpicture}

    \caption[FIR Filter Topology Example]
        {One possible topology for a FIR filter (direct form)}
    \label{fig:filtertopologies:fir}
\end{figure}

%>>>

\subsection{CIC Filters} %<<< ------------------------------------------------ %
\label{subsec:CIC_filters}

CIC  filters  were  first  introduced  in 1981  in  \cite{1163535}  by  Eugene
B. Hogenauer. They can be implemented both as decimation filters (reduction in
sampling rate)  and interpolation filters (increase  in sampling rate).

\subsubsection{General Description}
\label{subsubsec:cic:general_description}

A   CIC  filter   is  a   cascade  of   integrator  and   comb  stages,   with
either   a   sampling    rate   compressor   (in   case    of   a   decimator)
or   a    sampling   rate    expander   (in    case   of    an   interpolator)
between   the   integrator   and   comb   sections.    A   single   integrator
stage   is   shown    in   Figure~\ref{fig:filtertopologies:integrator},   and
Figure~\ref{fig:filtertopologies:comb}   shows   a   single  comb   stage   in
feedforward form. Figure~\ref{fig:filtertopologies:cic}  shows a  complete CIC
filter with three stages.

\begin{figure}
    \centering
    \begin{minipage}[t][][b]{0.45\textwidth}
        \centering
        \tikzsetnextfilename{integratorTopology}
% https://tex.stackexchange.com/a/183092/131649
\begin{tikzpicture}
    \coordinate (in)  at (0,0);
    \coordinate (out) at (4,0);

    % branching coordinates
    \coordinate (b1) at (3,0);

    % Delay elements
    \node[draw] (d1) at  (2,1) {$z^{-1}$};

    % Adders
    \node[draw,circle, inner sep=0.3mm] (a1) at (1,0) {$+$};

    % Lines
    \draw[-latex] (in) -- (a1);
    \draw[-latex] (a1) -- (out);
    \draw[-latex] (b1) |- (d1);
    \draw[-latex] (d1) -| (a1);
\end{tikzpicture}

        \caption[Integrator Stage]{A single integrator stage}
        \label{fig:filtertopologies:integrator}
    \end{minipage}
    \begin{minipage}[t][][b]{0.45\textwidth}
        \centering
        \tikzsetnextfilename{combTopology}
% https://tex.stackexchange.com/a/183092/131649
\begin{tikzpicture}
    \coordinate (in)  at (0,0);
    \coordinate (out) at (4,0);

    % branching coordinates
    \coordinate (b1) at (1,0);

    % Delay elements
    \node[draw] (d1) at (2,1) {$z^{-M}$};

    % Adder
    \node[draw,circle, inner sep=0.3mm] (a1) at (3,0) {$+$};

    % subtractors
    \node[above right=0.2ex] (s1) at (a1) {$-$};

    % Lines
    \draw[-latex] (in) -- (a1);
    \draw[-latex] (a1) -- (out);
    \draw[-latex] (b1) |- (d1);
    \draw[-latex] (d1) -| (a1);
\end{tikzpicture}

        \caption[Comb Stage]{A single comb stage in feedforward form}
        \label{fig:filtertopologies:comb}
    \end{minipage}
\end{figure}

\begin{figure}
    \centering
    %\tikzsetnextfilename{cicTopology}
% https://tex.stackexchange.com/a/183092/131649
\newcommand*\freqzFileCICB{images/cic/cic913.csv}
\newcommand*\freqzFileCombB{images/cic/comb913.csv}
\newcommand*\freqzFileIntB{images/cic/integrator3.csv}
\pgfplotstableread[col sep=comma]{\freqzFileIntB}\freqzTableIntB
\pgfplotstableread[col sep=comma]{\freqzFileCombB}\freqzTableCombB
\pgfplotstableread[col sep=comma]{\freqzFileCICB}\freqzTableCICB
\begin{tikzpicture}[
%    trim axis left,
%    trim axis right,
]
    \pgfplotsset{every axis/.style={
        height=20mm,
        grid=none,
        y filter/.code={\pgfmathparse{20*log10(\pgfmathresult))}},
        x filter/.code={\pgfmathparse{\pgfmathresult / 3.141592654}},
        %xlabel=Normalized Frequency,
        %ylabel=Magnitude,
        %x unit=\times\,\pi\,\si{\radian}/\si{\sample},
        %y unit=\si{dB},
        xtick={},
        xticklabels={},
        ytick={},
        yticklabels={},
        axis line style={draw=none},
        tick style={draw=none},
        },
    }

    \coordinate (in)  at  (0,0);
    \coordinate (out) at (10,0);

    % branching coordinates
    \coordinate (b1) at (2.25,0); % delta: 1.5
    \coordinate (b2) at (3.85,0); % delta: 1.5
    \coordinate (b3) at (5.8,0); % delta: 1.8
    \coordinate (b4) at (7.55,0); % delta: 1.5

    % Delay elements
    \node[draw] (d1) at  (1.6,1) {$z^{-1}$};
    \node[draw] (d2) at  (3.2,1) {$z^{-1}$};
    \node[draw] (d3) at  (6.5,1) {$z^{-M}$};
    \node[draw] (d4) at  (8.25,1) {$z^{-M}$};

    % Downsampler
    \node[draw] (r1) at (4.85,0) {$R\downarrow$};

    % Adders
    \node[draw,circle, inner sep=0.3mm] (a1) at  (1.1,0) {$+$};
    \node[draw,circle, inner sep=0.3mm] (a2) at  (2.7,0) {$+$};
    \node[draw,circle, inner sep=0.3mm] (a3) at  (7.05,0) {$+$};
    \node[draw,circle, inner sep=0.3mm] (a4) at  (8.8,0) {$+$};

    % subtractors
    \node[above right=0.2ex] (s1) at (a3) {$-$};
    \node[above right=0.2ex] (s2) at (a4) {$-$};

    % Lines
    \draw[-latex] (in) -- (a1);
    \draw[-latex] (a4) -- (out);
    \draw[-latex] (b1) |- (d1);
    \draw[-latex] (b2) |- (d2);
    \draw[-latex] (a1) -- (a2);
    \draw[-latex] (a2) -- (r1);
    \draw[-latex] (r1) -- (a3);
    \draw[-latex] (a3) -- (a4);
    \draw[-latex] (d1) -| (a1);
    \draw[-latex] (d2) -| (a2);
    \draw[-latex] (b3) |- (d3);
    \draw[-latex] (b4) |- (d4);
    \draw[-latex] (d3) -| (a3);
    \draw[-latex] (d4) -| (a4);

    % Annotations
    \node[above left =2.5ex] at (a1) {$f_s$};
    \node[above right=2.5ex] at (a4) {$f_s/R$};

    \begin{axis}[
            at = {(0,17mm)},
            xmin=0,
            xmax=1,
            ymin=-30,
            ymax=140,
            width=4.5cm,
            anchor=south west,
        ]
        \addplot[thick,q1,-] table[x=w, y=abs(H)] \freqzTableIntB;
    \end{axis}

    \begin{axis}[
            at = {(100mm,17mm)},
            xmin=0,
            xmax=1,
            ymin=-80,
            ymax=30,
            width=4.5cm,
            anchor=south east,
        ]
        \addplot[thick,q1,-] table[x=w, y=abs(H)] \freqzTableCombB;
    \end{axis}

    \begin{axis}[
            at = {(0,-4mm)},
            xmin=0,
            xmax=1,
            ymin=-80,
            ymax=65,
            width=10cm,
            anchor=north west,
        ]
        \addplot[thick,q1,-] table[x=w, y=abs(H)] \freqzTableCICB;
    \end{axis}
\end{tikzpicture}

    \caption[CIC Filter Topology]
        {CIC decimation filter topology with three integrator and comb stages}
    \label{fig:filtertopologies:cic}
\end{figure}

The integrator stages have a transfer function of
\begin{equation}
    \label{eq:cic:integrator_stage}
    H_I(z) = \frac{1}{1-z^{-1}}
\end{equation}

The comb stages run at the reduced frequency of $f_s/R$ and have the transfer
function
\begin{equation}
    \label{eq:cic:comb_stage}
    H_C(z) = 1 - z^{-RM}
\end{equation}
where  $M$  is the  \emph{differential  delay},  one  of the  filter's  design
parameters.

The  transfer function  of  a  complete CIC  filter  (referenced  to the  high
sampling rate  $f_s$) consisting of $N$  stages is deduced by  multiplying the
transfer functions of the $N$ cascaded integrator and comb stages, yielding
\begin{equation}
    \label{eq:cic:complete}
    H_{CIC}(z) = H_I^N(z) \cdot H_C^N(z) = 
    \frac{\left(1 - z^{-RM}\right)^N}{\left( 1 - z^{-1} \right)^N} =
    \left[\sum_{k = 0}^{RM-1} z^{-k}\right]^N
\end{equation}

Looking at  the last form  of the CIC  filter's transfer function,  it becomes
evident  that it  is in  essence a  FIR filter  with unitary  coefficients. Of
particular note is the fact that this is so despite each stage having feedback
or feedforward paths and  the integrator stages having poles at  $f = 0$ (i.e.
the integrators  by themselves are  not in fact  BIBO stable, even  though the
complete  system  is). The  fact  that  the  resulting  filter  has  no  poles
can  be  intuitively   understood  by  looking  at   the  frequency  responses
of  the   integrator  and  comb   stages,  and  finally  their   cascade  (see
Section~\ref{subsubsec:cic:frequency_characteristics}).

CIC filters are well-suited to large reductions in sampling rates because they
are very  economical in  their resource  usage. This economy  is based  on six
primary factors (see also \cite{1163535}):
\begin{itemize}\tightlist
    \item
        The filter requires no multipliers.
    \item
        There are no filter coefficients to store.
    \item
        The amount  of storage needed  for intermediate results is  reduced by
        running the comb  stages at a lower sampling  rate. A conventional FIR
        filter topology implementing the  same transfer function would require
        more resources for storing its intermediate results because the entire
        filter would run at the incoming sampling rate.
    \item
        The  topology  of  the  filter   has  a  high  degree  of  regularity;
        consisting of two  primary building blocks. This lends  itself well to
        optimization.
    \item
        The control logic can be kept simple.
    \item
        The  same  filter  design can  be  used  for  a  large range  of  rate
        change  factors $R$,  requiring  minimal  adaption in  circuitry. This
        effect  can be  seen  in the  frequency response  plotted  in the  top
        plot of  Figure~\ref{fig:cic:freq_responses:var}.
\end{itemize}

However, CIC filters do suffer from some drawbacks. The two primary ones are:
\begin{itemize}\tightlist
    \item
        For large rate  change factors $R$, the register growth  of the filter
        can become very large. TODO: see section blabla
    \item
        A   CIC  filter   has   only  three   design  parameters   determining
        its   frequency  response: Rate   change   factor  $R$,   differential
        delay   $M$,    and   the    number   of   stages    $N$. The   amount
        of   fine-tuning   which   can    be   conducted   on   the   filter's
        frequency   response  is   therefore   extremely   limited  (more   in
        Section~\ref{subsubsec:cic:frequency_characteristics}).
\end{itemize}

% TODO:  why  is  register  growth  an   isssue  if  it  is  not  actually  an
% issue? Clarify:   overflow   is   no   issue  for   calculating   the   comb
% stages. However, the registers must still  be able to represent the expected
% output.

As  can  be  seen in  Equation~\ref{eq:cic:integrator_stage},  the  integrator
stages have  unity feedback coefficients. In  the case of CIC  decimators, the
registers of  the integrators  will therefore  suffer from  register overflow.
This causes no harm as long as two conditions are fulfilled:
\begin{itemize}\tightlist
    \item
        The filter's  implementation is based  on two's complement  or another
        number system allowing wrap-around between  its most positive and most
        negative numbers.
    \item
        The maximum  magnitude which is expected  at the output is  within the
        range of that number system.
\end{itemize}

A numerical  example to demonstrate this  effect and better explain  the inner
workings  of a  CIC filter  can be  found in  Appendix~\ref{sec:app:cic_simu},
starting on page~\pageref{sec:app:cic_simu}.


\subsubsection{Frequency Characteristics}
\label{subsubsec:cic:frequency_characteristics}

This section presents some of  the more important frequency characteristics of
the  CIC  filter. We  will  start  with  some  considerations  about  how  the
integrators and comb  sections interact in the frequency domain  to create the
CIC filter's frequency response.

As  shown  in  the  topmost plot  in  Figure~\ref{fig:cic:freq_responses},  an
integrator is  in essence a lowpass  filter, with a pole  at $f = 0$.   A comb
filter is  a filter  which attenuates one  specific frequency  component along
with its multiples (in a notch comb  filter; there is also the inverse concept
of a  peak filter  which only  lets a  certain frequency  and multiples  of it
pass).  It is also  evident that comb filters have no poles  (a fact which can
be deduced from Equation~\ref{eq:cic:comb_stage} as well, of course).

Cascading integrators and  combs results in a frequency response  like the one
in the bottom  plot from Figure~\ref{fig:cic:freq_responses}. The integrator's
pole at $f = 0$ compensates for  the comb section's zero at the same location,
leading to a significant, but finite, DC gain of the CIC filter.

One  drawback of  CIC filtes  is that  they have  no clearly  defined passband
as  such. Rather,  their  frequency  response starts  dropping  off  right  as
the  frequency axis  goes  beyond  zero. This effect  (  also  referred to  as
\emph{passband  droop}  or  \emph{passband  attenuation}) is  visible  in  the
magnified section  of the  bottom plot  in Figure~\ref{fig:cic:freq_responses}
and  in   Figure~\ref{fig:cic:freq_responses:passband:attenuation}. Since  CIC
filters lack  a clearly defined  transition band edge, defining  the frequency
band which  is to  actually be  used, i.e.  the actual  passband, is  a design
decision  and can  vary even  when  using the  same filter,  depending on  the
application.

The  amount  of  passband  droop  is  constant for  a  given  product  of  the
differential  delay $M$  and  the cutoff  frequency $f_c$,  where  $f_c$ is  a
fraction  of the  lower sampling  rate (i.e.  a fraction  of the  first lobe's
width). Figure~\ref{fig:cic:freq_responses:passband:attenuation}    highlights
this  effect  for  two  different  filters. Table~\ref{tab:cic:pb_attenuation}
in                 Appendix~\ref{sec:app:cic_filter_tables}                 on
page~\pageref{tab:cic:pb_attenuation}  contains a  list with  more values  for
some common configurations.

Because  of the  passband droop,  a CIC  filer by  itself is  rarely a  viable
solution. Rather, it is generally deployed as  the first element in a chain of
filters,  where the  later stages  are FIR  filters. Due to  the CIC  filter's
frugality  in terms  of resource  usage, it  is ideally  suited as  an initial
stage, where the most samples per time need to be processed. The fact that FIR
filters need  to perform many  more computations  (and more complex  ones) per
sample is then no longer as much of  a problem, since those FIR filters run at
lower sampling frequencies and have therefore many more clock cycles available
to compute each  output. Also, because the frequency response of  a FIR filter
can be very finely tuned to a  desired profile, they can be used to compensate
for the  CIC filter's passband droop;  this is generally known  as a \emph{CIC
compensation filter}.
TODO: altera application note

Another  effect which  must be  taken  into consideration  when designing  CIC
filters is  the amount  of aliasing  which occurs from  the stopband  into the
passband. A  region  of  width  $f_c$  above and  below  each  $M$th  null  is
folded  back  into  the  filter's  passband. This  effect  is  highlighted  in
Figure~\ref{fig:cic:freq_responses:passband:aliasing}.   The  gravity of  this
effect depends  on the  width of  the cutoff  frequency $f_c$  as well  as the
differential delay $M$. Table  TODO in Appendix TODO contains  some values for
common ranges for M and $f_c$.

As  mentioned, the  CIC filter  has  only three  design paramterers: Its  rate
change factor  $R$, the differential delay  $M$ and the number  of stages $N$.
The influence  of these paramters  on the  CIC filter's frequency  response is
portrayed in Figure~\ref{fig:cic:freq_responses:var}. Some things of note are:
\begin{itemize}\tightlist
    \item
        Increasing $R$  increases the amount of  nulls as well as  the overall
        gain of the filter.
    \item
        Increasing  $M$ also  increases the  number of  nulls as  well as  the
        filter's gain.  Note that for  CIC decimators, the region around every
        $M$th null is folded back into the passband.

        For practical purposes, $M$ is usually  set to \num{1} or \num{2}, see
        \cite{1163535}.
    \item
        Adding more stages leads to a  high increase in filter gain, since $N$
        occurs in the exponent of the filter's transfer function. It does not,
        however, change the number or placement of the nulls.
\end{itemize}

\begin{figure}
    \centering
        %\tikzsetnextfilename{cicFreqResponsesStages}
\newcommand*\freqzFileIntB{images/cic/integrator3.csv}
\newcommand*\freqzFileCombB{images/cic/comb913.csv}
\newcommand*\freqzFileCICB{images/cic/cic913.csv}
\pgfplotstableread[col sep=comma]{\freqzFileIntB}\freqzTableIntB
\pgfplotstableread[col sep=comma]{\freqzFileCombB}\freqzTableCombB
\pgfplotstableread[col sep=comma]{\freqzFileCICB}\freqzTableCICB
\begin{tikzpicture}[
        %spy using outlines={magnification=4, connect spies}
        trim axis left,
        trim axis right,
    ]
     \pgfplotsset{every axis/.style={
            height=45mm,
            width=\textwidth,
            grid=none,
            y filter/.code={\pgfmathparse{20*log10(\pgfmathresult))}},
            x filter/.code={\pgfmathparse{\pgfmathresult / 3.141592654}},
            xlabel=Normalized Frequency,
            ylabel=Magnitude,
            x unit=\times\,\pi\,\si{\radian}/\si{\sample},
            y unit=\si{dB},
        },
        every axis legend/.append style={
            at={(1,-0.25)},  % if attached to bottom plot
            anchor=north east,
            cells={anchor=east},
        },
    }
    \begin{axis}[
            title=Integrator,
            at = {(0,0)},
            xmin=0,
            xmax=1,
            ymin=-30,
            ymax=140,
            xtick={0,0.5,1},
            ytick={120,100,80,60,40,20,0,-20},
            xticklabels={0,$f_s/4$,$f_s/2$},
        ]
        \addplot[thick,q1,-] table[x=w, y=abs(H)] \freqzTableIntB;
    \end{axis}

    \begin{axis}[
            title=Comb Filter,
            at = {(0,-65mm)},
            xmin=0,
            xmax=1,
            ymin=-80,
            ymax=30,
            xtick={0,0.5,1},
            ytick={20,0,-20,-40,-60},
            xticklabels={0,$f_s/4$,$f_s/2$},
        ]
        \addplot[thick,q1,-] table[x=w, y=abs(H)] \freqzTableCombB;
    \end{axis}

    \begin{axis}[
            title=CIC Filter,
            at = {(0,-130mm)},
            xmin=0,
            xmax=1,
            ymin=-80,
            ymax=65,
            xtick={0,0.5,1},
            ytick={60,40,20,0,-20,-40,-60},
            xticklabels={0,$f_s/4$,$f_s/2$},
        ]
        \addplot[thick,q1,-] table[x=w, y=abs(H)] \freqzTableCICB;

        %\coordinate (spypoint)     at (0.033,58);
        %\coordinate (magnifyglass) at (rel axis cs:0.123,-0.40);
    \end{axis}

    %\spy [black, width=3cm,height=1cm] 
    %    on (spypoint) in node[fill=white] at (magnifyglass);
    %\node[anchor=west,xshift=20mm] at (magnifyglass) {Passband Droop};
\end{tikzpicture}

        \caption[Frequency Responses for Integrators, Combs and CIC Filters]{%
            Frequency responses  for integrators, combs and  their combination
            into a three-stage CIC filter with a rate change factor of \num{9}
            and a differential delay of \num{1}.\protect\newline
            Note that \num{4.5}  lobes fit into the plot for  the comb filter,
            due to $R\cdot M = 9$ (the order of the comb filter).
            The enlarged  box shows  a close-up of  the CIC  filter's passband
            droop.%
        }
        \label{fig:cic:freq_responses}
\end{figure}

\begin{figure}
    \centering
        \tikzsetnextfilename{cicFreqResponsesVar}
\newcommand*\freqzFileCICA{images/cic/cic313.csv}
\newcommand*\freqzFileCICB{images/cic/cic913.csv}
\newcommand*\freqzFileCICC{images/cic/cic911.csv}
\newcommand*\freqzFileCICD{images/cic/cic921.csv}
\newcommand*\freqzFileCICE{images/cic/cic911.csv}
\newcommand*\freqzFileCICF{images/cic/cic913.csv}
\pgfplotstableread[col sep=comma]{\freqzFileCICA}\freqzTableCICA
\pgfplotstableread[col sep=comma]{\freqzFileCICB}\freqzTableCICB
\pgfplotstableread[col sep=comma]{\freqzFileCICC}\freqzTableCICC
\pgfplotstableread[col sep=comma]{\freqzFileCICD}\freqzTableCICD
\pgfplotstableread[col sep=comma]{\freqzFileCICE}\freqzTableCICE
\pgfplotstableread[col sep=comma]{\freqzFileCICF}\freqzTableCICF
\begin{tikzpicture}[
    trim axis left,
    trim axis right,
]
     \pgfplotsset{every axis/.style={
            height=45mm,
            width=\textwidth,
            grid=none,
            y filter/.code={\pgfmathparse{20*log10(\pgfmathresult))}},
            x filter/.code={\pgfmathparse{\pgfmathresult / 3.141592654}},
            xlabel=Normalized Frequency,
            ylabel=Magnitude,
            x unit=\times\,\pi\,\si{\radian}/\si{\sample},
            y unit=\si{\dB},
        },
        every axis legend/.append style={
            at={(0.02,0.05)},
            anchor=south west,
            cells={anchor=east},
        },
    }
    \begin{axis}[
            title=Influence of Rate Change $R$,
            at = {(0,0)},
            xmin=0,
            xmax=1,
            ymin=-100,
            ymax=65,,
            xtick={0,0.5,1},
            ytick={60,0,-60,-120},
            xticklabels={0,$f_s/4$,$f_s/2$},
        ]
        \addplot[thick,q1,-] table[x=w, y=abs(H)] \freqzTableCICA;
        \addplot[thick,q5,-] table[x=w, y=abs(H)] \freqzTableCICB;

        \legend{$R = 3$, $R = 9$};
    \end{axis}

    \begin{axis}[
            title=Influence of Differential Delay $M$,
            at = {(0,-65mm)},
            xmin=0,
            xmax=1,
            ymin=-40,
            ymax=30,
            xtick={0,0.5,1},
            ytick={20,0,-20,-40},
            xticklabels={0,$f_s/4$,$f_s/2$},
        ]
        \addplot[thick,q1,-] table[x=w, y=abs(H)] \freqzTableCICC;
        \addplot[thick,q5,-] table[x=w, y=abs(H)] \freqzTableCICD;

        \legend{$M = 1$, $M = 2$};
    \end{axis}

    \begin{axis}[
            title=Influence of Number of Stages $N$,
            at = {(0,-130mm)},
            xmin=0,
            xmax=1,
            ymin=-100,
            ymax=65,
            xtick={0,0.5,1},
            ytick={60,0,-60,-120},
            xticklabels={0,$f_s/4$,$f_s/2$},
        ]
        \addplot[thick,q1,-] table[x=w, y=abs(H)] \freqzTableCICE;
        \addplot[thick,q5,-] table[x=w, y=abs(H)] \freqzTableCICF;

        \legend{$N = 1$, $N = 3$};
    \end{axis}
\end{tikzpicture}

        \caption[Influence of Design Parameters on Frequency Response]{%
            The influence  of the design paramters  $R$, $M$ and $N$  on a CIC
            filter' frequency response.\protect\newline
            Inreaseing $R$ and $M$, respectively, leads to an increased number
            of nulls, as visible in the top  two plots, as well as an increase
            in the DC gain.\protect\newline
            Adding more stages does not change  the location of the nulls, but
            does add significant DC gain.%
        }
        \label{fig:cic:freq_responses:var}
\end{figure}

\begin{figure}
    \centering
        \newcommand*\freqzFileCICA{images/cic/pbattenuation914.csv}
\newcommand*\freqzFileCICB{images/cic/pbattenuation924.csv}
\pgfplotstableread[col sep=comma]{\freqzFileCICA}\freqzTableCICA
\pgfplotstableread[col sep=comma]{\freqzFileCICB}\freqzTableCICB
\begin{tikzpicture}[
        spy using outlines={magnification=4, connect spies}
    ]
     \pgfplotsset{every axis/.style={
            height=60mm,
            width=\textwidth,
            grid=none,
            y filter/.code={\pgfmathparse{20*log10(\pgfmathresult))}},
            x filter/.code={\pgfmathparse{\pgfmathresult / 3.141592654}},
            xlabel=Normalized Frequency,
            ylabel=Magnitude,
            x unit=\times\,\pi\,\si{\radian}/\si{\sample},
            y unit=\si{dB},
        },
    }
    \begin{axis}[
            title=Passband Droop in CIC Filters,
            at = {(0,0)},
            xmin=0,
            xmax=0.12,
            xtick={
                0.111111,
                0.055555,
                0.027777,
                0%
            },
            xticklabels={%
                $\frac{f_\mathrm{s,low}}{2}$,
                $\frac{f_\mathrm{s,low}}{4}$,
                $\frac{f_\mathrm{s,low}}{8}$,
                0%
            },
            ytick={
                0,
                -3.65,
                -10,
                -15
            },
            ymin=-15,
            ymax=1,
        ]
        \draw[da0,thick] (0      ,-3.65) -- (0.058   ,-3.65);
        \draw[da0,thick] (0.02777,-3)    -- (0.02777,-15);
        \draw[da0,thick] (0.05555,-3)    -- (0.05555,-15);

        \addplot[thick,q1,-] table[x=w, y=abs(H)] \freqzTableCICA;
        \addplot[thick,q3,-] table[x=w, y=abs(H)] \freqzTableCICB;
        \legend{%
            $M=1${,} $f_c = \frac{f_\mathrm{s,low}}{4}$,
            $M=2${,} $f_c = \frac{f_\mathrm{s,low}}{8}$
        };
    \end{axis}
\end{tikzpicture}

        \caption[CIC Filter: Passband and Aliasing Attenuation]{%
            Passband  attenation for  two CIC  filters with  $R=9$, $N=4$  and
            $M=1$ and  $M=2$, respectively. The  attenuation is  identical for
            the bandwidth-differential delay product,  which is $1/8$ for both
            of these configurations.\protect\newline
            The   attenuation   is   \SI{-3.65}{\dB}  in   both   cases;   the
            value  can  be   found  in  Table~\ref{tab:cic:pb_attenuation}  on
            page~\pageref{tab:cic:pb_attenuation}.%
        }
        \label{fig:cic:freq_responses:passband:attenuation}
\end{figure}

\begin{figure}
    \centering
        \newcommand*\freqzFileCICA{images/cic/pbAliasing914Complete.csv}
\newcommand*\freqzFileCICB{images/cic/pbAliasing914Null1Left.csv}
\newcommand*\freqzFileCICC{images/cic/pbAliasing914Null1Right.csv}
\newcommand*\freqzFileCICD{images/cic/pbAliasing914Null2Left.csv}
\newcommand*\freqzFileCICE{images/cic/pbAliasing914Null2Right.csv}
\newcommand*\freqzFileCICF{images/cic/pbAliasing914Null3Left.csv}
\newcommand*\freqzFileCICG{images/cic/pbAliasing914Null3Right.csv}
\newcommand*\freqzFileCICH{images/cic/pbAliasing914Null4Left.csv}
\newcommand*\freqzFileCICI{images/cic/pbAliasing914Null4Right.csv}
\pgfplotstableread[col sep=comma]{\freqzFileCICA}\freqzTableCICA
\pgfplotstableread[col sep=comma]{\freqzFileCICB}\freqzTableCICB
\pgfplotstableread[col sep=comma]{\freqzFileCICC}\freqzTableCICC
\pgfplotstableread[col sep=comma]{\freqzFileCICD}\freqzTableCICD
\pgfplotstableread[col sep=comma]{\freqzFileCICE}\freqzTableCICE
\pgfplotstableread[col sep=comma]{\freqzFileCICF}\freqzTableCICF
\pgfplotstableread[col sep=comma]{\freqzFileCICG}\freqzTableCICG
\pgfplotstableread[col sep=comma]{\freqzFileCICH}\freqzTableCICH
\pgfplotstableread[col sep=comma]{\freqzFileCICI}\freqzTableCICI
\begin{tikzpicture}[
        spy using outlines={magnification=3},
    ]
     \pgfplotsset{every axis/.style={
            height=140mm,
            width=\textwidth,
            grid=none,
        },
        every axis legend/.append style={
            at={(1,-0.15)},
            anchor=north east,
            cells={anchor=west},
        },
    }
    \begin{axis}[
            at = {(0,0)},
            y filter/.code={\pgfmathparse{20*log10(\pgfmathresult))}},
            x filter/.code={\pgfmathparse{\pgfmathresult / 3.141592654}},
            ylabel=$\left|H\right| (\si{\dB})$,
            xmin=0.001,
            xmax=1,
            ymin=-250,
            ymax=100,
            xtick={
                0,
                0.222222,
                0.444444,
                0.666666,
                0.888888,
                1%
            },
            xticklabels={%
                $0$,
                $1 \cdot f_\mathrm{s,low}$,
                $2 \cdot f_\mathrm{s,low}$,
                $3 \cdot f_\mathrm{s,low}$,
                $4 \cdot f_\mathrm{s,low}$,
                $f_\mathrm{s,high}/2$%
            },
        ]
        \fill[q0,opacity=0.20] ($(0.2222*0.75,100)$)        rectangle ($(0.2222*1.00,-250)$);
        \fill[q1,opacity=0.20] ($(0.2222*1.00,100)$)        rectangle ($(0.2222*1.25,-250)$);
        \fill[q2,opacity=0.20] ($(0.4444-0.2222*0.25,100)$) rectangle ($(0.4444+0.2222*0.00,-250)$);
        \fill[q3,opacity=0.20] ($(0.4444-0.2222*0.00,100)$) rectangle ($(0.4444+0.2222*0.25,-250)$);
        \fill[q4,opacity=0.20] ($(0.6666-0.2222*0.25,100)$) rectangle ($(0.6666+0.2222*0.00,-250)$);
        \fill[q5,opacity=0.20] ($(0.6666-0.2222*0.00,100)$) rectangle ($(0.6666+0.2222*0.25,-250)$);
        \fill[q6,opacity=0.20] ($(0.8888-0.2222*0.25,100)$) rectangle ($(0.8888+0.2222*0.00,-250)$);
        \fill[q7,opacity=0.20] ($(0.8888-0.2222*0.00,100)$) rectangle ($(0.8888+0.2222*0.25,-250)$);

        \fill[br0,opacity=0.50] (0,100) rectangle ($(0.2222*0.25,-250)$);
        \node[rotate=90,anchor=east] at ($(0.2222*0.25*0.5,-100)$) {%
            \footnotesize
            Passband: $f_c = 0.25 \cdot f_\mathrm{s,low}$
        };

        \draw[thick,da1] ($(0,76.31416-41.8)$) -- ($(0.2222*0.425,76.31416-41.8)$);
        \draw[thick,da1] (0,76.31416) -- ($(0.2222*0.425,76.31416)$);
        \draw[thick,da1,-latex] ($(0.2222*0.375,76.31416+5)$) -- ($(0.2222*0.375,76.31416)$);
        \draw[thick,da1] ($(0.2222*0.375,76.31416)$) -- ($(0.2222*0.375,76.31416-41.8)$);
        \draw[thick,da1,-latex] ($(0.2222*0.375,-50)$) -- ($(0.2222*0.375,76.31416-41.8)$);
        \node[thick,da1,rotate=90,anchor=north west] at ($(0.2222*0.375,-50)$) {\footnotesize\SI{41.8}{\dB}};

        \addplot[very thick,da1,-] table[x=w, y=abs(H)] \freqzTableCICA;
        \addplot[line cap=round,very thick,q0,-] table[x=w, y=abs(H)] \freqzTableCICB;
        \addplot[line cap=round,very thick,q1,-] table[x=w, y=abs(H)] \freqzTableCICC;
        \addplot[line cap=round,very thick,q2,-] table[x=w, y=abs(H)] \freqzTableCICD;
        \addplot[line cap=round,very thick,q3,-] table[x=w, y=abs(H)] \freqzTableCICE;
        \addplot[line cap=round,very thick,q4,-] table[x=w, y=abs(H)] \freqzTableCICF;
        \addplot[line cap=round,very thick,q5,-] table[x=w, y=abs(H)] \freqzTableCICG;
        \addplot[line cap=round,very thick,q6,-] table[x=w, y=abs(H)] \freqzTableCICH;
        \addplot[line cap=round,very thick,q7,-] table[x=w, y=abs(H)] \freqzTableCICI;

        \legend{%
            {},
            Left side of first null (flipped),
            Right side of first null (not flipped),
            Left side of second null (flipped),
            Right side of second null (not flipped),
            Left side of third null (flipped),
            Right side of third null (not flipped),
            Left side of fourth null (flipped),
            Right side of fourth null (not flipped),
        };

        %\coordinate (spypoint)     at (0.0300,9);
        %\coordinate (magnifyglass) at (rel axis cs:0.078,-0.35);
    \end{axis}

    %\spy [br0, width=1.9cm,height=5.3cm] 
    %    on (spypoint) in node[fill=white] at (magnifyglass);
\end{tikzpicture}

        \caption[CIC Filter: Passband and Aliasing Attenuation]{%
            Passband aliasing for  a CIC filter with $R =  9$, $N=4$ and $M=1$
            and a  cutoff frequency of $f_c  = 0.25$, referenced to  the lower
            sampling frequency $f_\mathrm{s,low}$.\protect\newline
            The  region of  width  $f_c$  around every  $M$th  null is  folded
            back  into the  passband. The regions  beyond that  are of  course
            folded  back  as   well,  but  since  we   choose  to  arbitrarily
            limit  the  passband,  those  regions   are  not  of  interest  to
            us.\protect\newline
            The  resulting passband  aliasing  attenuation is  \SI{41.8}{\dB},
            as     indicated     in     Table~\ref{tab:cic:pb_aliasing}     on
            page~\pageref{tab:cic:pb_aliasing}.\protect\newline
            It  is  also  evident   that  the  aliasing  attenuation  improves
            significantly as  the cutoff frequency decreases  and the passband
            comes closer to \num{0}.%
        }
        \label{fig:cic:freq_responses:passband:aliasing}
\end{figure}

TODO: passband droop table, aliasing attenuation, 

\subsubsection{Register Growth}
\label{subsubsec:cic:register_growth}

As shown by Hogenauer in~\cite{1163535}, the maximum register growth is
\begin{equation}
    \label{eq:cic:maximum_register_growth}
    G_\mathrm{max} = (R \cdot M)^N
\end{equation}
The most  significant bit $B_\mathrm{max}$ of  the output register as  well as
for all  stages (both the  integrators and the comb  stages) of the  filter is
determined to be
\begin{equation}
    \label{eq:cic:maximum_register_growth:bit_width}
    B_\mathrm{max} = \lceil N \log_2 RM + B_\mathrm{in} - 1 \rceil
\end{equation}
where $B_\mathrm{in}$ is  the bit width of the input  register.  For high rate
change  factors, these  values can  become very  large.  A  filter with  three
stages, a  differential delay of  \num{1}, a rate  change of \num{128}  and an
input  width of  \num{16}\,bits  yields  \num{36}\,bits output  width at  full
precision.

\subsubsection{Errors Due to Truncation and Rounding}
\label{subsubsec:cic:truncation_and_rounding}

In practical cases, it is often not feasible to retain full precision; in such
situations, either truncation or rounding may  be used at each filter stage to
reduce register widths and keep resource usage within certain limits. For this
purpose, it is necessary  to know the system function from  the $j$th stage up
to and including the last:
\begin{equation}
    \label{eq:cic:truncation_rounding:system_function}
    H_j(z) = \left\lbrace
        \begin{aligned}
            H_I^{N-j+1}H_C^N                        &
            = \sum_{k=0}^{(RM-1)N+j-1} h_j[k]z^{-k} &
            \quad j                                 &
            = 1, 2, \cdots, N                       \\
            H_C^{j-N}                                          &
            = \hspace{1.1em}\sum_{k=0}^{2N+1-j} h_j[k]z^{-kRM} &
            \quad j                                            &
            = N+1, \cdots, 2N
        \end{aligned}
    \right.
\end{equation}
where
\begin{equation}
    \label{eq:cic:truncation_rounding:system_function}
    h_j[k] = \left\lbrace
        \begin{aligned}
            \sum_{l=0}^{\lfloor k/(RM) \rfloor} (-1)^l 
            {{N}\choose{l}}{{N-j+k-RMl}\choose{k = RMl}} &
            j                                            &
            = 1, 2, \cdots N                             \\
            (-1)^k{{2N+1-j}\choose{k}} &
            j                          &
            = N+1, \cdots 2N
        \end{aligned}
    \right.
\end{equation}
are the  impulse response  coefficients. These functions  are also  derived by
Hogenauer in~\cite{1163535}.

In a  filter with $N$ stages,  there are $2N+1$  error sources in the  case of
limited precision: Each stage,  and the output register. Each  error source is
presumed to have  white noise characteristics, i.e. its  noise is uncorrelated
to its input as well as other error sources.  The error at the $j$th source is
assumed to have a uniform probability distribution with a width of
\begin{equation}
    \label{eq:cic:truncation_rounding:probability_distribution}
    E_j = \left\lbrace
        \begin{aligned}
            0        & \quad\text{without truncation or rounding}\\
            2^{B_j}  & \quad\text{otherwise}
        \end{aligned}
    \right.
\end{equation}
where the number of bits discarded at the $j$th error source is $B_j$. The mean
of this error is
\begin{equation}
    \label{eq:cic:truncation_rounding:mean}
    \mu_j = \left\lbrace
        \begin{aligned}
            \frac{1}{2}E_j & \quad\text{for truncation}\\
            0              & \quad\text{otherwise}
        \end{aligned}
    \right.
\end{equation}
and the variance comes out to
\begin{equation}
    \label{eq:cic:truncation_rounding:variance}
    \sigma_j^2 = \frac{1}{12}E_j^2.
\end{equation}

The total mean error at the filter's output due to the $j$th stage is
\begin{equation}
    \label{eq:cic:truncation_rounding:total_mean_error_jth_stage}
    \mu_{T_j} = \mu_jD_j
\end{equation}
where
\begin{equation}
    \label{eq:cic:truncation_rounding:mean_error_gain}
    D_j = \left\lbrace
        \begin{aligned}
            (RM)^N         & \quad j = 1\\
            0              & \quad j = 2, 3, \cdots, 2N\\
            1              & \quad j = 2N+1
        \end{aligned}
    \right.
\end{equation}
is the \emph{mean  error gain} for the $j$th error  source. Note that only the
first and the last  error source contribute to the filter's  mean error at the
output. This is because  the sum of the impulse response  coefficients is zero
for all other  stages. Consequently, whether one chooses to  truncate or round
is  without  consequence except  in  the  case of  the  first  and last  error
sources. In an analogous manner, the total variance computes to
\begin{equation}
    \label{eq:cic:truncation_rounding:total_variance_jth_stage}
    \sigma_{T_j}^2 = \sigma_j^2F_j^2
\end{equation}
where
\begin{equation}
    \label{eq:cic:truncation_rounding:variance_error_gain}
    F_j = \left\lbrace
        \begin{aligned}
            \sum_k h_j^2[k]  & \quad j = 1, 2, \cdots, 2N\\
            1                & \quad j = 2N+1
        \end{aligned}
    \right.
\end{equation}
is called the \emph{variance error gain} for the $jth$ error source.

We can now compute the global mean error and variance of the filter:
\begin{align}
    \label{eq:cic:truncation_rounding:global:mean_error}
    \mu_T &= \sum_{j = 1}^{2N+1} \mu_{T_j} = \mu_{T_1} + \mu_{T_{2N+1}}\\
    \label{eq:cic:truncation_rounding:global:variance}
    \sigma_{T}^2 &= \sum_{j=1}^{2N+1} \sigma_{T_j}
\end{align}

These equations  are used  to calculate  the properties of  the CIC  filter as
deployed in our design. TODO: see section blabla

\subsubsection{Summary}
\label{subsubsec:cic:summary}

In conclusion, the key properties of CIC filters are:
\begin{itemize}\tightlist
    \item
        They can be implemented both as decimators and interpolators.
    \item
        Neither multipliers nor storage for coefficients are needed.
    \item
        CIC  decimation  filters have  a  high  gain, leading  to  significant
        register  growth.  Truncation  or rounding  can be  used to  limit the
        resource usage, both at the filter's output and internally.
    \item
        The three design parameters are  the rate change $R$, the differential
        delay $M$ and the number of stages $N$.
    \item
        The  presence of  passband  droop requires  a  compensation filter  to
        achieve a flat passband response.
\end{itemize}

Compensation Filter?

%>>>
%>>>

%\subsection{Challenges in Downsampling}
%\label{subsec:downsampling}
%
%The most obvious way to downsample from a sequence of values is of course to simply pick
%each nth sample. However, this has some serious drawbacks which make it an unworkable solution
%in most cases.
%
%\textbf{Fancy Graphics of downsampling without LP filter with explanations}
%
%
%\subsection{Digital Low-Pass Filters}
%\label{subsec:digital-lp-filters}
%
%The obvious solution to this predicament is to apply a (digital) low-pass filter to the
%sequence of values before downsampling. For this purpose, three types of filters are commonly
%used, each with their own specific advantages and drawbacks: IIR, FIR, CIC.
%
%For theseandthose reasons, we will use FIR and CIC in our system.
%
%\textbf{Fancy graphics of LP filter, downsampling and folding back}
%
%%>>>
%
%\section{Designing a Filter System}
%\label{sec:designing-a-filter-system}
%
%Talking about which type of filter has which properties is all good and well in theory, but
%how does one actually apply this knowledge to a practical problem? This section answers that
%question insofar as it applies to our project.
%
%\begin{itemize}\tightlist
%    \item
%        limited HW resources
%    \item
%        single-stage vs. multi-stage
%    \item
%        TBW issue with multi-stage
%    \item
%        filters at lower frequencies use fewer resources
%    \item
%        halfband filtres
%    \item
%        CIC: compensation filters
%\end{itemize}

%^^A vim: foldenable foldcolumn=4 foldmethod=marker foldmarker=<<<,>>>
