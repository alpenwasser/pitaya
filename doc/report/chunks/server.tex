% ==============================================================================
%
%                                 S E R V E R
%
% ==============================================================================
\chapter{Server} % ----------------------------------------------------------- %
\label{ch:server}
% ---------------------------------------------------------------------------- %
% ==============================================================================
%
%                               O V E R V I E W
%
% ==============================================================================

Once the  FPGA has  recorded data, that  data has to  be transmitted  over the
network. Since implementing  networking in  hardware is  not feasible  in most
cases,  this  is  done  via  the  ARM  Cortex  A9  core  that  already  has  a
PHY\footnote{
    A chip implementing the physical layer of the OSI model%
}.
To  control all  the hardware  of  the SoC,  an embedded  Ubuntu GNU/Linux  is
running on the  ARM core which can control all  hardware components, including
the logger running on the FPGA.  An  application is then needed that reads the
necessary data  from the  RAM and sends  it to the  network. This part  of the
overall product is designated as  the \emph{server}. This section explains the
design choices and internal structure of the server application.

\section{Requirements} % <<< ------------------------------------------------- %
\label{sec:server:requirements}
% ---------------------------------------------------------------------------- %

The basic functional requirements of the server application are:
\begin{itemize}\tightlist
    \item
        Read the system status and transmit it over the network.
    \item
        Receive commands  over the  network, translate  them where  needed and
        relay them to the FPGA IP.
    \item
        Read data from the RAM and transmit it over the network.
\end{itemize}

%>>>
% ==============================================================================
%
%                         D E S I G N   C H O I C E S
%
% ==============================================================================
\section{Design Choices} % <<< ----------------------------------------------- %
\label{sec:server:design_choices}
% ---------------------------------------------------------------------------- %

\paragraph{As  the ZYNQ  Logger} comes  with a  kernel module  that has  to be
interfaced via \code{IOCTL} calls, it  is recommended to write the application
in C or C++. This is due to the nature of Linux, which still requires mostly C
for interfacing. There  are some \code{IOCTL}  interfaces in Python  and Rust,
for example,  but those bring additional  problems on ARM Linux  since not all
libraries and features are available.

Since  the  server application  is  rather  complex,  C++  is a  good  choice,
eliminating some of the caveats  which C has. Additionally, the entire feature
set of  C can be used  in C++ as well,  so the choice carries  no penalties in
terms of features.

\paragraph{The  WebSockets  protocol}  is  mandatory, due  to  the  choice  of
JavaScript on the client side.  This is  less of a design choice on the server
side,  and  more of  an  inherited  requirement  from  the front-end  part  of
the  project. For  this  purpose,  the  uWebSockets  (\emph{uWS})  library  is
chosen.  It carries a very small footprint and offers good performance, though
documentation is  somewhat lacking. Its performance  is high enough  to ensure
that in any given scenario, the  server application will not be the bottleneck
of the  overall data pipeline;  the network  connection will choke  before uWS
reaches the limits of its capabilities.

uWS  is  based  on  epoll,  libuv  or  boost::asio  depending  on  the  user's
choice. All  of  those are  asynchronous  (more  on  this  topic can  be  read
in~\cite{uws:async}) libraries  which makes  networking very  convenient. uWS
comes with  callbacks that  can be  registered for  each WebSockets  event (see
Section~\ref{subsec:gui:websockets}). Furthermore the  user can hook  into the
event loop and register other events such as a reocurring timeout (timer).

\paragraph{JSON} is used  as a data format for settings  and statistics. It is
our format of  choice because it has by  far the largest user base  of all the
available formats,  the specification  is simple and  JavaScript can  parse it
natively into a JavaScript object. This also another point where the choice of
C++ offers significant  benefits over C, since  it has a number  of high level
(i.e.  easy to use)  libraries  that  can  serialize  and,  more  importantly,
deserialize JSON objects.

%>>>
% ==============================================================================
%
%                         I M P L E M E N T A T I O N
%
% ==============================================================================
\section{Implementation} % <<< ----------------------------------------------- %
\label{sec:server:implementation}
% ---------------------------------------------------------------------------- %

While the server's  overall task is complex, its implementation  has been kept
as simple  as possible. Indeed, the application  fits into a single  file (not
counting the loaded libraries) of a  few hundred lines, and consists mainly of
a single asynchronous event loop.

The overall application state is stored  in a struct. This allows any event to
access any information  it might need, such as the  socket handle, the current
sampling rate,  or the requested number  of bytes.  Most of  the application's
functionality  can  be  deduced  from  its  event  loop,  which  is  shown  in
Figure~\ref{fig:server:eventstructure}.

\begin{figure}
    \centering
    \begin{tikzpicture}[%
        align=center,
        text=q1,
        draw=sq5,
    ]
    \small
    \sffamily

    \begin{scope}[
        every node/.style = draw,
        terminal/.append style={
            rounded rectangle,
            fill=sq5,
            text=q1,
            inner sep=2mm,
        }, % data packages
        sign/.style={
            inner sep=2mm,
            rounded corners=1mm,
            fill=sq5,
            text=q1,
        },         % custom signal style
        circ/.style={
            inner sep=2mm,
            rounded corners=1mm,
            double,
            fill=sq5,
            draw=q1,
            text=q1,
        }, % circuitry
        proc/.style={
            inner sep=2mm,
            rounded corners=1mm,
            fill=sq5,
            text=q1,
        },       % process/activity
        dec/.style={
            inner sep=2mm,
            rounded corners=1mm,
            fill=sq5,
            text=q1,
        },       % decision/activity
        stor/.style={
            fill=cyan!30
        },         % storage
        dbtable/.style={
            text=sq5,
            draw=q1,
            rounded corners=1mm,
            inner sep=2mm,
        } % database tables
    ]
        \node (newUART) [
            dec,
            align=center,
            diamond,
            minimum width=33mm,
            minimum height=33mm,
        ] at (0,0) {neuer\\UART\\Request?};
        \node (init) [
            proc,
            above=of newUART,
            align=center,
        ] {Hardware\\initialisieren};
        \node (forMe) [
            dec,
            align=center,
            diamond,
            right=of newUART,
            minimum width=33mm,
            minimum height=33mm,
        ] {F\"ur mich?};
        \node (package) [
            proc,
            right=of forMe,
            align=center,
        ] {Datenpaket mit\\Moving Average\\ und Addresse\\erstellen};
        \node (crc) [
            proc,
            below=of package,
            align=center,
        ] {CRC brechnen\\und zu Daten\\hinzuf\"ugen};
        \node (send) [
            proc,
            left=of crc,
            align=center,
        ] {Datenpaket\\senden};
        \node (getADC) [
            proc,
            below=of send,
            align=center,
        ] {ADC-Messung\\ausf\"uhren};
        \node (movAvg) [
            proc,
            below right=of getADC,
            align=center,
        ] {Moving\\Average\\berechnen};
        \node (LED) [
            proc,
            below left=of movAvg,
            align=center,
        ] {LED\\blinken\\lassen};
        \node (systick) [
            proc,
            above left=of LED,
            align=center,
        ] {auf n\"achsten\\\texttt{systick} warten};
    \end{scope}

    \begin{scope}[
            rounded corners,
            every path/.append style={draw=q1,},
            >=latex',
    ]
        \draw[-latex] (init) -- (newUART);
        \draw[-latex] (newUART) -- node[midway,anchor=south] {ja} (forMe);
        \draw[-latex] (forMe) -- node[midway,anchor=south] {ja} (package);
        \draw[-latex] (package) -- (crc);
        \draw[-latex] (crc) -- (send);
        \draw[-latex] (forMe) edge[bend left] node[midway,anchor=north] {nein} (newUART);
        \draw[-latex] (send) edge[bend left] (newUART);
        \draw (newUART) edge[loop below]  node {nein} (newUART);

        \draw[-latex] (getADC) edge[bend left] (movAvg);
        \draw[-latex] (movAvg) edge[bend left] (LED);
        \draw[-latex] (LED) edge[bend left] (systick);
        \draw[-latex] (systick) edge[bend left] (getADC);

    \end{scope}

\end{tikzpicture}
    \caption[Server Event Structure]{%
        The server's event structure.%
    }
    \label{fig:server:eventstructure}
\end{figure}
%>>>

%^^A vim: foldenable foldcolumn=4 foldmethod=marker foldmarker=<<<,>>>
