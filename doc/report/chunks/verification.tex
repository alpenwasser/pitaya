% ==============================================================================
%
%                           V E R I F I C A T I O N
%
% ==============================================================================
% ==============================================================================
%
%                               O V E R V I E W
%
% ==============================================================================
\chapter{Verification} % <<< ------------------------------------------------- %
\label{ch:verification}
% ---------------------------------------------------------------------------- %

Designing filters  in Matlab  and running  FPGA simulations  in Vivado  is all
nice  and well,  but  in the  end,  what  counts is  how  the system  performs
in  practice. This chapter  outlines  which sorts  of  measurements have  been
performed, the rationale behind them, and summarizes the results.

All  the metrics  presented in  this  chapter are  based  on the  same set  of
measurements, unless  otherwise mentioned. For  each filter  chain, \num{2000}
measurements have  been made, each  with a size  of \num{8192} samples  in the
time domain at one frequency. Between each of the \num{2000} runs for a single
chain, the signal frequency increases equidistantly from \num{0} to five times
the  filter's  outgoing  sampling frequency. Covering  \num{2000}  frequencies
allows to  cover the  passband ripple  and the  transition band  in sufficient
resolution to detect possible issues.

For the  filter chains  with $R=5$  decimators at their  ends (the  chains for
decimation factors by \num{5}, \num{25},  \num{125} and \num{625}), this means
that the output filter's passband is first traversed, and then its stopband up
until half its incoming sampling rate. This enables us to measure the passband
and  stopband magnitude  response. Furthermore, by  performing an  FFT of  the
filter chain's output at specific frequencies,  we can verify how strongly the
stopband aliases back into the passband. This is the problem illustrated in
Figures~\ref{fig:aliasing:iirCopies},
\ref{fig:cic:freq_responses:passband:aliasing},
and~\ref{fig:stl125:moving_averager}.

For the filter chains with half-band filters as their final stages (\num{1250}
and \num{2500}),  this frequency range is  actually too large, since  we would
only need to measure up to twice  the outgoing sampling rate in order to cover
the final filter's  magnitude response range.  However, there is  no real harm
in going higher (a bit of resolution is lost), and it simplifies the measuring
process, so no particular distinction is made here.

Note that it is generally not necessary to measure at even higher frequencies.
As can be seen in Section~\ref{sec:multi_stage_filter_designs} and some of the
frequency responses  for the filter  cascades which  are being used  (e.g. the
chain for  $R=25$, see page~\pageref{sec:filter_frequency_responses:chain25}),
it  is   the  final  filter  in   a  cascade  which  determines   the  chain's
overall   behavior   in  the   relevant   frequency   regions  (assuming   the
individual   filters  in   the   chains  have   been   sensibly  designed   to
match   each  other's   characteristics,   also  explained   in  the   section
\emph{\nameref{sec:multi_stage_filter_designs}}).    Increasing    the   input
frequency  further simply  moves into  regions  which are  ever more  strongly
attenuated,  so unless  serious issues  are encountered,  the higher  spectral
ranges are not of interest  in verifying the filter chain's functionality. The
measurements for the  \num{1250} and \num{2500} chains, which  do indeed cover
more than the necessary frequency range, confirm this.

All   measurements   are   adjusted   to   have   the   right   scaling   with
Equation~\ref{eqn:verification:transform}  where $x_\mathrm{unsigned}$  is the
measured sample.
\begin{align}
    \label{eqn:verification:transform}
    x_{i,\mathrm{signed}} &= \left(x_{i,\mathrm{unsigned}} - 2^{15}\right) \cdot 2^{-15} \\
    x_{i,\mathrm{V}}      &= x_{i,\mathrm{signed}} \cdot \SI{1.1}{\V}
\end{align}

% >>>
% ==============================================================================
%
%                R M S   F R E Q U E N C Y   R E S P O N S E
%
% ==============================================================================
\section{RMS Frequency Response} % <<< --------------------------------------- %
\label{sec:verification:rms}
% ---------------------------------------------------------------------------- %

To verify  that the  filters perform as  expected, the RMS  of all  samples is
calculated and  plotted against their  respective input frequency, as  seen in
Figure~\ref{fig:verification:rmsAll}.  The RMS looks  good and the passband is
practically flat. Two things stand out as not too positive:
\begin{itemize}\tightlist
    \item 
        The passband frequency  response has a slight slope  upwards. It is at
        no  point  higher  than  \SI{20}{\mV},  effectively  having  an  error
        below  \SI{2}{\percent}  in  all  of  the  filter  chains  except  the
        $R=5$  chain.  
        This is  mostly in  line with  the case in  the filter  performance as
        predicted  by  Matlab; a  ripple  in  the cascade's  passband  between
        \SI{0.67}{\percent} and \SI{2.8}{\percent} is predicted.
        What is  peculiar is that  the overall  shape of the  passband doesn't
        quite  align  with  Matlab's  designs. No  behavior  occurs  which  we
        consider to be deal-breaking though.
    \item 
        The passband of $R=5$ chain looks deformed, on the other hand, and not
        at all what  is expected. Something has gone  horribly wrong here. The
        previous statements do not apply for this case.
\end{itemize}
On the other hand, these results are as expected:
\begin{itemize}
    \item 
        The  Passband has  a  slight ripple  of  $\pm\SI{1}{\percent}$ in  all
        filter chains.  This is well within the filter design specification in
        theory.  Matlab predicts a  ripple in the cascade's passband between
        \SI{0.67}{\percent} and \SI{2.8}{\percent}, depending on the chain.
    \item 
        The  passband   shows  a   gain  loss. If  the   optimum  case   of  a
        \SI{2}{\bit}  win   were  achieved,   all  of  the   amplitudes  would
        be   at  $\frac{1}{\sqrt{2}}   =  \SI{0.707}{\V}$   (since  input   is
        \SI{2}{V_{\mathrm{PP}}}).    Since   not   all   filter   chains   win
        the   same   amount  of   bits,   this   sadly   is  not   the   case.
        Table~\ref{tab:verification:results} lists  the actual number  of bits
        won, along with  correction factors which need to be  applied to scale
        the output correctly.
\end{itemize}
Overall,  these  results  are  mostly acceptable  in  our  view,  particularly
considering that  this system is  in essence a  first prototype which  has not
gone through any performance tuning  yet. However, there is certainly room for
further investigation and improvement.

\begin{figure}
    \centering
    %\tikzsetnextfilename{rmsAll}
\newcommand*\freqzFileA{images/verification/results/calcs5.csv}
\pgfplotstableread[col sep=comma]{\freqzFileA}\freqzTableA

\newcommand*\freqzFileB{images/verification/results/calcs25.csv}
\pgfplotstableread[col sep=comma]{\freqzFileB}\freqzTableB

\newcommand*\freqzFileC{images/verification/results/calcs125.csv}
\pgfplotstableread[col sep=comma]{\freqzFileC}\freqzTableC

\newcommand*\freqzFileD{images/verification/results/calcs625.csv}
\pgfplotstableread[col sep=comma]{\freqzFileD}\freqzTableD

\newcommand*\freqzFileE{images/verification/results/calcs1250.csv}
\pgfplotstableread[col sep=comma]{\freqzFileE}\freqzTableE

\newcommand*\freqzFileF{images/verification/results/calcs2500.csv}
\pgfplotstableread[col sep=comma]{\freqzFileF}\freqzTableF


\begin{tikzpicture}[
    trim axis left,
    trim axis right,
]
    \pgfplotsset{every axis/.style={
            height=55mm,
            width=0.8\textwidth,
            grid=none,
            % y filter/.code={\pgfmathparse{20*log10(\pgfmathresult))}},
            % x filter/.code={\pgfmathparse{\pgfmathresult * 25e3}},
            xlabel=Frequency,
            ylabel=$V_\mathrm{RMS}$,
            x unit=\times\,\pi\,\si{\radian}/\si{\sample},
            y unit=\si{\V},
            ymajorgrids=true,
            xmajorgrids=true,
        },
        every axis legend/.append style={
            %at={(1,-0.15)},
            %anchor=north east,
            cells={anchor=west},
        },
        legend image post style={mark options={scale=1}},
    }
    \begin{axis}[
            at = {(0,0)},
            xmin=0,
            xmax=2000,
            xtick={0,400,800,1200,1600,2000},
            xticklabels={%
                $0$,
                $\frac{f_\mathrm{s}}{2}$,
                $2\frac{f_\mathrm{s}}{2}$,
                $3\frac{f_\mathrm{s}}{2}$,
                $4\frac{f_\mathrm{s}}{2}$,
                $5\frac{f_\mathrm{s}}{2}$%
            },
        ]
        % Scaling Factors
        % 1.1398 R = 2500
        % 1.1171 R = 1250
        % 1.2529 R =  625
        % 1.2787 R =  125
        % 1.5450 R =   25
        % 1.7278 R =    5

        %\addplot[thick,q1,-,smooth,only marks,mark options={scale=0.2},
        %    y filter/.code={\pgfmathparse{1.7278 * \pgfmathresult}},
        %] table[x=f, y=xrms] \freqzTableA;
        \addplot[thick,ct4,-,smooth,only marks,mark options={scale=0.2},
            y filter/.code={\pgfmathparse{1.5450 * \pgfmathresult}},
        ] table[x=f, y=xrms] \freqzTableB;
        \addplot[thick,q3,-,smooth,only marks,mark options={scale=0.2},
            y filter/.code={\pgfmathparse{1.2787 * \pgfmathresult}},
        ] table[x=f, y=xrms] \freqzTableC;
        \addplot[thick,q4,-,smooth,only marks,mark options={scale=0.2},
            y filter/.code={\pgfmathparse{1.2529 * \pgfmathresult}},
        ] table[x=f, y=xrms] \freqzTableD;
        \addplot[thick,q5,-,smooth,only marks,mark options={scale=0.2},
            y filter/.code={\pgfmathparse{1.1171 * \pgfmathresult}},
        ] table[x=f, y=xrms] \freqzTableE;
        \addplot[thick,q6,-,smooth,only marks,mark options={scale=0.2},
            y filter/.code={\pgfmathparse{1.1398 * \pgfmathresult}},
        ] table[x=f, y=xrms] \freqzTableF;

        \draw (rel axis cs:0,0.88375) -- (rel axis cs:1,0.88375);

        \legend{%
            %$R=5$,%
            $R=25$,%
            $R=125$,%
            $R=625$,%
            $R=1250$,%
            $R=2500$%
        }
    \end{axis}
    
\end{tikzpicture}

    \caption[RMS at Filter Chain Output]{%
        RMS  at   the  output   of  each   filter  chain   over  a   range  of
        frequencies. Input is a sine wave with $\SI{2}{V_{\mathrm{PP}}}$.%
    }
    \label{fig:verification:rmsAll}
\end{figure}


% >>>
% ==============================================================================
%
%                M E A N   F R E Q U E N C Y   R E S P O N S E
%
% ==============================================================================
\section{Mean Frequency Response} % <<< -------------------------------------- %
\label{sec:verification:mean}
% ---------------------------------------------------------------------------- %

Because it  became apparent during tests  that the device has  an offset (even
without filters!), the  mean of each sample has been  calculated too such that
at later stages  the offset can be calibrated out  in the scoping application.
The results are shown in Figure~\ref{fig:verification:meanAll}.

Achieving  a  good  SNR is  also  one  of  project's  goals, so  the  SNR  has
been  calculated  plotted   for  each  input  frequency  as   well,  shown  in
Figure~\ref{fig:verification:snrAll}.

\begin{figure}
    \centering
    \tikzsetnextfilename{meanAll}
\newcommand*\freqzFileA{images/verification/results/calcs5.csv}
\pgfplotstableread[col sep=comma]{\freqzFileA}\freqzTableA

\newcommand*\freqzFileB{images/verification/results/calcs25.csv}
\pgfplotstableread[col sep=comma]{\freqzFileB}\freqzTableB

\newcommand*\freqzFileC{images/verification/results/calcs125.csv}
\pgfplotstableread[col sep=comma]{\freqzFileC}\freqzTableC

\newcommand*\freqzFileD{images/verification/results/calcs625.csv}
\pgfplotstableread[col sep=comma]{\freqzFileD}\freqzTableD

\newcommand*\freqzFileE{images/verification/results/calcs1250.csv}
\pgfplotstableread[col sep=comma]{\freqzFileE}\freqzTableE

\newcommand*\freqzFileF{images/verification/results/calcs2500.csv}
\pgfplotstableread[col sep=comma]{\freqzFileF}\freqzTableF


\begin{tikzpicture}[
    trim axis left,
    trim axis right,
]
     \pgfplotsset{every axis/.style={
            height=45mm,
            width=\textwidth,
            grid=none,
            % y filter/.code={\pgfmathparse{20*log10(\pgfmathresult))}},
            % x filter/.code={\pgfmathparse{\pgfmathresult * 25e3}},
            xlabel=Frequency,
            ylabel=$V_{RMS}$,
            % x unit=\si{\Hz},
            y unit=\si{\V},
            % ytick={-80,-40,0,40},
        },
        every axis legend/.append style={
            %at={(1,-0.15)},
            %anchor=north east,
            cells={anchor=west},
        },
    }
    \begin{axis}[
            title={The Mean Frequency Response},
            at = {(0,0)},
            xmin=0,
            xmax=2000,
            %ymin=-120,
            %ymax=5,
            xtick={400,2000},
            xticklabels={$\frac{f_s}{2}$,$5\frac{f_s}{2}$},
            legend={%
                R=5,%
                R=25,%
                R=125,%
                R=625,%
                R=1250,%
                R=2500%
            }
        ]
        \addplot[thick,q1,-,smooth,only marks,mark options={scale=0.2}] table[x=f, y=xmean] \freqzTableA;
        \addplot[thick,q2,-,smooth,only marks,mark options={scale=0.2}] table[x=f, y=xmean] \freqzTableB;
        \addplot[thick,q3,-,smooth,only marks,mark options={scale=0.2}] table[x=f, y=xmean] \freqzTableC;
        \addplot[thick,q4,-,smooth,only marks,mark options={scale=0.2}] table[x=f, y=xmean] \freqzTableD;
        \addplot[thick,q5,-,smooth,only marks,mark options={scale=0.2}] table[x=f, y=xmean] \freqzTableE;
        \addplot[thick,q6,-,smooth,only marks,mark options={scale=0.2}] table[x=f, y=xmean] \freqzTableF;
    \end{axis}
    
\end{tikzpicture}

    \caption[Mean at Filter Outputs]{%
        The  mean  at  the  output  of  each filter  stage  over  a  range  of
        frequencies.  Since the  input signal is a sine wave  and the function
        generator which provides it has  been verified to have no (meaningful)
        offset, these lines should be located at zero volts.%
    }
    \label{fig:verification:meanAll}
\end{figure}
%>>>
% ==============================================================================
%
%                S N R   F R E Q U E N C Y   R E S P O N S E
%
% ==============================================================================
\section{SNR Frequency Response}% <<< ---------------------------------------- %
\label{sec:verification:snr}
% ---------------------------------------------------------------------------- %

Two series of measurements  are used for SNR: An automated one  by us which is
evaluated with  Matlab's SNR algorithms. Those  determine SNR for a  signal by
separating signal and  noise components, calculating the  respective power and
the ratio of signal power to noise power components\footnote{%
    This is similar to what our scope does.%
}.
This series of measurements  allows us to get a good idea of  the shape of the
system's overall frequency response for the SNR.

However, Matlab's SNR algorithms are  not perfect; the automated separation of
signal from  noise components  is prone  to errors. Consequently,  the results
achieved by  this method are  below the system's true  capabilities. To assess
actual  system performance,  SNR  measurements referenced  against a  measured
noise floor have also been performed  by Mr. Gut. For this, the noise floor is
measured by terminating  the device with a \SI{50}{\ohm}  resistor, instead of
extracting it from  a signal. Afterwards, a signal is put  through the system,
the power in  the signal is measured,  and its power compared  to the measured
noise floor. This method  is more precise than the automated  method, and also
happens to give better results.

Indeed,  when  comparing  our  new  system against  the  measurements  of  the
STEMlab's stock configuration by our predecessors\footnote{%
    Those  measurements  are also  referenced  against  a \SI{50}{\ohm}  noise
    floor.%
}, our system achieves an SNR of \SI{84}{\dB} for an input frequency of \SI{3}{\kHz}
at a sampling rate of \SI{50}{\kHz}! The stock configuration needs to downsample
by a factor of \num{65536} (corresponding to a sampling frequency of \SI{1.9}{\kHz}
in order to achieve an SNR of \SI{83.5}{\dB}\footnote{%
    The higher the downsampling ratio, the  better the achievable SNR tends to
    be.%
}.

\begin{figure}
    \centering
    \tikzsetnextfilename{snrAll}
\newcommand*\freqzFileA{images/verification/results/calcs5.csv}
\pgfplotstableread[col sep=comma]{\freqzFileA}\freqzTableA

\newcommand*\freqzFileB{images/verification/results/calcs25.csv}
\pgfplotstableread[col sep=comma]{\freqzFileB}\freqzTableB

\newcommand*\freqzFileC{images/verification/results/calcs125.csv}
\pgfplotstableread[col sep=comma]{\freqzFileC}\freqzTableC

\newcommand*\freqzFileD{images/verification/results/calcs625.csv}
\pgfplotstableread[col sep=comma]{\freqzFileD}\freqzTableD

\newcommand*\freqzFileE{images/verification/results/calcs1250.csv}
\pgfplotstableread[col sep=comma]{\freqzFileE}\freqzTableE

\newcommand*\freqzFileF{images/verification/results/calcs2500.csv}
\pgfplotstableread[col sep=comma]{\freqzFileF}\freqzTableF


\begin{tikzpicture}[
    trim axis left,
    trim axis right,
]
    \pgfplotsset{every axis/.style={
            height=55mm,
            width=\textwidth,
            grid=none,
            % y filter/.code={\pgfmathparse{20*log10(\pgfmathresult))}},
            % x filter/.code={\pgfmathparse{\pgfmathresult * 25e3}},
            xlabel=Frequency,
            ylabel=$\text{SNR}$,
            x unit=\times\,\pi\,\si{\radian}/\si{\sample},
            y unit=\si{\dB},
            ytick={120,90,60,30,0,-30},
            ymajorgrids=true,
            xmajorgrids=true,
        },
        every axis legend/.append style={
            %at={(1,-0.15)},
            %anchor=north east,
            cells={anchor=west},
        },
        legend image post style={mark options={scale=1}},
    }
    \begin{axis}[
            title={SNR Frequency Response},
            at = {(0,0)},
            xmin=0,
            xmax=2000,
            %ymin=-120,
            %ymax=5,
            xtick={0,400,800,1200,1600,2000},
            xticklabels={%
                $0$,
                $\frac{f_\mathrm{s}}{2}$,
                $2\frac{f_\mathrm{s}}{2}$,
                $3\frac{f_\mathrm{s}}{2}$,
                $4\frac{f_\mathrm{s}}{2}$,
                $5\frac{f_\mathrm{s}}{2}$%
            },
        ]
        \addplot[thick,q1,-,smooth,only marks,mark options={scale=0.2}] table[x=f, y=xsnr] \freqzTableA;
        \addplot[thick,ct4,-,smooth,only marks,mark options={scale=0.2}] table[x=f, y=xsnr] \freqzTableB;
        \addplot[thick,q3,-,smooth,only marks,mark options={scale=0.2}] table[x=f, y=xsnr] \freqzTableC;
        \addplot[thick,q4,-,smooth,only marks,mark options={scale=0.2}] table[x=f, y=xsnr] \freqzTableD;
        \addplot[thick,q5,-,smooth,only marks,mark options={scale=0.2}] table[x=f, y=xsnr] \freqzTableE;
        \addplot[thick,q6,-,smooth,only marks,mark options={scale=0.2}] table[x=f, y=xsnr] \freqzTableF;
        \legend{%
            $R=5$,%
            $R=25$,%
            $R=125$,%
            $R=625$,%
            $R=1250$,%
            $R=2500$%
        }
    \end{axis}
    
\end{tikzpicture}

    \caption[SNR at Filter Outputs]{%
        The  SNR  at  the  output  of  each  filter  stage  over  a  range  of
        frequencies.   Note the  downward  spikes in  the passband. These  are
        caused by  Matlab's algorithms during  post-processing and are  not an
        actual system issue.%
    }
    \label{fig:verification:snrAll}
\end{figure}

\begin{table}
    \centering
    \caption[SNR Referenced Against \SI{50}{\ohm}]{%
        SNR  measurements  referenced  against   a  true  \SI{50}{\ohm}  noise
        floor,  measured  by  Mr. Gut. The results  constitute  a  significant
        improvement   over   the   stock  system's   SNR   capabilities   (see
        Table~\ref{tab:stl125:measurements_bucher_kuery}). Only the  chain for
        $f_\mathrm{s}  = \SI{25}{\MHz}$  lies  outside  the expected  pattern;
        this  has already  been seen  in its  magnitude frequency  response in
        Section~\ref{sec:verification:rms}.%
    }
    \label{tab:verification:results}
    \begin{tabular}{SSSSSS}
        \toprule
        {\parbox[t]{10mm}{\raggedleft $f_\mathrm{s}$                       \\(\si{\kHz})}          } &
        {\parbox[t]{10mm}{\raggedleft $f_\mathrm{signal}$                  \\(\si{\kHz})}          } &
        {\parbox[t]{10mm}{\raggedleft $V_\mathrm{sine,RMS}$                \\(\si{\milli\volt})}   } &
        {\parbox[t]{15mm}{\raggedleft $V_\mathrm{noise,RMS}$               \\(\si{\milli\volt})}   } &
        {\parbox[t]{15mm}{\raggedleft $\mathrm{SNR}_\mathrm{\SI{50}{\ohm}}$\\(\si{\dB})}           } &
        {\parbox[t]{10mm}{\raggedleft $\mathrm{SNR}_\mathrm{autom.}$       \\(\si{\dB})}           } \\
        \midrule
           50 &    1 & 707 & 40e-6  & 84   & 71 \\
          100 &   25 & 707 & 55e-6  & 82   & 71 \\
          200 &   50 & 708 & 59e-6  & 81.5 & 69 \\
         1000 &  250 & 708 & 75e-6  & 79.5 & 69 \\
         5000 & 1250 & 707 & 130e-3 & 79.5 & 69 \\
        25000 & 6250 & 707 & 260e-6 & 62   & 69 \\
        \bottomrule
    \end{tabular}
\end{table}

% >>>
% ==============================================================================
%
%                   S T O P B A N D   A T T E N U A T I O N
%
% ==============================================================================
\section{Stopband Attenuation} % <<< ----------------------------------------- %
\label{sec:verification:snr}
% ---------------------------------------------------------------------------- %

Measuring stopband attenuation  is relevant in order to verify  whether or not
the aliasing effect  of the stopband into the passband  during downsampling is
attenuated to the specified degree of \SI{60}{\dB}. See
Figures~\ref{fig:aliasing:iirCopies},
\ref{fig:cic:freq_responses:passband:aliasing}
and~\ref{fig:stl125:moving_averager} for illustrations of the phenomenon.

For this purpose,  the power density spectrum is plotted
properly scaled and adjusted  as seen in Equation~\ref{eqn:verification:power}
through~\ref{eqn:verification:power_density}:
\begin{align}
    x_{i,\mathrm{corrected}} &= x_{i,\mathrm{V}} \cdot \sqrt{\frac{1}{2f_s N}} \label{eqn:verification:power} \\
    X                        &= FFT\left(x_{i,\mathrm{corrected}}\right)       \\
    X_{i,\mathrm{one}}       &= X_i \cdot 2, i < \frac{N}{2}+1                 \\
    X_{i,\mathrm{abs}}       &= |X_{i,\mathrm{one}}|                           \\
    S_{\si{\dB}}             &= 10\log_{10}(X_{i,\mathrm{abs}}^2)              \label{eqn:verification:power_density}
\end{align}

Figures~\ref{fig:verification:fB5}  through \ref{fig:verification:fB5}  depict
the results. Each plot  contains one frequency which falls  into the passband,
one which falls into the filter's edge, and one which falls into the stopband.
The passband and edge frequency components  are expected to be relatively high
(depending  on where  exactly in  the edge  the measurement  point falls). The
measurement point  in the stopband  should be \SI{60}{\dB} below  the passband
measurement point. Most satisfactorily, this is  the case, and the measurement
results line up very nicely with the specifications.

\begin{figure}
    \centering
    \tikzsetnextfilename{foldingBack5}
\newcommand*\freqzFileA{images/verification/results/attenuation5.csv}
\pgfplotstableread[col sep=comma]{\freqzFileA}\freqzTableA


\begin{tikzpicture}[
    trim axis left,
    trim axis right,
]
     \pgfplotsset{every axis/.style={
            height=45mm,
            width=\textwidth,
            grid=none,
            % y filter/.code={\pgfmathparse{20*log10(\pgfmathresult))}},
            % x filter/.code={\pgfmathparse{\pgfmathresult * 25e3}},
            xlabel=Frequency,
            ylabel=$V_{RMS}$,
            % x unit=\si{\Hz},
            y unit=\si{\V},
            % ytick={-80,-40,0,40},
        },
        every axis legend/.append style={
            %at={(1,-0.15)},
            %anchor=north east,
            cells={anchor=west},
        },
    }
    \begin{axis}[
            title={Attenuation in the Edge and Stopband in Contrast to the Passband for R=5},
            at = {(0,0)},
            xmin=0,
            xmax=4096,
            %ymin=-120,
            %ymax=5,
            xtick={4096},
            xticklabels={$\frac{f_s}{2}$},
            legend={%
                $5\frac{f_s}{2}\cdot \frac{10}{200}$ (Passband),%
                $5\frac{f_s}{2}\cdot \frac{41}{200}$ (Edge),%
                $5\frac{f_s}{2}\cdot \frac{70}{200}$ (Stopband),%
            },
        ]
        \addplot[thick,q1,-,smooth,only marks,mark options={scale=0.2}] table[x=f, y=s100] \freqzTableA;
        \addplot[thick,q2,-,smooth,only marks,mark options={scale=0.2}] table[x=f, y=s410] \freqzTableA;
        \addplot[thick,q3,-,smooth,only marks,mark options={scale=0.2}] table[x=f, y=s700] \freqzTableA;
    \end{axis}
    
\end{tikzpicture}

    \caption[Attenuation in the Edge and Stopband in Contrast to the Passband for R=5]{%
        Attenuation in the edge and stopband in contrast to the passband for $R=5$%
    }
    \label{fig:verification:fB5}
\end{figure}

\begin{figure}
    \centering
    \tikzsetnextfilename{foldingBack25}
\newcommand*\freqzFileA{images/verification/results/attenuation25.csv}
\pgfplotstableread[col sep=comma]{\freqzFileA}\freqzTableA


\begin{tikzpicture}[
    trim axis left,
    trim axis right,
]
    \pgfplotsset{every axis/.style={
            height=45mm,
            width=\textwidth,
            grid=none,
            % y filter/.code={\pgfmathparse{20*log10(\pgfmathresult))}},
            % x filter/.code={\pgfmathparse{\pgfmathresult * 25e3}},
            xlabel=Frequency,
            ylabel=$S_{xx}$,
            % x unit=\si{\Hz},
            y unit=\si{\dB},
            x unit=\times\,\pi\,\si{\radian}/\si{\sample},
            % ytick={-80,-40,0,40},
            legend columns=3,
        },
        every axis legend/.append style={
            at={(0.995,0.02)},
            anchor=south east,
            cells={anchor=west},
        },
    }
    \begin{axis}[
            title={Attenuation in the Edge and Stopband in Contrast to the Passband for $R=25$},
            at = {(0,0)},
            xmin=0,
            xmax=4096,
            %ymin=-120,
            %ymax=5,
            xtick={4096},
            xticklabels={$\frac{f_\mathrm{s}}{2}$},
        ]
        \addplot[thick,q1,-,smooth,only marks,mark options={scale=0.2}] table[x=f, y=s100] \freqzTableA;
        \addplot[thick,ct4,-,smooth,only marks,mark options={scale=0.2}] table[x=f, y=s410] \freqzTableA;
        \addplot[thick,q5,-,smooth,only marks,mark options={scale=0.2}] table[x=f, y=s700] \freqzTableA;
        \legend{%
            $f_\mathrm{in}=5\frac{f_\mathrm{s}}{2}\cdot \frac{10}{200}$ (Passband),%
            $f_\mathrm{in}=5\frac{f_\mathrm{s}}{2}\cdot \frac{41}{200}$ (Edge),%
            $f_\mathrm{in}=5\frac{f_\mathrm{s}}{2}\cdot \frac{70}{200}$ (Stopband),%
        }
    \end{axis}
    
\end{tikzpicture}

    \caption[Attenuation in the Edge and Stopband in Contrast to the Passband for R=25]{%
        Attenuation in the edge and stopband in contrast to the passband for $R=25$%
    }
    \label{fig:verification:fB25}
\end{figure}

\begin{figure}
    \centering
    \tikzsetnextfilename{foldingBack125}
\newcommand*\freqzFileA{images/verification/results/attenuation125.csv}
\pgfplotstableread[col sep=comma]{\freqzFileA}\freqzTableA


\begin{tikzpicture}[
    trim axis left,
    trim axis right,
]
    \pgfplotsset{every axis/.style={
            height=45mm,
            width=\textwidth,
            grid=none,
            % y filter/.code={\pgfmathparse{20*log10(\pgfmathresult))}},
            % x filter/.code={\pgfmathparse{\pgfmathresult * 25e3}},
            xlabel=Frequency,
            ylabel=$S_{xx}$,
            % x unit=\si{\Hz},
            y unit=\si{\dB},
            x unit=\times\,\pi\,\si{\radian}/\si{\sample},
            % ytick={-80,-40,0,40},
        },
        every axis legend/.append style={
            at={(0.96,0.07)},
            anchor=south east,
            cells={anchor=west},
        },
    }
    \begin{axis}[
            title={Attenuation in the Edge and Stopband in Contrast to the Passband for R=125},
            at = {(0,0)},
            xmin=0,
            xmax=4096,
            %ymin=-120,
            %ymax=5,
            xtick={4096},
            xticklabels={$\frac{f_s}{2}$},
        ]
        \addplot[thick,q1,-,smooth,only marks,mark options={scale=0.2}] table[x=f, y=s100] \freqzTableA;
        \addplot[thick,ct4,-,smooth,only marks,mark options={scale=0.2}] table[x=f, y=s410] \freqzTableA;
        \addplot[thick,q5,-,smooth,only marks,mark options={scale=0.2}] table[x=f, y=s700] \freqzTableA;
        \legend{%
            $f_{in}=5\frac{f_s}{2}\cdot \frac{10}{200}$ (Passband),%
            $f_{in}=5\frac{f_s}{2}\cdot \frac{41}{200}$ (Edge),%
            $f_{in}=5\frac{f_s}{2}\cdot \frac{70}{200}$ (Stopband),%
        }
    \end{axis}
    
\end{tikzpicture}

    \caption[Attenuation in the Edge and Stopband in Contrast to the Passband for R=125]{%
        Attenuation in the edge and stopband in contrast to the passband for $R=125$%
    }
    \label{fig:verification:fB125}
\end{figure}

\begin{figure}
    \centering
    \tikzsetnextfilename{foldingBack625}
\newcommand*\freqzFileA{images/verification/results/attenuation625.csv}
\pgfplotstableread[col sep=comma]{\freqzFileA}\freqzTableA


\begin{tikzpicture}[
    trim axis left,
    trim axis right,
]
    \pgfplotsset{every axis/.style={
            height=55mm,
            width=\textwidth,
            grid=none,
            % y filter/.code={\pgfmathparse{20*log10(\pgfmathresult))}},
            % x filter/.code={\pgfmathparse{\pgfmathresult * 25e3}},
            xlabel=Frequency,
            ylabel=$S_{xx}$,
            % x unit=\si{\Hz},
            y unit=\si{\dB},
            x unit=\times\,\pi\,\si{\radian}/\si{\sample},
            ymax=0,ymin=-180,
            ytick={0,-30,-60,-90,-120,-150,-180},
            ymajorgrids=true,
        },
        every axis legend/.append style={
            at={(0,-0.1)},
            anchor=north west,
            cells={anchor=west},
            /tikz/column 1/.style={column sep= 5pt},
            /tikz/column 2/.style={column sep=10pt},
            /tikz/column 3/.style={column sep= 5pt},
            /tikz/column 4/.style={column sep=10pt},
            /tikz/column 5/.style={column sep= 5pt},
        },
        legend image post style={mark options={scale=1}},
    }
    \begin{axis}[
            title={Attenuation in Passband, Edge and Stopband for $R=625$},
            at = {(0,0)},
            xmin=0,
            xmax=4096,
            %ymin=-120,
            %ymax=5,
            xtick={4096},
            xticklabels={$\frac{f_\mathrm{s}}{2}$},
        ]
        \addplot[thick,q1,-,smooth,only marks,mark options={scale=0.2}] table[x=f, y=s100] \freqzTableA;
        \addplot[thick,ct4,-,smooth,only marks,mark options={scale=0.2}] table[x=f, y=s410] \freqzTableA;
        \addplot[thick,q5,-,smooth,only marks,mark options={scale=0.2}] table[x=f, y=s700] \freqzTableA;
        \legend{%
            $f_\mathrm{in}=5\frac{f_\mathrm{s}}{2}\cdot \frac{10}{200}$ (Passband),%
            $f_\mathrm{in}=5\frac{f_\mathrm{s}}{2}\cdot \frac{41}{200}$ (Edge),%
            $f_\mathrm{in}=5\frac{f_\mathrm{s}}{2}\cdot \frac{70}{200}$ (Stopband),%
        }
    \end{axis}
    
\end{tikzpicture}

    \caption[Attenuation in the Edge and Stopband in Contrast to the Passband for R=625]{%
        Attenuation in the edge and stopband in contrast to the passband for $R=625$%
    }
    \label{fig:verification:fB625}
\end{figure}

\begin{figure}
    \centering
    \tikzsetnextfilename{foldingBack1250}
\newcommand*\freqzFileA{images/verification/results/attenuation1250.csv}
\pgfplotstableread[col sep=comma]{\freqzFileA}\freqzTableA


\begin{tikzpicture}[
    trim axis left,
    trim axis right,
]
    \pgfplotsset{every axis/.style={
            height=45mm,
            width=\textwidth,
            grid=none,
            % y filter/.code={\pgfmathparse{20*log10(\pgfmathresult))}},
            % x filter/.code={\pgfmathparse{\pgfmathresult * 25e3}},
            xlabel=Frequency,
            ylabel=$S_{xx}$,
            % x unit=\si{\Hz},
            y unit=\si{\dB},
            x unit=\times\,\pi\,\si{\radian}/\si{\sample},
            % ytick={-80,-40,0,40},
        },
        every axis legend/.append style={
            at={(0.96,0.07)},
            anchor=south east,
            cells={anchor=west},
        },
    }
    \begin{axis}[
            title={Attenuation in the Edge and Stopband in Contrast to the Passband for R=1250},
            at = {(0,0)},
            xmin=0,
            xmax=4096,
            %ymin=-120,
            %ymax=5,
            xtick={4096},
            xticklabels={$\frac{f_s}{2}$},
        ]
        \addplot[thick,q1,-,smooth,only marks,mark options={scale=0.2}] table[x=f, y=s100] \freqzTableA;
        \addplot[thick,ct4,-,smooth,only marks,mark options={scale=0.2}] table[x=f, y=s410] \freqzTableA;
        \addplot[thick,q5,-,smooth,only marks,mark options={scale=0.2}] table[x=f, y=s700] \freqzTableA;
        \legend{%
            $f_{in}=5\frac{f_s}{2}\cdot \frac{10}{200}$ (Passband),%
            $f_{in}=5\frac{f_s}{2}\cdot \frac{41}{200}$ (Edge),%
            $f_{in}=5\frac{f_s}{2}\cdot \frac{70}{200}$ (Stopband),%
        }
    \end{axis}
    
\end{tikzpicture}

    \caption[Attenuation in the Edge and Stopband in Contrast to the Passband for R=1250]{%
        Attenuation in the edge and stopband in contrast to the passband for $R=1250$%
    }
    \label{fig:verification:fB5}
\end{figure}

\begin{figure}
    \centering
    \tikzsetnextfilename{foldingBack2500}
\newcommand*\freqzFileA{images/verification/results/attenuation2500.csv}
\pgfplotstableread[col sep=comma]{\freqzFileA}\freqzTableA


\begin{tikzpicture}[
    trim axis left,
    trim axis right,
]
    \pgfplotsset{every axis/.style={
            height=45mm,
            width=\textwidth,
            grid=none,
            % y filter/.code={\pgfmathparse{20*log10(\pgfmathresult))}},
            % x filter/.code={\pgfmathparse{\pgfmathresult * 25e3}},
            xlabel=Frequency,
            ylabel=$S_{xx}$,
            % x unit=\si{\Hz},
            y unit=\si{\dB},
            x unit=\times\,\pi\,\si{\radian}/\si{\sample},
            % ytick={-80,-40,0,40},
            legend columns=3,
        },
        every axis legend/.append style={
            at={(0.995,0.02)},
            anchor=south east,
            cells={anchor=west},
        },
    }
    \begin{axis}[
            title={Attenuation in the Edge and Stopband in Contrast to the Passband for $R=2500$},
            at = {(0,0)},
            xmin=0,
            xmax=4096,
            %ymin=-120,
            %ymax=5,
            xtick={4096},
            xticklabels={$\frac{f_\mathrm{s}}{2}$},
        ]
        \addplot[thick,q1,-,smooth,only marks,mark options={scale=0.2}] table[x=f, y=s100] \freqzTableA;
        \addplot[thick,ct4,-,smooth,only marks,mark options={scale=0.2}] table[x=f, y=s410] \freqzTableA;
        \addplot[thick,q5,-,smooth,only marks,mark options={scale=0.2}] table[x=f, y=s700] \freqzTableA;
        \legend{%
            $f_\mathrm{in}=5\frac{f_\mathrm{s}}{2}\cdot \frac{10}{200}$ (Passband),%
            $f_\mathrm{in}=5\frac{f_\mathrm{s}}{2}\cdot \frac{41}{200}$ (Edge),%
            $f_\mathrm{in}=5\frac{f_\mathrm{s}}{2}\cdot \frac{70}{200}$ (Stopband),%
        }
    \end{axis}
    
\end{tikzpicture}

    \caption[Attenuation in the Edge and Stopband in Contrast to the Passband for R=2500]{%
        Attenuation in the edge and stopband in contrast to the passband for $R=2500$%
    }
    \label{fig:verification:fB5}
\end{figure}

% >>>
% ==============================================================================
%
%                                S U M M A R Y
%
% ==============================================================================
\section{Summary} % <<< ------------------------------------------------------ %
\label{sec:verification:summary}
% ---------------------------------------------------------------------------- %

In  conclusion, the  filter  chains  perform their  task  mostly according  to
specifications. The main point  of concern is the stark passband  droop in the
chain for $R=5$ chain. More investigating is  needed to determine the cause of
this behavior. But overall, we consider it to be a success.

Table~\ref{tab:verification:results}  lists  the  resulting  mean  values  for
comparison  against each  other. \emph{SNR  won} is  the theoretical  SNR gain
caused by the bits which are gained  along the processing chain. It is not the
actually achieved  improvement in SNR  as other factors  play a role  as well.
\emph{Correction} is the correction factor which  needs to be applied in order
to achieve the proper gain (\SI{0.707}{\volt} RMS for a \SI{2}{\V_\mathrm{PP}}
signal, see Section~\ref{sec:verification:rms}).

\begin{table}
    \centering
    \caption[Mean Metrics for All Filter Chains]{Mean metrics for all filter chains}
    \label{tab:verification:results}
    \begin{tabular}{rrrrrrr}
        \toprule
        {\scshape $R$                 }& 
        {\scshape $V_\mathrm{RMS}$ (\si{V})  }& 
        {\scshape $V_\mathrm{Mean}$ (\si{V}) }& 
        {\scshape $S_\mathrm{SNR}$ (\si{dB}) }&  % TODO: unit correct?
        {\parbox[t]{16mm}{\raggedleft\scshape Corr.\\(\si{1})}}& 
        {\parbox[t]{16mm}{\raggedleft\scshape Bits\\won (\si{1})}}& 
        {\parbox[t]{16mm}{\raggedleft\scshape SNR\\won (\si{dB})}}\\
        \midrule
        5           & 0.6203   & -0.1800   & 79.0054   & 1.1398   & 1.8113   & 10.9038\\
        25          & 0.6329   & -0.1690   & 76.9049   & 1.1171   & 1.8403   & 11.0784\\
        125         & 0.5643   & -0.0159   & 73.6582   & 1.2529   & 1.6748   & 10.0820\\
        625         & 0.5529   & -0.0155   & 71.8813   & 1.2787   & 1.6453   & 9.9048\\
        1250        & 0.4576   & -0.0130   & 69.7006   & 1.5450   & 1.3724   & 8.2617\\
        2500        & 0.4092   & -0.0127   & 63.5121   & 1.7278   & 1.2111   & 7.2908\\
        \bottomrule
    \end{tabular}
\end{table}

% >>>

%^^A vim: foldenable foldcolumn=4 foldmethod=marker foldmarker=<<<,>>>
