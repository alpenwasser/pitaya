\chapter{Filter Design} % <<< ------------------------------------------------ %
\label{ch:filter_design}
% ---------------------------------------------------------------------------- %

Some    key    points    underlying     the    theory    of    filters    have
been   treated   in  Chapter~\ref{ch:analog-to-digital_data_aquisition},   and
Chapter~\ref{ch:mission}  defines  the  overall objectives  of  this  project;
implementing a  custom filtering system  on the FPGA among  them. This chapter
will now develop a concrete concept for that filtering system, address some of
the issues encountered when moving from  the theory of filters to the practice
of designing them, and then define the  actual filters which are to be used on
the FPGA. The implementation of those filters  on the FPGA is addressed in the
following chapter, beginning on page~\pageref{ch:fpga}.


\section{Requirements} % <<< ------------------------------------------------- %
\label{sec:requirements}
% ---------------------------------------------------------------------------- %

The overarching  objective is downsampling the  signal coming out of  the ADC,
This  section will  derive  upper  and lower  boundaries  for the  downsampled
frequency range, and then define the specific downsampling ratios to be used.

The total data rate from the ADC is
\sisetup{inter-unit-product = \ensuremath { { } \cdot { } }}
\begin{alignat}{4}
    S                  &= \SI{125}{\mega\sample\per\second}                                  & & \nonumber  \\
    N_\mathrm{ch}      &= 2                                                                  & & \nonumber  \\
    B_\mathrm{ch}      &= \SI{14}{\bit\per\sample}                                           & & \nonumber  \\
    B_\mathrm{ch,pad}  &= \SI{2}{\bit\per\sample}                                            & & \nonumber  \\
    B_\mathrm{ADC}     &= N_\mathrm{ch} \cdot \left(B_\mathrm{ch} + B_\mathrm{ch,pad}\right) \,\,&=&\,\, \SI{32}{\bit\per\sample} \nonumber \\
    R                  &= S \cdot B_\mathrm{ADC}                                             \,\,&=&\,\, \SI{4}{\giga\bit\per\second} \label{eq:adc_data_rate}
\end{alignat}
%\sisetup{inter-unit-product = \,}
where
\begin{align*}
    S                  &: \text{sampling rate}                        \\
    N_\mathrm{ch}      &: \text{number of channels}                   \\
    B_\mathrm{ch}      &: \text{channel width}                        \\
    B_\mathrm{ch,pad}  &: \text{padding per channel}                  \\
    B_\mathrm{ADC}     &: \text{total width of bit stream out of ADC} \\
    R                  &: \text{total data rate out of ADC in bit}
\end{align*}

The   upper   boundary  for   the   resulting   sampling   rate  is   set   by
the   STEMlab's   network    connection,   which   has   a    data   rate   of
\SI{1000}{\mega\bit\per\second}. A downsampling factor of  at least \num{4} is
therefore required  for real-time data transmission. Because  \num{125} is not
divisible by \num{4}, a factor of  \num{5} is chosen instead. This makes for a
resulting data  rate of  \SI{800}{\mega\bit\per\second}, which should  also be
easily sufficient for protocol overhead.

On the lower end  of the spectrum, the system should still  be able to process
audio  signals. Common  sampling  frequencies for  audio  are  \SI{44.1}{\kHz}
for  audio applications  like audio  CDs, and  \SI{48}{\kHz} for  audio-visual
applications. Neither of these frequencies fit very nicely into \SI{125}{\MHz}
(requiring large  prime factor  for the  rate change),  so the  lower boundary
is  specified  as \SI{50}{\kHz},  corresponding  to  a downsampling  ratio  of
\num{2500}.

To cover additional use cases, it is  desirable to specify a few more sampling
frequencies between  these two boundaries. Table~\ref{tab:downsampling_ratios}
contains  the   complete  list   of  downsampling   ratios,  along   with  the
corresponding sampling frequencies.

\begin{table}
    \centering
    \caption[List of Downsampling Ratios and Resultant Frequencies]{
        The list  of downsampling  ratios which are  to be  implemented, along
        with the resultant sampling frequencies.%
    }
    \label{tab:downsampling_ratios}
    \begin{tabular}{SS}
        \toprule
        {R} & {$f_s$ (\si{\kHz})} \\
        \midrule
           5 & 25000 \\
          25 &  5000 \\
         125 &  1000 \\
         625 &   200 \\
        1250 &   100 \\
        2500 &    50 \\
        \bottomrule
    \end{tabular}
\end{table}
%>>>


\section{Concept} % <<< ------------------------------------------------------ %
\label{sec:Concept}
% ---------------------------------------------------------------------------- %

Based  on the  downsampling factors  from Table~\ref{tab:downsampling_ratios},
this section presents the concept for a filtering sytem which implements those
ratios. This  process needs  to take  into  account contraints  both from  the
filtering process itself as well as available resources on the hardware.



%>>>
\subsection{Filter Chains} % <<< --------------------------------------------- %
\label{subsec:filter_chains}
% ---------------------------------------------------------------------------- %

%>>>

%>>>

%
%\section{Designing a Filter System}
%\label{sec:designing-a-filter-system}
%
%Talking about which type of filter has which properties is all good and well in theory, but
%how does one actually apply this knowledge to a practical problem? This section answers that
%question insofar as it applies to our project.
%
%\begin{itemize}\tightlist
%    \item
%        limited HW resources
%    \item
%        single-stage vs. multi-stage
%    \item
%        TBW issue with multi-stage
%    \item
%        filters at lower frequencies use fewer resources
%    \item
%        halfband filtres
%    \item
%        CIC: compensation filters
%\end{itemize}

%^^A vim: foldenable foldcolumn=4 foldmethod=marker foldmarker=<<<,>>>
