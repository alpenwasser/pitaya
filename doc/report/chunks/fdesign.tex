% ==============================================================================
%
%                          F I L T E R   D E S I G N
%
% ==============================================================================
\chapter{Filter Design} % ---------------------------------------------------- %
\label{ch:filter_design}
% ---------------------------------------------------------------------------- %
% ==============================================================================
%
%                               O V E R V I E W
%
% ==============================================================================

Some    key    points    underlying     the    theory    of    filters    have
been   treated   in  Chapter~\ref{ch:analog-to-digital_data_aquisition},   and
Chapter~\ref{ch:mission}  defines  the  overall objectives  of  this  project;
implementing a custom  filtering system on the FPGA among  them.  This chapter
develops a concrete  concept for that filtering system, addresses  some of the
issues encountered when  moving from the theory of filters  to the practice of
designing  them,  and  specifies  the  filters which  are  to  be  used.   The
implementation of those filters on the  FPGA is addressed in the next chapter,
beginning on page~\pageref{ch:fpga}.


% ==============================================================================
%
%                           R E Q U I R E M E N T S
%
% ==============================================================================
\section{Requirements} % <<< ------------------------------------------------- %
\label{sec:requirements}
% ---------------------------------------------------------------------------- %

The overarching objective  is to downsample the signal coming  out of the ADC.
In this  section, we  derive upper  and lower  boundaries for  the downsampled
frequency range, and then define the specific downsampling ratios to be used.

The upper  boundary for the  resulting sampling rate  is set by  the STEMlab's
network connection,  which has a capacity  of \SI{1000}{\mega\bit\per\second}.
The total data rate from the ADC is
\sisetup{inter-unit-product = \ensuremath { { } \cdot { } }}
\begin{alignat}{4}
    S                  &= \SI{125}{\mega\sample\per\second}                                  & & \nonumber  \\
    N_\mathrm{ch}      &= 2                                                                  & & \nonumber  \\
    B_\mathrm{ch}      &= \SI{14}{\bit\per\sample}                                           & & \nonumber  \\
    B_\mathrm{ch,pad}  &= \SI{2}{\bit\per\sample}                                            & & \nonumber  \\
    B_\mathrm{ADC}     &= N_\mathrm{ch} \cdot \left(B_\mathrm{ch} + B_\mathrm{ch,pad}\right) \,\,&=&\,\, \SI{32}{\bit\per\sample} \nonumber \\
    R                  &= S \cdot B_\mathrm{ADC}                                             \,\,&=&\,\, \SI{4}{\giga\bit\per\second} \label{eq:adc_data_rate}
\end{alignat}
\begin{conditions}
    S                  & sampling rate                         \\
    N_\mathrm{ch}      & number of channels                    \\
    B_\mathrm{ch}      & channel width                         \\
    B_\mathrm{ch,pad}  & padding per channel                   \\
    B_\mathrm{ADC}     & total width of bit stream out of ADC  \\
    R                  & total data rate out of ADC in bit     \\
\end{conditions}
%\sisetup{inter-unit-product = \,}
Comparing  the ADC's  data  rate from  Equation~\ref{eq:adc_data_rate} to  the
network's available  capacity, a  downsampling factor of  at least  \num{4} is
required for  real-time data transmission. Because \num{125}  is not divisible
by \num{4}  and some room  for protocol overhead is  needed as well,  a factor
of  \num{5}  is chosen  instead. This  makes  for  a  resulting data  rate  of
\SI{800}{\mega\bit\per\second}.

On the lower end  of the spectrum, the system should still  be able to process
audio signals. Common  sampling frequencies for audio  are \SI{44.1}{\kHz} for
compact  discs, and  \SI{48}{\kHz}  for the  audio  component of  audio-visual
applications. Neither  of these  frequencies  fit  nicely into  \SI{125}{\MHz}
(requiring large  prime factor  for the  rate change),  so the  lower boundary
is  specified  as \SI{50}{\kHz},  corresponding  to  a downsampling  ratio  of
\num{2500}.

To cover a  wider range of use cases, additional  sampling frequencies between
these   two  boundaries   are  specified. Table~\ref{tab:ratio_decompositions}
contains  the   complete  list   of  downsampling   ratios,  along   with  the
corresponding sampling frequencies.

\begin{table}
    \centering
    \caption[Downsampling Ratios, Decompositions, and Target Frequencies]{
        The chosen downsampling ratios, their prime factor decompositions, the
        downsampling  ratios  distributed  across stages,  and  the  resultant
        sampling rates%
    }
    \label{tab:ratio_decompositions}
    \begin{tabular}{S >{$}r<{$} >{$}l<{$} S}
        \toprule
        {R}  & \text{Decomposition}                                                 & \text{Stages}                                 & {$f_s$ (\si{\kHz})} \\
        \midrule
           5 &  5                                                   = 5^1           &   5                                           & 25000               \\
          25 &  5 \cdot 5                                           = 5^2           &   5 \rightarrow 5                             &  5000               \\
         125 &  5 \cdot 5 \cdot 5                                   = 5^3           &  25 \rightarrow 5                             &  1000               \\
         625 &  5 \cdot 5 \cdot 5 \cdot 5 \cdot 5                   = 5^4           &  25 \rightarrow 5 \rightarrow 5               &   200               \\
        1250 &  2 \cdot 5 \cdot 5 \cdot 5 \cdot 5 \cdot 5           = 2^1 \cdot 5^4 & 125 \rightarrow 5 \rightarrow 2               &   100               \\
        2500 &  2 \cdot 2 \cdot 5 \cdot 5 \cdot 5 \cdot 5 \cdot 5   = 2^1 \cdot 5^4 & 125 \rightarrow 5 \rightarrow 2 \rightarrow 2 &    50               \\
        \bottomrule
    \end{tabular}
\end{table}


%>>>
% ==============================================================================
%
%                        C A S C A D E   C O N C E P T
%
% ==============================================================================
\section{Cascade Concept} % <<< ---------------------------------------------- %
\label{sec:cascade_concept}
% ---------------------------------------------------------------------------- %

Based on  the downsampling factors  from Table~\ref{tab:ratio_decompositions},
this section presents the general  concept for filter cascades which implement
those rate changes.

As  discussed  in  Section~\ref{sec:multi_stage_filter_designs},  implementing
high downsampling  ratios in a  single stage is  generally not a  sound design
choice. Consequently, the downsampling ratios  must be decomposed into smaller
factors. Table~\ref{tab:ratio_decompositions}  contains  prime  decompositions
for all ratios, and then recombines the  primes into factors which can be used
as downsampling  ratios in a  multi-stage design.  These factors  must fulfill
the following criteria:
\begin{itemize}\tightlist
    \item
        Filters  for  different stages  should  be  re-usable across  multiple
        downsampling ratios  in order  to save resources.
    \item
        The factors for the individual stages  should be large enough to be of
        utility,  but small  enough so  as not  to make  the resulting  filter
        impractically narrow and large.
    \item
        No stage  should be preceded  by a  stage with a  smaller downsampling
        ratio than it has. The cascades go from larger ratios to smaller ones,
        as outlined in Section~\ref{sec:multi_stage_filter_designs}.
\end{itemize}

CIC  filters  are  well-suited  for  large   rate  changes,  but  are  not  an
optimal solution  for smaller  ones. As an  example of  a low-rate  change CIC
filter,  Figure~\ref{fig:cic_simu:freqz} on  page~\pageref{fig:cic_simu:freqz}
in  Appendix~\ref{sec:app:cic_simu}  shows the  frequency  response  of a  CIC
filter  with a  rate change  of  \num{2}.  Based  on that  observation, it  is
reasonable to  implement the  lower rate change  factors without  CIC filters,
while using  CIC filters  as the  first element  in the  filter chain  for the
higher rate  changes. This allows  taking advantage of  the CIC  filter's high
computational efficiency for large downsampling rates, while still having good
frequency response behavior for the lower rate changes.

Other choices are of course possible. Particularly in the case of $R=625$, one
may  choose  to implement  a  chain  of $125  \rightarrow  5$  instead of  $25
\rightarrow  5 \rightarrow  5$.   The  two implementations  as  they would  be
in  the  final  design  are compared  in  Figure~\ref{fig:dec625_variants}  on
\pageref{fig:dec625_variants} in Appendix~\ref{sec:dec625_variants}. While the
$125 \rightarrow 5$ chain would offer better stopband attunation behavior over
certain frequency  ranges and  therefore improved SNR,  the $25  \rightarrow 5
\rightarrow 5$ chain  offers the advantage that if the  design is ever changed
and only  the higher sampling rates  are implemented (removing the  chains for
$R=1250$ and $R=2500$), it can re-use the elements from the higher chains. The
$25 \rightarrow  5 \rightarrow 5$  chain is  therefore chosen.

With  the above  considerations in  mind, an  overall concept  for the  filter
cascades  can  be  devised; Figure~\ref{fig:fdesign:chain_concept}  shows  the
result. In total,  there are seven  different filters: Two CIC  filters, their
compensators (one of which is used to  downsample as well), a steep and a flat
FIR filter,  and one half-band filter. It  would be possible to  use a second,
less  steep half-band  filter in  the chain  for $R=2500$,  but because  those
filters run  at very  low sampling  frequencies anyway,  using a  steep filter
twice carries no significant penalty over using a second, flatter filter.

\begin{figure}
    \centering
    \tikzsetnextfilename{chainConcepts}
\begin{tikzpicture}[
        node distance=5mm,
        every node/.style={
            font=\ttfamily,
            rounded corners=1mm,
            text centered,
            minimum width=4em,
            minimum height=3.5ex,
            text height=1.5ex,
            text depth=0.25ex,
        },
        5steep/.style={
            draw=sqC,
            fill=sq5,
        },
        5flat/.style={
            draw=sq9,
            fill=sq3,
        },
        2steep/.style={
            draw=dv-7,
            fill=dv-3,
        },
        2flat/.style={
            draw=dv-5,
            fill=dv-1,
        },
        cic25/.style={
            draw=q4,
            fill=q4!80!white,
        },
        cic125/.style={
            draw=q5,
            fill=q5!80!white,
        },
        cfir25/.style={
            draw=q4,
            fill=q4!50!white,
        },
        cfir125/.style={
            draw=q5,
            fill=q5!50!white,
        },
        compressor/.style={
            draw=br0,
            fill=br1,
            minimum width=1em,
        },
    ]
    \node[5steep]                      (5steep5) at (0,0) {5steep};
    \node[compressor,right=of 5steep5] (5compr5a)         {$5\downarrow$};
    \coordinate[left=of 5steep5]   (in5);
    \coordinate[right=of 5compr5a] (out5);
    \draw[-latex] (in5)      -- (5steep5);
    \draw[-latex] (5steep5)  -- (5compr5a);
    \draw[-latex] (5compr5a) -- (out5);

    \node[5flat   ,below=of 5steep5]    (5flat25)   {5flat};
    \node[compressor,right=of 5flat25]  (5compr25a) {$5\downarrow$};
    \node[5steep  ,right=of 5compr25a]  (5steep25)  {5steep};
    \node[compressor,right=of 5steep25] (5compr25b) {$5\downarrow$};
    \coordinate[left=of 5flat25]    (in25);
    \coordinate[right=of 5compr25b] (out25);
    \draw[-latex] (in25)      -- (5flat25);
    \draw[-latex] (5flat25)   -- (5compr25a);
    \draw[-latex] (5compr25a) -- (5steep25);
    \draw[-latex] (5steep25)  -- (5compr25b);
    \draw[-latex] (5compr25b) -- (out25);

    \node[cic25   ,below=of 5flat25]     (25cic125)   {CIC25};
    \node[compressor,right=of 25cic125]  (5compr125a) {$25\downarrow$};
    \node[cfir25  ,right=of 5compr125a]  (1cfir125)   {CFIR25};
    \node[5steep  ,right=of 1cfir125]    (5steep125)  {5steep};
    \node[compressor,right=of 5steep125] (5compr125b) {$5\downarrow$};
    \coordinate[left=of 25cic125]    (in125);
    \coordinate[right=of 5compr125b] (out125);
    \draw[-latex] (in125)      -- (25cic125);
    \draw[-latex] (25cic125)   -- (5compr125a);
    \draw[-latex] (5compr125a) -- (1cfir125);
    \draw[-latex] (1cfir125)   -- (5steep125);
    \draw[-latex] (5steep125)  -- (5compr125b);
    \draw[-latex] (5compr125b) -- (out125);

    \node[cic25   ,below=of 25cic125]    (25cic625)    {CIC25};
    \node[compressor,right=of 25cic625]  (25compr625a) {$25\downarrow$};
    \node[cfir25  ,right=of 25compr625a] (1cfir625)    {CFIR25};
    \node[5flat   ,right=of 1cfir625]    (5flat625)    {5flat};
    \node[compressor,right=of 5flat625]  (5compr625b)  {$5\downarrow$};
    \node[5steep  ,right=of 5compr625b]  (5steep625)   {5steep};
    \node[compressor,right=of 5steep625] (5compr625c)  {$5\downarrow$};
    \coordinate[left=of 25cic625]    (in625);
    \coordinate[right=of 5compr625c] (out625);
    \draw[-latex] (in625)      -- (25cic625);
    \draw[-latex] (25cic625)   -- (25compr625a);
    \draw[-latex] (25compr625a) -- (1cfir625);
    \draw[-latex] (1cfir625)   -- (5flat625);
    \draw[-latex] (5flat625)   -- (5compr625b);
    \draw[-latex] (5compr625b) -- (5steep625);
    \draw[-latex] (5steep625)  -- (5compr625c);
    \draw[-latex] (5compr625c) -- (out625);

    \node[cic125  ,below=of 25cic625]      (125cic1250)    {CIC125};
    \node[compressor,right=of 125cic1250]  (125compr1250a) {$125\downarrow$};
    \node[cfir125 ,right=of 125compr1250a] (5cfir1250)     {CFIR125};
    \node[compressor,right=of 5cfir1250]   (5compr1250b)   {$5\downarrow$};
    \node[2steep  ,right=of 5compr1250b]   (2steep1250)    {2steep};
    \node[compressor,right=of 2steep1250]  (2compr1250c)   {$2\downarrow$};
    \coordinate[left=of 125cic1250]   (in1250);
    \coordinate[right=of 2compr1250c] (out1250);
    \draw[-latex] (in1250)        -- (125cic1250);
    \draw[-latex] (125cic1250)    -- (125compr1250a);
    \draw[-latex] (125compr1250a) -- (5cfir1250);
    \draw[-latex] (5cfir1250)     -- (5compr1250b);
    \draw[-latex] (5compr1250b)   -- (2steep1250);
    \draw[-latex] (2steep1250)    -- (2compr1250c);
    \draw[-latex] (2compr1250c)   -- (out1250);

    \node[cic125 ,below=of 125cic1250]    (125cic2500)    {CIC125};
    \node[compressor,right=of 125cic2500] (125compr2500a) {$125\downarrow$};
    \node[cfir125,right=of 125compr2500a] (5cfir2500)     {CFIR125};
    \node[compressor,right=of 5cfir2500]  (5compr2500b)   {$5\downarrow$};
    \node[2steep ,right=of 5compr2500b]   (2flat2500)     {2steep}; % Not a typo
    \node[compressor,right=of 2flat2500]  (2compr2500c)   {$2\downarrow$};
    \node[2steep ,right=of 2compr2500c]   (2steep2500)    {2steep};
    \node[compressor,right=of 2steep2500] (2compr2500d)   {$2\downarrow$};
    \coordinate[left=of 125cic2500]   (in2500);
    \coordinate[right=of 2compr2500d] (out2500);
    \draw[-latex] (in2500)        -- (125cic2500);
    \draw[-latex] (125cic2500)    -- (125compr2500a);
    \draw[-latex] (125compr2500a) -- (5cfir2500);
    \draw[-latex] (5cfir2500)     -- (5compr2500b);
    \draw[-latex] (5compr2500b)   -- (2flat2500);
    \draw[-latex] (2flat2500)     -- (2compr2500c);
    \draw[-latex] (2compr2500c)   -- (2steep2500);
    \draw[-latex] (2steep2500)    -- (2compr2500d);
    \draw[-latex] (2compr2500d)   -- (out2500);
\end{tikzpicture}

    \caption[Filter Chain Concept]{The concept for the filter chains}
    \label{fig:fdesign:chain_concept}
\end{figure}

%>>>
% ==============================================================================
%
%                  F I L T E R   S P E C I F I C A T I O N S
%
% ==============================================================================
\section{Filter Specifications} % <<< ---------------------------------------- %
\label{sec:fdesign:filter_specifications}
% ---------------------------------------------------------------------------- %

Having determined  the filters needed,  this section outlines  the constraints
imposed upon  them and the specifications  used for the  filter design process
based on those  constraints.  For this purpose, it is  no longer sufficient to
merely consider  the filters  in the mathematical  sense and  the requirements
derived from that; resource usage on  the hardware must be taken into account.
The hardware places two main constraints on the design:
\begin{itemize}\tightlist
    \item
        Number of available LUTs: \num{17600}
    \item
        Number of available DSP slices: \num{80}
\end{itemize}
The number  of LUTs  is relevant  for storage  (filter coefficients),  the CIC
filters\footnote{%
    The CIC compiler block by Xilinx can be configured to utilize LUTs instead
    of DSP  slices for  its computations. The  FIR compiler  can only  use DSP
    slices for its computations.%
},
the  rest of  the processing  system, and  control logic. The  DSP slices  can
therefore  be  reserved  for  the  FIR filters. Because  the  device  has  two
channels, only 40 slices may be used per channel.

To  have a  realistic gauge  for  resource usage  of  the FIR  filters, it  is
necessary to keep  in mind the two factors which  primarily influence resource
usage:
\begin{itemize}\tightlist
    \item
        the frequency  at which the  filter runs (its incoming  sampling rate)
        and
    \item
         the number of coefficients, and therefore, adders and multipliers.
\end{itemize}
Because the final filter in a cascade  is the one which determines the overall
transition  band  (see  Section~\ref{sec:multi_stage_filter_designs}),  it  is
desirable to have  maximally steep output filters  in a cascade. Consequently,
the filter \code{5steep} should be as  sharp as possible. Since that filter is
not just  used as the  final stage  in some cascades,  but also as  the single
filter for the $R=5$ chain, it runs at the highest sampling frequency%
\footnote{%
    It should be noted at this point  that a filter which is configured to run
    at a  high sampling  rate can be  re-used at lower  sampling rates  in the
    Xilinx toolchain. The filter's behavior in that case is correct, even when
    being run at a lower rate than maximally possible.%
}.
\code{5steep}  is therefore  the most  critical  filter in  terms of  resource
usage. The filter \code{5flat} also runs  at the highest sampling frequency in
the $R=25$  chain, but because it  is not the  final filter in that  chain, it
need not be as steep.

While it is possible to estimate the needed resources of a given filter design
based on  its specifications,  reliable figures  are best  obtained by  way of
experiment. The FPGA toolchain might make optimizations which are hard to take
into  account  when  performing  estimates  by hand.   The  results  of  these
measurements are  available in  Appendix~\ref{sec:fir_filter_resouce_usage} on
page~\pageref{sec:fir_filter_resouce_usage}. Based on those  figures, a filter
size  of  around  \num{250}  is  determined  to  be  the  upper  boundary  for
\code{5steep}; a filter  of that size uses about \num{25}  DSP slices, leaving
\num{15} slices for other filters per  channel.  Of these, \code{5flat} is the
most  critical, because  it must  also be  able to  run at  the full  incoming
sampling frequency. \code{2steep}  runs at a much lower sampling  rate and can
therefore be of significant size without  a notable penalty in resource usage.
The final implementation of \code{5steep} is  actually smaller, at a length of
\num{204} coefficients and a DSP slice count of \num{22} per channel, to allow
some flexibility for FPGA design changes.

Based  on  these  findings  and   the  measurements  of  the  STEMlab's  stock
configuration from  Section~\ref{subsec:stl125:ds_default}, it is  possible to
define performance specifications  for the filters without needing  to prod in
the dark,  so to  speak. The following  paragraphs explain  the considerations
which lead to  the final filter specifications. The results  are summarized in
Table~\ref{tab:filter_specs}.

\paragraph{Requirements  for   \code{5steep}:}   Based   on  the   results  of
the  moving  averager   used  in  the  STEMlab's   stock  configuration,  even
a  FIR  filter  with   moderate  performance  characteristics  should  already
offer   significant   improvements. In   order  to   achieve   notable   gains
over  the   default  configuration,   the  following  performance   goals  for
\code{5steep}   are  specified,   according  to   the  pattern   explained  in
Section~\ref{subsec:FIR_filters}:
\begin{itemize}\tightlist
    \item
        Passband ripple: better than \SI{0.25}{\dB}
    \item
        Stopband attenuation: \SI{60}{\dB} or better
    \item
        Transition band width: $0.05 \cdot f_\mathrm{s}/2$ or better
\end{itemize}
The     stopband    attenuation     criterion    is     also    applied     to
all    other     filters. Anything    else    would    be     a    waste    of
resources,   as   shown    in   Figure~\ref{fig:fdesign:cascade:ast_demo}   in
Section~\ref{sec:multi_stage_filter_designs}.

\paragraph{Requirements   for  \code{5flat}:} This   filter   need  not   have
a   drop-off  as   sharp   as   \code{5steep},  as   long   as   the  end   of
its   transition   band   does   not   overlap   with   the   first   spectral
copy  of  \code{5steep} (see  Figure~\ref{fig:fdesign:cascade:good_vs_bad}  in
Section~\ref{sec:multi_stage_filter_designs}).   However, because  it is  in a
cascade with  \code{5steep}, a sharper  requirement on its passband  ripple is
imposed,  in order  not  to worsen  overall passband  ripple  behavior of  the
cascade too much.

\paragraph{Requirements for \code{2steep}:} Due to  the low frequency at which
the half-band filters run, they use  very few resources (\num{1} DSP slice per
channel).   Consequently, only  a  single  filter needs  to  be specified,  as
mentioned in the previous section.

\paragraph{Requirements     for     \code{CIC25}:} The     relevant     design
criteria    for   the    CIC    filter   are    its   stopband    attenuation,
its   decimation    rate,   and   the   cutoff    frequency/desired   passband
width    (see   Figure~\ref{fig:cic:freq_responses:passband:attenuation}    in
Section~\ref{subsubsec:cic:frequency_characteristics}).  The  cutoff frequency
is chosen such that it matches the  frequency band which is of interest at the
end of the filter chain  for $R=125$, i.e. $f_\mathrm{P} = f_\mathrm{s,high}/R
=  0.008$. This means  that  the passband  of  \code{CIC25} and  \code{CFIR25}
combined is too wide by a factor of \num{5} for the $R=625$ chain, but this is
of no concern because it will be cut  off by \code{5flat} as the last stage in
that case. This  allows the re-use of  the CIC and its  compensator across two
chains without changing their design parameters.

\paragraph{Requirements for \code{CIC125}:} The same considerations as for the
other CIC  filter apply. The filter  and its  compensator are specified  for a
rate  change  of  \num{125}  and  \num{5} instead  of  \num{25}  and  \num{1},
respectively, and  the cutoff frequency  is set to  match the filter  chain of
$R=1250$. This makes it twice as wide as it needs to be for $R=2500$, which is
corrected by a second half-band filter.

\paragraph{Requirements  for  compensators:} The  compensators  are  specified
according to the considerations from Section~\ref{subsubsec:cic:compensators},
with the added feature of \code{CIC125} also being used as a decimator.

\paragraph{Summary:} With  the  above   considerations  and  the  experimental
results for  resource usage  from Appendix~\ref{sec:fir_filter_resouce_usage},
it  is  possible  to  formulate  a complete  set  of  specifications  for  the
filters. They  are  compiled   in  Table~\ref{tab:filter_specs}.   Translating
the   specifications    from   Table~\ref{tab:filter_specs}    into   absolute
frequencies   results  in   the  values   from  Table~\ref{tab:tb_widths}. The
frequency    responses    of   all    filters    and    filter   chains    are
listed    in   Appendix~\ref{sec:filter_frequency_responses},    starting   on
page~\pageref{sec:filter_frequency_responses}.

\begin{table}
    \centering
    \caption[Summary of Filter Specifications]{
        The   target   filter   specifications. These  parameters   are  based
        both  on  the  desired  frequency   domain  behavior  of  the  filters
        as   well  as   the  feasibility   of  implementation   in  terms   of
        resource   usage. For  resource   considerations,  the   results  from
        Appendix~\ref{sec:fir_filter_resouce_usage} are used as a guideline.%
    }
    \label{tab:filter_specs}
    \newcommand*\NA{\footnotesize N/A}
    \begin{tabular}{>{\ttfamily}lSSSS}
        \toprule
        \sffamily Filter                                                          &
        {\parbox[t]{26.5mm}{Passband Edge \\ ($\times \pi \si{\radian\per\sample}$)}} &
        {\parbox[t]{26.5mm}{Stopband Edge \\ ($\times \pi \si{\radian\per\sample}$)}} &
        {\parbox[t]{26.5mm}{Passband Ripple \\ (\si{\dB})}}                           &
        {\parbox[t]{26.5mm}{Stopband \\ Attenuation (\si{\dB})}}                     \\
        \midrule
        5steep  & 0.2    & 0.225  & 0.2   & 60 \\
        5flat   & 0.2    & 0.3    & 0.05  & 60 \\
        CIC25   & 0.008  & {\NA}  & {\NA} & 60 \\
        CFIR25  & 0.008  & 0.016  & 0.05  & 60 \\
        CIC125  & 0.0016 & {\NA}  & {\NA} & 60 \\
        CFIR125 & 0.0016 & 0.0024 & 0.05  & 60 \\
        \midrule
        & 
        \multicolumn{2}{l}{{\parbox[t]{53.0mm}{Transition  Band Width \\ ($\times \pi \si{\radian\per\sample}$)}}} &
        &
        {\parbox[t]{26.5mm}{Stopband \\ Attenuation (\si{\dB})}}                     \\
        \midrule
        2steep  & 0.004  & {\NA}   & {\NA}  & 60 \\
        \bottomrule
    \end{tabular}
\end{table}

\begin{table}
    \centering
    \caption[Transition Band Widths]{%
        The  expected   relative  and  absolute  transition   band  widths  of
        the  various   filter  chains,   based  on  the   specifications  from
        Table~\ref{tab:filter_specs}.%
    }
    \label{tab:tb_widths}
    \begin{tabular}{SSS}
        \toprule
        {Chain}                                                                                          &
        {\parbox[t]{40mm}{Relative TB Width \\of Final Filter \\($\times \pi \si{\radian\per\sample}$)}} &
        {\parbox[t]{40mm}{Absolute TB Width \\of Chain \\(\si{\kHz})}}                                   \\
        \midrule
           5 &  0.025 & 1562.5 \\
          25 &  0.025 &  312.5 \\
         125 &  0.025 &   62.5 \\
         625 &  0.025 &   12.5 \\
        1250 &  0.040 &     4.0 \\
        2500 &  0.040 &     2.0 \\
        \bottomrule
    \end{tabular}
\end{table}

%>>>

%^^A vim: foldenable foldcolumn=4 foldmethod=marker foldmarker=<<<,>>>
