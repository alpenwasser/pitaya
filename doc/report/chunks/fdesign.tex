\chapter{Filter Design} % <<< ------------------------------------------------ %
\label{ch:filter_design}
% ---------------------------------------------------------------------------- %

\begin{figure}
    \centering
    \tikzsetnextfilename{tbwDemoAbsolute}
\newcommand*\freqzFileTBWA{images/fdesign/tbw-fs-high.csv}
\newcommand*\freqzFileTBWB{images/fdesign/tbw-fs-low.csv}
\pgfplotstableread[col sep=comma]{\freqzFileTBWA}\freqzTableTBWA
\pgfplotstableread[col sep=comma]{\freqzFileTBWB}\freqzTableTBWB
\begin{tikzpicture}
     \pgfplotsset{every axis/.style={
            height=60mm,
            width=\textwidth,
            grid=none,
            x unit prefix=M,
            change x base=true,
            x SI prefix=mega,
            x unit=\si{Hz},
            xticklabel style={font=\footnotesize},
            y unit=\si{dB},
            ylabel=Magnitude,
            xlabel=Frequency,
            xmin=0,
            xmax=1000000,
            ymax=5,
            ymin=-120,
            y filter/.code={\pgfmathparse{20*log10(\pgfmathresult))}},
        },
    }
    \begin{axis}[
            title={Filter Running at \SI{2}{\MHz}{,} $\text{Transition Band Width} = 0.1 \cdot f_s/2 = \SI{100}{\kHz}$},
            at = {(0,0)},
            xtick={0,0.3e6,0.4e6,1e6},
        ]
        \fill[dv-2] (300000,-120) rectangle (400000,5);
        \addplot[thick,q1,-] table[x=F, y=abs(H)] \freqzTableTBWA;
    \end{axis}
    \begin{axis}[
            title={Filter Running at \SI{1}{\MHz}{,} $\text{Transition Band Width} = 0.1 \cdot f_s/2 = \SI{50}{\kHz}$},
            at = {(0,-70mm)},
            xtick={0,0.15e6,0.2e6,0.5e6,0.8e6,0.85e6,1e6},
        ]
        \fill[dv-2] (150000,-120) rectangle (200000,5);
        \fill[dv-2] (800000,-120) rectangle (850000,5);
        \addplot[thick,q1,-] table[x=F, y=abs(H)] \freqzTableTBWB;
    \end{axis}
\end{tikzpicture}

    \caption
        [Influence of Sampling Frequency on Absolute Transition Band Width]{%
        Influence of sampling frequency on a filter's absolute transition band
        width.  Both plots depict the same filter; in the top plot it is being
        run  at \SI{2}{\MHz},  while in  the bottom  plot it  is being  run at
        \SI{1}{\MHz}. Because the \emph{relative} transition band width is the
        same in both cases ($0.1 \cdot f_s/2$), the \emph{absolute} transition
        band width is twice as large in the top plot as in the bottom plot. Or
        alternatively: The same filter can achieve a steeper passband-stopband
        transition if it  is being run at a lower  sampling frequency. This is
        why cascading  filters is often  a desirable thing  to do in  order to
        save resources.%
    }
    \label{fic:fdesign:tbw_width}
\end{figure}

\begin{figure}
    \centering
    \tikzsetnextfilename{cascadeDemoGood}
\newcommand*\freqzFileCascA{images/fdesign/cascadeDemoHBGoodCascade.csv}
\newcommand*\freqzFileCascB{images/fdesign/cascadeDemoHBGoodStage1.csv}
\newcommand*\freqzFileCascC{images/fdesign/cascadeDemoHBGoodStage2.csv}
\pgfplotstableread[col sep=comma]{\freqzFileCascA}\freqzTableCascA
\pgfplotstableread[col sep=comma]{\freqzFileCascB}\freqzTableCascB
\pgfplotstableread[col sep=comma]{\freqzFileCascC}\freqzTableCascC
\begin{tikzpicture}
     \pgfplotsset{every axis/.style={
            height=60mm,
            width=\textwidth,
            grid=none,
            y filter/.code={\pgfmathparse{20*log10(\pgfmathresult))}},
            x filter/.code={\pgfmathparse{\pgfmathresult / 3.141592654}},
            xlabel=Normalized Frequency,
            ylabel=Magnitude,
            x unit=\times\,\pi\,\si{\radian}/\si{\sample},
            y unit=\si{dB},
            ymax=5,
            ymin=-100,
            xmin=0,
            xmax=1,
            xmajorgrids=true,
            ymajorgrids=true,
            ytick={
                0,
                -40,
                -80
            },
        },
    }
    \begin{axis}[
            title=Cascade,
            at = {(0,0)},
            xtick={
                0,
                0.175,
                0.325,
                0.5,
                1
            },
            xticklabels={
                0,
                0.175,
                0.325,
                $f_{s2}/2$,
                $f_{s1}/2$
            },
        ]
        \addplot[thick,q1,-] table[x=W, y=abs(H)] \freqzTableCascA;
        \addplot[thin,gray,dashed,-] table[x=W, y=abs(H)] \freqzTableCascB;
        \addplot[thin,gray,dashed,-] table[x=W, y=abs(H)] \freqzTableCascC;
    \end{axis}
    \begin{axis}[
            title=Stage 1: Runs at $f_{s1}$,
            at = {(0,-70mm)},
            xtick={
                0,
                0.35,
                0.50,
                0.65,
                1
            },
            xticklabels={
                0,
                0.35,
                0.50,
                0.65,
                1
            },
        ]
        \addplot[thick,q1,-] table[x=W, y=abs(H)] \freqzTableCascB;
    \end{axis}
    \begin{axis}[
            title={Stage 2: Runs at $f_{s2} = f_{s1} \div R_1 = f_{s1} \div 2$},
            at = {(0,-140mm)},
            xtick={
                0,
                0.175,
                0.325,
                0.5,
                0.675,
                0.825,
                1
            },
            xticklabels={
                0,
                0.175,
                0.325,
                0.500,
                0.675,
                0.825,
                1
            },
        ]
        \addplot[thick,q1,-] table[x=W, y=abs(H)] \freqzTableCascC;
    \end{axis}
\end{tikzpicture}

    \caption[Cascade: No Transition Band Overlap]{%
        Cascade of two halfband filters with no overlap in their transition
        band. The cascade fulfills the requirements.%
    }
    \label{fig:fdesign:cascade:good}
\end{figure}
\begin{figure}
    \centering
    \newcommand*\freqzFileCascD{images/fdesign/cascadeDemoHBBadCascade.csv}
\newcommand*\freqzFileCascE{images/fdesign/cascadeDemoHBBadStage1.csv}
\newcommand*\freqzFileCascF{images/fdesign/cascadeDemoHBBadStage2.csv}
\pgfplotstableread[col sep=comma]{\freqzFileCascD}\freqzTableCascD
\pgfplotstableread[col sep=comma]{\freqzFileCascE}\freqzTableCascE
\pgfplotstableread[col sep=comma]{\freqzFileCascF}\freqzTableCascF
\begin{tikzpicture}
     \pgfplotsset{every axis/.style={
            height=60mm,
            width=\textwidth,
            grid=none,
            y filter/.code={\pgfmathparse{20*log10(\pgfmathresult))}},
            x filter/.code={\pgfmathparse{\pgfmathresult / 3.141592654}},
            xlabel=Normalized Frequency,
            ylabel=Magnitude,
            x unit=\times\,\pi\,\si{\radian}/\si{\sample},
            y unit=\si{dB},
            ymax=5,
            ymin=-100,
            xmin=0,
            xmax=1,
            xmajorgrids=true,
            ymajorgrids=true,
            ytick={
                0,
                -40,
                -80
            },
        },
    }
    \begin{axis}[
            title=Cascade,
            at = {(0,0)},
            xtick={
                0,
                0.1,
                0.4,
                0.5,
                1
            },
            xticklabels={
                0,
                0.2,
                0.4,
                $f_{s2}/2$,
                $f_{s1}/2$
            },
        ]
        \addplot[thick,q1,-] table[x=W, y=abs(H)] \freqzTableCascD;
        \addplot[thin,gray,dashed,-] table[x=W, y=abs(H)] \freqzTableCascE;
        \addplot[thin,gray,dashed,-] table[x=W, y=abs(H)] \freqzTableCascF;
    \end{axis}
    \begin{axis}[
            title=Stage 1: Runs at $f_{s1}$,
            at = {(0,-70mm)},
            xtick={
                0,
                0.2,
                0.5,
                0.8,
                1
            },
        ]
        \addplot[thick,q2,-] table[x=W, y=abs(H)] \freqzTableCascE;
    \end{axis}
    \begin{axis}[
            title={Stage 2: Runs at $f_{s2} = f_{s1} \div R_1 = f_{s1} \div 2$},
            at = {(0,-140mm)},
            xtick={
                0,
                0.1,
                0.4,
                0.5,
                0.6,
                0.9,
                1
            },
        ]
        \addplot[thick,q3,-] table[x=W, y=abs(H)] \freqzTableCascF;
    \end{axis}
\end{tikzpicture}

    \caption[Cascade: Transition Band Overlap]{%
        Cascade  of   two  halfband   filters  with  their   transition  bands
        overlapping,  leading  to  a  spike above  the  desired  threshold  of
        \SI{-40}{\dB}  in  the  stop  band. This  configuration  is  therefore
        unsuitable.%
    }
    \label{fig:fdesign:cascade:bad}
\end{figure}

%>>>

%
%\section{Designing a Filter System}
%\label{sec:designing-a-filter-system}
%
%Talking about which type of filter has which properties is all good and well in theory, but
%how does one actually apply this knowledge to a practical problem? This section answers that
%question insofar as it applies to our project.
%
%\begin{itemize}\tightlist
%    \item
%        limited HW resources
%    \item
%        single-stage vs. multi-stage
%    \item
%        TBW issue with multi-stage
%    \item
%        filters at lower frequencies use fewer resources
%    \item
%        halfband filtres
%    \item
%        CIC: compensation filters
%\end{itemize}

%^^A vim: foldenable foldcolumn=4 foldmethod=marker foldmarker=<<<,>>>
