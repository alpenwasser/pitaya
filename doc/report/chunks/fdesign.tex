\chapter{Filter Design} % <<< ------------------------------------------------ %
\label{ch:filter_design}
% ---------------------------------------------------------------------------- %

Some    key    points    underlying     the    theory    of    filters    have
been   treated   in  Chapter~\ref{ch:analog-to-digital_data_aquisition},   and
Chapter~\ref{ch:mission}  defines  the  overall objectives  of  this  project;
implementing a  custom filtering system  on the FPGA among  them. This chapter
will now develop a concrete concept for that filtering system, address some of
the issues encountered when moving from  the theory of filters to the practice
of designing them, and then define the  actual filters which are to be used on
the FPGA. The implementation of those filters  on the FPGA is addressed in the
following chapter, beginning on page~\pageref{ch:fpga}.


\section{Requirements} % <<< ------------------------------------------------- %
\label{sec:requirements}
% ---------------------------------------------------------------------------- %

The overarching  objective is downsampling the  signal coming out of  the ADC.
This  section will  derive  upper  and lower  boundaries  for the  downsampled
frequency range, and then define the specific downsampling ratios to be used.

The total data rate from the ADC is
\sisetup{inter-unit-product = \ensuremath { { } \cdot { } }}
\begin{alignat}{4}
    S                  &= \SI{125}{\mega\sample\per\second}                                  & & \nonumber  \\
    N_\mathrm{ch}      &= 2                                                                  & & \nonumber  \\
    B_\mathrm{ch}      &= \SI{14}{\bit\per\sample}                                           & & \nonumber  \\
    B_\mathrm{ch,pad}  &= \SI{2}{\bit\per\sample}                                            & & \nonumber  \\
    B_\mathrm{ADC}     &= N_\mathrm{ch} \cdot \left(B_\mathrm{ch} + B_\mathrm{ch,pad}\right) \,\,&=&\,\, \SI{32}{\bit\per\sample} \nonumber \\
    R                  &= S \cdot B_\mathrm{ADC}                                             \,\,&=&\,\, \SI{4}{\giga\bit\per\second} \label{eq:adc_data_rate}
\end{alignat}
\begin{conditions}
    S                  & sampling rate                         \\
    N_\mathrm{ch}      & number of channels                    \\
    B_\mathrm{ch}      & channel width                         \\
    B_\mathrm{ch,pad}  & padding per channel                   \\
    B_\mathrm{ADC}     & total width of bit stream out of ADC  \\
    R                  & total data rate out of ADC in bit     \\
\end{conditions}
%\sisetup{inter-unit-product = \,}

The   upper   boundary  for   the   resulting   sampling   rate  is   set   by
the   STEMlab's   network    connection,   which   has   a    data   rate   of
\SI{1000}{\mega\bit\per\second}. A downsampling factor of  at least \num{4} is
therefore required  for real-time data transmission. Because  \num{125} is not
divisible by \num{4}, a factor of  \num{5} is chosen instead. This makes for a
resulting data  rate of  \SI{800}{\mega\bit\per\second}, which should  also be
easily sufficient for protocol overhead.

On the lower end  of the spectrum, the system should still  be able to process
audio  signals. Common  sampling  frequencies for  audio  are  \SI{44.1}{\kHz}
for  audio CDs,  and \SI{48}{\kHz}  for  the audio  component of  audio-visual
applications. Neither  of these  frequencies  fit  nicely into  \SI{125}{\MHz}
(requiring large  prime factor  for the  rate change),  so the  lower boundary
is  specified  as \SI{50}{\kHz},  corresponding  to  a downsampling  ratio  of
\num{2500}.

To cover  additional use cases,  some additional sampling  frequencies between
these   two  boundaries   are  specified. Table~\ref{tab:ratio_decompositions}
contains  the   complete  list   of  downsampling   ratios,  along   with  the
corresponding sampling frequencies.

\begin{table}
    \centering
    \caption[Downsampling Ratios, Decompositions, and Target Frequencies]{
        The  chosen downsampling  ratios, their  prime factor  decompositions,
        the  downsampling ratios  distribet across  stages, and  the resultant
        sampling rates%
    }
    \label{tab:ratio_decompositions}
    \begin{tabular}{S >{$}r<{$} >{$}l<{$} S}
        \toprule
        {R}  & \text{Decomposition}                                                 & \text{Stages}                                 & {$f_s$ (\si{\kHz})} \\
        \midrule
           5 &  5                                                   = 5^1           &   5                                           & 25000               \\
          25 &  5 \cdot 5                                           = 5^2           &   5 \rightarrow 5                             &  5000               \\
         125 &  5 \cdot 5 \cdot 5                                   = 5^3           &  25 \rightarrow 5                             &  1000               \\
         625 &  5 \cdot 5 \cdot 5 \cdot 5 \cdot 5                   = 5^4           &  25 \rightarrow 5 \rightarrow 5               &   200               \\
        1250 &  2 \cdot 5 \cdot 5 \cdot 5 \cdot 5 \cdot 5           = 2^1 \cdot 5^4 & 125 \rightarrow 5 \rightarrow 2               &   100               \\
        2500 &  2 \cdot 2 \cdot 5 \cdot 5 \cdot 5 \cdot 5 \cdot 5   = 2^1 \cdot 5^4 & 125 \rightarrow 5 \rightarrow 2 \rightarrow 2 &    50               \\
        \bottomrule
    \end{tabular}
\end{table}


%>>>


\section{Cascade Concept} % <<< ---------------------------------------------- %
\label{sec:cascade_concept}
% ---------------------------------------------------------------------------- %

Based  on the  downsampling factors  from Table~\ref{tab:downsampling_ratios},
this section  presents the  general concept for  the cascades  which implement
those rate changes.

As  discussed  in  Section~\ref{sec:multi_stage_filter_designs},  implementing
high   downsampling   ratios   in   a    single   stage   is   generally   not
a   sound   design   choice. Consequently,  the   downsampling   ratios   from
Table~\ref{tab:downsampling_ratios} must  be decomposed into  smaller factors,
which can then be distributed along a chain. These factors must fulfill the
following criteria:
\begin{itemize}\tightlist
    \item
        Filters  for  different stages  should  be  re-usable across  multiple
        downsampling ratios  in order  to save resources. Rate  change factors
        which  are  common  to  multiple rate  change  factors  are  therefore
        preferred.
    \item
        The factors for the individual stages  should be large enough to be of
        utility,  but small  enough so  as not  to make  the resulting  filter
        impractically narrow.
\end{itemize}
CIC  filters  are  well-suited  for  large   rate  changes,  but  are  not  an
optimal solution  for smaller  ones. As an  example of  a low-rate  change CIC
filter,  Figure~\ref{fig:cic_simu:freqz} on  page~\pageref{fig:cic_simu:freqz}
in  Appendix~\ref{sec:app:cic_simu}  shows the  frequency  response  of a  CIC
filter with a rate change of \num{2}.

Based on that observation, it is reasonable to implement the lower rate change
factors without CIC  filters, while using CIC filters as  the first element in
the filter  chain for  the higher rate  changes. This allows  taking advantage
of  the CIC  filter's  high computational  efficiency  for large  downsampling
rates,  while still  having good  frequency  response behavior  for the  lower
rate  changes. Table~\ref{tab:ratio_decompositions}  shows  the  prime  factor
decomposition for the overall rate changes,  as well as the factors chosen for
the individual filter chain stages.

Other choices are of course possible. Particularly in the case of $R=625$, one
may  choose  to implement  a  chain  of $125  \rightarrow  5$  instead of  $25
\rightarrow  5 \rightarrow  5$.   The  two implementations  as  they would  be
in  the  final  design  are compared  in  Figure~\ref{fig:dec625_variants}  on
\pageref{fig:dec625_variants} in Appendix~\ref{sec:dec625_variants}. While the
$125 \rightarrow 5$ chain would offer better stopband attunation behavior over
certain frequency  ranges and  therefore improved SNR,  the $25  \rightarrow 5
\rightarrow 5$ chain  offers the advantage that if the  design is ever changed
and only  the higher sampling rates  are implemented (removing the  chains for
$R=1250$ and $R=2500$), it can re-use the elements from the higher chains. The
$25 \rightarrow  5 \rightarrow 5$  chain is  therefore chosen.
%>>>

\section{Filter Specifications} % <<< ---------------------------------------- %
\label{sec:fdesign:filter_specifications}
% ---------------------------------------------------------------------------- %

Continuing  from Table~\ref{tab:ratio_decompositions},  this section  presents
the  concept for  the cascades  which  are used  in our  design. Specifically,
requirements and constraints for the  filters in those cascades are specified.
For this purpose, it is no longer sufficient to merely consider the filters in
the mathematical  sense; resource  usage on  the hardware  must be  taken into
account.

The hardware places two main contrainst on the design:
\begin{itemize}\tightlist
    \item
        Number of available LUTs: \num{17600}
    \item
        Number of available DSP slices: \num{80}
\end{itemize}
The number  of LUTs  is relevant  for storage  (filter coefficients),  the CIC
filters\footnote{%
    The CIC compiler block by Xilinx can be configured to utilize LUTs instead
    of DSP  slices for  its computations. The  FIR compiler  can only  use DSP
    slices for its computations.%
},
the  rest of  the processing  system, and  control logic. The  DSP slices  can
therefore  be  reserved  for  the  FIR filters. Because  the  device  has  two
channels, only 40 slices may be used per channel.

In order to have  a realistic gauge for resource usage of  the FIR filters, it
is  necessary to  keep  in  mind the  two  factors  which primarily  influence
resource usage:
\begin{itemize}\tightlist
    \item
        the sampling frequency at which the filter runs (its incoming sampling
        rate),
    \item
         and the number of coefficients, and therefore, adders and multipliers.
\end{itemize}
It is  therefore important to  know which filters in  the design run  at which
sampling  rates. For this  to  be possible,  a concept  of  the six  different
filter  chains   is  needed. Based   on  Table~\ref{tab:ratio_decompositions},
Figure~\ref{fig:fdesign:chain_concept} depicts  that concept. It  includes all
the required  filters of the  design, including  the compensators for  the CIC
filters. Note that  one of  the compensators  is itself  used as  a decimation
filter in the case of the $R=1250$ and $R=2500$ chains.

\begin{figure}
    \centering
    \tikzsetnextfilename{chainConcepts}
\begin{tikzpicture}[
        node distance=5mm,
        every node/.style={
            font=\ttfamily,
            rounded corners=1mm,
            text centered,
            minimum width=4em,
            minimum height=3.5ex,
            text height=1.5ex,
            text depth=0.25ex,
        },
        5steep/.style={
            draw=sqC,
            fill=sq5,
        },
        5flat/.style={
            draw=sq9,
            fill=sq3,
        },
        2steep/.style={
            draw=dv-7,
            fill=dv-3,
        },
        2flat/.style={
            draw=dv-5,
            fill=dv-1,
        },
        cic25/.style={
            draw=q4,
            fill=q4!80!white,
        },
        cic125/.style={
            draw=q5,
            fill=q5!80!white,
        },
        cfir25/.style={
            draw=q4,
            fill=q4!50!white,
        },
        cfir125/.style={
            draw=q5,
            fill=q5!50!white,
        },
        compressor/.style={
            draw=br0,
            fill=br1,
            minimum width=1em,
        },
    ]
    \node[5steep]                      (5steep5) at (0,0) {5steep};
    \node[compressor,right=of 5steep5] (5compr5a)         {$5\downarrow$};
    \coordinate[left=of 5steep5]   (in5);
    \coordinate[right=of 5compr5a] (out5);
    \draw[-latex] (in5)      -- (5steep5);
    \draw[-latex] (5steep5)  -- (5compr5a);
    \draw[-latex] (5compr5a) -- (out5);

    \node[5flat   ,below=of 5steep5]    (5flat25)   {5flat};
    \node[compressor,right=of 5flat25]  (5compr25a) {$5\downarrow$};
    \node[5steep  ,right=of 5compr25a]  (5steep25)  {5steep};
    \node[compressor,right=of 5steep25] (5compr25b) {$5\downarrow$};
    \coordinate[left=of 5flat25]    (in25);
    \coordinate[right=of 5compr25b] (out25);
    \draw[-latex] (in25)      -- (5flat25);
    \draw[-latex] (5flat25)   -- (5compr25a);
    \draw[-latex] (5compr25a) -- (5steep25);
    \draw[-latex] (5steep25)  -- (5compr25b);
    \draw[-latex] (5compr25b) -- (out25);

    \node[cic25   ,below=of 5flat25]     (25cic125)   {CIC25};
    \node[compressor,right=of 25cic125]  (5compr125a) {$25\downarrow$};
    \node[cfir25  ,right=of 5compr125a]  (1cfir125)   {CFIR25};
    \node[5steep  ,right=of 1cfir125]    (5steep125)  {5steep};
    \node[compressor,right=of 5steep125] (5compr125b) {$5\downarrow$};
    \coordinate[left=of 25cic125]    (in125);
    \coordinate[right=of 5compr125b] (out125);
    \draw[-latex] (in125)      -- (25cic125);
    \draw[-latex] (25cic125)   -- (5compr125a);
    \draw[-latex] (5compr125a) -- (1cfir125);
    \draw[-latex] (1cfir125)   -- (5steep125);
    \draw[-latex] (5steep125)  -- (5compr125b);
    \draw[-latex] (5compr125b) -- (out125);

    \node[cic25   ,below=of 25cic125]    (25cic625)    {CIC25};
    \node[compressor,right=of 25cic625]  (25compr625a) {$25\downarrow$};
    \node[cfir25  ,right=of 25compr625a] (1cfir625)    {CFIR25};
    \node[5flat   ,right=of 1cfir625]    (5flat625)    {5flat};
    \node[compressor,right=of 5flat625]  (5compr625b)  {$5\downarrow$};
    \node[5steep  ,right=of 5compr625b]  (5steep625)   {5steep};
    \node[compressor,right=of 5steep625] (5compr625c)  {$5\downarrow$};
    \coordinate[left=of 25cic625]    (in625);
    \coordinate[right=of 5compr625c] (out625);
    \draw[-latex] (in625)      -- (25cic625);
    \draw[-latex] (25cic625)   -- (25compr625a);
    \draw[-latex] (25compr625a) -- (1cfir625);
    \draw[-latex] (1cfir625)   -- (5flat625);
    \draw[-latex] (5flat625)   -- (5compr625b);
    \draw[-latex] (5compr625b) -- (5steep625);
    \draw[-latex] (5steep625)  -- (5compr625c);
    \draw[-latex] (5compr625c) -- (out625);

    \node[cic125  ,below=of 25cic625]      (125cic1250)    {CIC125};
    \node[compressor,right=of 125cic1250]  (125compr1250a) {$125\downarrow$};
    \node[cfir125 ,right=of 125compr1250a] (5cfir1250)     {CFIR125};
    \node[compressor,right=of 5cfir1250]   (5compr1250b)   {$5\downarrow$};
    \node[2steep  ,right=of 5compr1250b]   (2steep1250)    {2steep};
    \node[compressor,right=of 2steep1250]  (2compr1250c)   {$2\downarrow$};
    \coordinate[left=of 125cic1250]   (in1250);
    \coordinate[right=of 2compr1250c] (out1250);
    \draw[-latex] (in1250)        -- (125cic1250);
    \draw[-latex] (125cic1250)    -- (125compr1250a);
    \draw[-latex] (125compr1250a) -- (5cfir1250);
    \draw[-latex] (5cfir1250)     -- (5compr1250b);
    \draw[-latex] (5compr1250b)   -- (2steep1250);
    \draw[-latex] (2steep1250)    -- (2compr1250c);
    \draw[-latex] (2compr1250c)   -- (out1250);

    \node[cic125 ,below=of 125cic1250]    (125cic2500)    {CIC125};
    \node[compressor,right=of 125cic2500] (125compr2500a) {$125\downarrow$};
    \node[cfir125,right=of 125compr2500a] (5cfir2500)     {CFIR125};
    \node[compressor,right=of 5cfir2500]  (5compr2500b)   {$5\downarrow$};
    \node[2steep ,right=of 5compr2500b]   (2flat2500)     {2steep}; % Not a typo
    \node[compressor,right=of 2flat2500]  (2compr2500c)   {$2\downarrow$};
    \node[2steep ,right=of 2compr2500c]   (2steep2500)    {2steep};
    \node[compressor,right=of 2steep2500] (2compr2500d)   {$2\downarrow$};
    \coordinate[left=of 125cic2500]   (in2500);
    \coordinate[right=of 2compr2500d] (out2500);
    \draw[-latex] (in2500)        -- (125cic2500);
    \draw[-latex] (125cic2500)    -- (125compr2500a);
    \draw[-latex] (125compr2500a) -- (5cfir2500);
    \draw[-latex] (5cfir2500)     -- (5compr2500b);
    \draw[-latex] (5compr2500b)   -- (2flat2500);
    \draw[-latex] (2flat2500)     -- (2compr2500c);
    \draw[-latex] (2compr2500c)   -- (2steep2500);
    \draw[-latex] (2steep2500)    -- (2compr2500d);
    \draw[-latex] (2compr2500d)   -- (out2500);
\end{tikzpicture}

    \caption[Filter Chain Concept]{The concept for the filter chains}
    \label{fig:fdesign:chain_concept}
\end{figure}

Because the final filter in a cascade  is the one which determines the overall
transition  band  (see  Section~\ref{sec:multi_stage_filter_designs}),  it  is
desirable to have maximally steep  output filters in a cascade. Therefore, the
filter \code{5steep} should be as sharp  as possible. Since that filter is not
just used as the  final stage in some cascades, but also  as the single filter
for the $R=5$ chain, it runs  at the highest sampling frequency%
\footnote{%
    It should be noted at this point  that a filter which is configured to run
    at a high  sampling rate can be re-used  as a lower stage in  a cascade in
    the Xilinx toolchain. The filter's behavior  in that case is correct, even
    when being run at a lower rate than maximally possible.%
}.
\code{5steep}  is therefore  the most  critical  filter in  terms of  resource
usage. The filter \code{5flat} also runs  at the highest sampling frequency in
the $R=25$  chain, but because it  is not the  final filter in that  chain, it
need not be as steep.

While  it is  possible to  estimate  the needed  resources of  a given  filter
design,  reliable  figures  are  best   obtained  by  way  of  experiment. The
FPGA  toolchain  might  make  optimizations   which  are  hard  to  take  into
account   when  performing   estimates  by   hand.   The   results  of   these
measurements are  available in  Appendix~\ref{sec:fir_filter_resouce_usage} on
page~\pageref{sec:fir_filter_resouce_usage}. Based on those  numbers, a filter
size  of around  \num{200}  is  determined to  be  a  reasonable boundary  for
\code{5steep}; a filter of that size uses about
TODO
DSP slices. This leaves
TODO
number  of  slices   for  the  remaining  filters   (per  channel). Of  these,
\code{5flat} is the most critical, because it  must also be able to run at the
full incoming sampling frequency. \code{2steep}  runs at a much lower sampling
rate  and can therefore be of significant size without a notable penalty in
resource usage.

Based  on  these  findings  and   the  measurements  of  the  STEMlab's  stock
configuration from  Section~\ref{subsec:stl125:ds_default}, it is  possible to
define performance specifications  for the filters without needing  to prod in
the dark,  so to  speak. The following  paragraphs explain  the considerations
which lead to  the final filter specifications. The results  are summarized in
Table~\ref{tab:filter_specs}.

\paragraph{Requirements  for   \code{5steep}:}   Based   on  the   results  of
the  moving  averager   used  in  the  STEMlab's   stock  configuration,  even
a  FIR  filter  with   moderate  performance  characteristics  should  already
offer   significant   improvements. In   order  to   achieve   notable   gains
over  the   default  configuration,   the  following  performance   goals  for
\code{5steep}   are  specified,   according  to   the  pattern   explained  in
Section~\ref{subsec:FIR_filters}:
\begin{itemize}\tightlist
    \item
        Passband ripple: better than \SI{0.25}{\dB}
    \item
        Stopband attenuation: \SI{60}{\dB} or better
    \item
        Transition band width: $0.05 \cdot f_\mathrm{s}/2$ or better
\end{itemize}
The     stopband    attenuation     criterion    is     also    applied     to
all    other     filters. Anything    else    would    be     a    waste    of
resources,   as   shown    in   Figure~\ref{fig:fdesign:cascade:ast_demo}   in
Section~\ref{sec:multi_stage_filter_designs}.

\paragraph{Requirements   for  \code{5flat}:} This   filter   need  not   have
a   drop-off  as   sharp   as   \code{5steep},  as   long   as   the  end   of
its   transition   band   does   not   overlap   with   the   first   spectral
copy  of  \code{5steep} (see  Figure~\ref{fig:fdesign:cascade:good_vs_bad}  in
Section~\ref{sec:multi_stage_filter_designs}).   However, because  it is  in a
cascade with  \code{5steep}, a sharper  requirement on its passband  ripple is
imposed,  in order  not  to worsen  overall passband  ripple  behavior of  the
cascade too much.

\paragraph{Requirements    for    \code{2steep}:} In   theory,    the    first
decimator-by-two in  the chain  for $R=2500$  could be  designed with  a wider
transition band than the one which is  being used as the last stage (hence the
designator \code{5steep}). However, because  these filters do not  have to run
at high sampling rates, the amount of DSP slices they use is very low (\num{1}
DSP slice per channel),  so the same filter can be  re-used without a resource
penalty.

TODO: halfband

\paragraph{Requirements     for     \code{CIC25}:} The     relevant     design
criteria    for   the    CIC    filter   are    its   stopband    attenuation,
its   decimation    rate,   and   the   cutoff    frequency/desired   passband
width    (see   Figure~\ref{fig:cic:freq_responses:passband:attenuation}    in
Section~\ref{subsubsec:cic:frequency_characteristics}).  The  cutoff frequency
is chosen such that it matches the  frequency band which is of interest at the
end of the filter chain  for $R=125$, i.e. $f_\mathrm{P} = f_\mathrm{s,high}/R
=  0.008$. This means  that  the passband  of  \code{CIC25} and  \code{CFIR25}
combined is too wide by a factor of \num{5} for the $R=625$ chain, but this is
of no concern because it will be cut  off by \code{5flat} as the last stage in
that  case. This allows  the  re-use of  the two  filters  across both  chains
without changing their design parameters.

\paragraph{Requirements for \code{CIC125}:} The same considerations as for the
other CIC  filter apply. The filter  and its  compensator are specified  for a
rate  change  of  \num{125}  and  \num{5} instead  of  \num{25}  and  \num{1},
respectively, and  the cutoff frequency  is set to  match the filter  chain of
$R=1250$. This makes it twice as wide as it needs to be for $R=2500$, which is
corrected by a second halfband filter.

\paragraph{Requirements     for     compensators:} The    compensators     are
specified     according    to     the    considerations     laid    out     in
Section~\ref{subsubsec:cic:compensators},   with   the    added   feature   of
\code{CIC125} also being used as a decimator.

\paragraph{Summary:} With  the  above   considerations  and  the  experimental
results for  resource usage  from Appendix~\ref{sec:fir_filter_resouce_usage},
it  is  possible  to  formulate  a complete  set  of  specifications  for  the
filters. They  are  compiled   in  Table~\ref{tab:filter_specs}.   Translating
the   specifications    from   Table~\ref{tab:filter_specs}    into   absolute
frequencies  results  in   the  values  from  Table~\label{tab:tb_widths}. The
frequency    responses    of   all    filters    and    filter   chains    are
depicted   in   Appendix~\ref{sec:filter_frequency_responses},   starting   on
page~\pageref{sec:filter_frequency_responses}.

\begin{table}
    \centering
    \caption[Summary of Filter Specifications]{
        The   target   filter   specifications. These  parameters   are  based
        both  on  the  desired  frequency   domain  behavior  of  the  filters
        as   well  as   the  feasibility   of  implementation   in  terms   of
        resource   usage. For  resource   considerations,  the   results  from
        Appendix~\ref{sec:fir_filter_resouce_usage} are used as a guideline.%
    }
    \label{tab:filter_specs}
    \newcommand*\NA{\footnotesize N/A}
    \begin{tabular}{>{\ttfamily}lSSSS}
        \toprule
        \sffamily Filter                                                          &
        {\parbox[t]{26.5mm}{Passband Edge \\ ($\times \pi \si{\radian\per\sample}$)}} &
        {\parbox[t]{26.5mm}{Stopband Edge \\ ($\times \pi \si{\radian\per\sample}$)}} &
        {\parbox[t]{26.5mm}{Passband Ripple \\ (\si{\dB})}}                           &
        {\parbox[t]{26.5mm}{Stopband \\ Attenuation (\si{\dB})}}                     \\
        \midrule
        5steep  & 0.2    & 0.225  & 0.2   & 60 \\
        5flat   & 0.2    & 0.3    & 0.05  & 60 \\
        CIC25   & 0.008  & {\NA}  & {\NA} & 60 \\
        CFIR25  & 0.008  & 0.016  & 0.05  & 60 \\
        CIC125  & 0.0016 & {\NA}  & {\NA} & 60 \\
        CFIR125 & 0.0016 & 0.0024 & 0.05  & 60 \\
        \midrule
        & 
        \multicolumn{2}{l}{{\parbox[t]{53.0mm}{Transition  Band Width \\ ($\times \pi \si{\radian\per\sample}$)}}} &
        &
        {\parbox[t]{26.5mm}{Stopband \\ Attenuation (\si{\dB})}}                     \\
        \midrule
        2steep  & 0.004  & {\NA}   & {\NA}  & 60 \\
        \bottomrule
    \end{tabular}
\end{table}

\begin{table}
    \centering
    \caption[Transition Band Widths]{%
        The  expected   relative  and  absolute  transition   band  widths  of
        the  various   filter  chains,   based  on  the   specifications  from
        Table~\ref{tab:filter_specs}.%
    }
    \label{tab:tb_widths}
    \begin{tabular}{SSS}
        \toprule
        {Chain}                                                                                          &
        {\parbox[t]{40mm}{Relative TB Width \\of Final Filter \\($\times \pi \si{\radian\per\sample}$)}} &
        {\parbox[t]{40mm}{Absolute TB Width \\of Chain \\(\si{\kHz})}}                                   \\
        \midrule
           5 &  0.025 & 1562.5 \\
          25 &  0.025 &  312.5 \\
         125 &  0.025 &   62.5 \\
         625 &  0.025 &   12.5 \\
        1250 &  0.040 &     4.0 \\
        2500 &  0.040 &     2.0 \\
        \bottomrule
    \end{tabular}
\end{table}

%25 -> 1 -> 5 -> 5 instead of 25 -> cfir5 -> 5: compensation filter is needed anyway
%but can be kept smaller. 5flat is needed anyway and runs at higher sampling frequency
%than cfir25.

%>>>

%>>>
%^^A vim: foldenable foldcolumn=4 foldmethod=marker foldmarker=<<<,>>>
