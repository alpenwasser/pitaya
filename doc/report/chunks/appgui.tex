% ==============================================================================
%
%                           O S C I L L O S C O P E
%
% ==============================================================================
\chapter{Oscilloscope} % ----------------------------------------------------- %
\label{ch:app:gui}
% ---------------------------------------------------------------------------- %

% ==============================================================================
%
%                             W E B S O C K E T S
%
% ==============================================================================
\section{WebSockets} % <<< --------------------------------------------------- %
\label{sec:app:gui:websockets}
% ---------------------------------------------------------------------------- %

WebSockets' final RFC 6455\cite{rfc:6455} was released in December 2011 and is
thus still quite young. It is meant to  compensate the lack of raw UDP and TCP
sockets in JavaScript; while those  would offer maximum flexibility, they also
pose  a  significant  security  risk,  and  are  therefore  not  available  in
JavaScript.  The  WebSockets protocol is  located in the Application  Layer of
the OSI model\footnote{%
    For  those not  familiar with  the OSI  model, Wikipedia  provides a  good
    overview in~\cite{wiki:osi}.%
}.
Instead  of directly  opening  a  raw WebSocket,  the  handshake  is done  via
HTTP(S). This brings  the benefit of  communicating through the same  ports as
the browser  (\num{80} or  \num{443}) which enables  the protocol  to function
through most  firewalls. Furthermore it greatly simplifies  the implementation
of handshakes for the programmer.

The client sends an upgrade request to the server which then opens a WebSocket
connection.   This allows  for  a  very convenient  way  to  use TCP  Sockets
without any entirely new  standards.  Section~1.5, \emph{Design Philosophy} in
RFC~6455~\cite{rfc:6455} explains it well:
\begin{quote}
    Basically it is intended to be as close to just exposing raw TCP to script
    as possible given the constraints of the Web.

    The  only exception  is that  WebSockets adds  framing to  make it  packet
    rather  than stream  based and  to differentiate  between binary  and text
    data.  This differentiation is  very useful for this project. Instructions
    to the  server are issued  via the text channel  whilst data is  sent back
    through the binary channel, allowing  for very convenient interfacing with
    close to no effort.
\end{quote}
In summary: WebSockets  are close-to-raw  TCP sockets  whose handle  is shared
through HTTP(S).

JavaScript provides an interface that  offers convenient sending and receiving
of  large amounts  of data.   As nearly  anything in  JavaScript this  is done
using callbacks. There  are callbacks  which handle connections,  messages and
errors. The code snippet in  Listing~\ref{lst:gui:jsws} gives some insight how
WebSockets in JavaScript are used. For  more detailed information, the reader
is referred to the Mozilla documentation~\cite{moz:ws}.

\begin{tcolorbox}[
        skin=alpenlisting,
        title={
            \refstepcounter{listing}
            \textbf{Listing \thelisting:} Using WebSockets in JavaScript
            \label{lst:gui:jsws}
            \addcontentsline{lol}{listing}{\protect\numberline{\thelisting}Using Websockets in JavaScript}
        }
    ]
    \inputminted[
        linenos,
        numbersep=4pt,
        style=solarizedlight,
    ]{javascript}{./code/websockets.js}
\end{tcolorbox}
%>>>
% ==============================================================================
%
%                             S T A T E   T R E E
%
% ==============================================================================
\section{State Tree of Oscilloscope} % <<< ----------------------------------- %
\label{sec:app:gui:state_tree_of_scope}
% ---------------------------------------------------------------------------- %
\begin{tcolorbox}[
        skin=alpenlisting,
        title={
            \refstepcounter{listing}
            \textbf{Listing \thelisting:} The state tree of the scope application
            \label{lst:gui:app_structure}
            \addcontentsline{lol}{listing}{\protect\numberline{\thelisting}Scope State Tree}
        },
        breakable,
        title after break={\textbf{Listing \thelisting\ (cont.):} The state tree of the scope application},
    ]
    \inputminted[
        linenos,
        numbersep=4pt,
        style=solarizedlight,
    ]{javascript}{./code/statetree.js}
\end{tcolorbox}
%>>>
% ==============================================================================
%
%                             M I T H R I L . J S
%
% ==============================================================================
\section{mithril.js} % <<< --------------------------------------------------- %
\label{sec:app:gui:mithril}
% ---------------------------------------------------------------------------- %

The   official  mithril   webpage  describes   mithril.js  in   the  following
way: ``Mithril  is  a modern  client-side  JavaScript  framework for  building
Single Page Applications. It's  small (< 8kb gzip), fast  and provides routing
and XHR utilities out of the box.''~\cite{mithril:home}

Mithril, like a  lot of other frameworks such as  React, Angular.js or Vue.js,
uses  a virtual  DOM. This means  that it  does not  modify the  DOM which  is
outlined by the  browser, but rather maintains its own  DOM. When a new render
call is issued, the  virtual DOM calculates all the deltas  that stem from new
content and applies them to the  real DOM. This allows mithril.js to calculate
and recalculate the DOM based on  a descriptive model.  The developer does not
have to manually modify an object's state but rather has to describe it.

A redraw generally happens when an event is triggered by any input element but
can also be  issued manually.  A virtual DOM consists  of many vnodes (virtual
nodes) and  can be  mounted on  any actual node  of the  browser's DOM  as the
example in Listing~\ref{lst:gui:mithrilmount} shows.

\begin{tcolorbox}[
        skin=alpenlisting,
        title={
            \refstepcounter{listing}
            \textbf{Listing \thelisting:} Basic creation and usage of mithril components in JavaScript
            \label{lst:gui:mithrilmount}
            \addcontentsline{lol}{listing}{\protect\numberline{\thelisting}Basic Usage of \code{mithril} Components}
        }
    ]
    \inputminted[
        linenos,
        numbersep=4pt,
        style=solarizedlight,
    ]{javascript}{./code/mithrilmount.js}
\end{tcolorbox}

A component can be mounted on any DOM  node and becomes a vnode in the virtual
DOM. The developer can create new components by simply creating an object that
holds at least a \code{view()} function that instantiates new vnodes.  The new
component  can then  be instantiated  via the  \code{m()} or  \code{m.mount()}
command.  As this section  should only give a base overview  on mithril and is
not meant to be a manual,  further information on mithril's features and usage
can be obtained on it's webpage~\cite{mithril:home}.
%>>>
% ==============================================================================
%
%                                 W e b G L
%
% ==============================================================================
\section{WebGL} % <<< -------------------------------------------------------- %
\label{sec:app:gui:webgl}
% ---------------------------------------------------------------------------- %

An  application uses  the \emph{canvas}  DOM element  which provides  a direct
interface to WebGL. The user can render  vertices to the canvas and even apply
shaders  or,  in  the  case  of our  scope  application,  simple  2D  geometry
calls. These are  sufficient for our  purposes since the scope  basically only
requires the drawing of lines.

Via  the canvas  one  can retrieve  a  2D rendering  context  on which  simple
geometry can  be drawn.   In JavaScript  this can  be done  using the  code in
Listing~\ref{lst:js2dcontext} which shows  how a single red line  can be drawn
on the canvas.

\begin{tcolorbox}[
        skin=alpenlisting,
        title={
            \refstepcounter{listing}
            \textbf{Listing \thelisting:} Getting a 2D Rendering Context from a Canvas and Drawing on it in JavaScript
            \label{lst:js2dcontext}
            \addcontentsline{lol}{listing}{\protect\numberline{\thelisting}Drawing on Canvas in JavaScript}
        }
    ]
    \inputminted[
        linenos,
        numbersep=4pt,
        style=solarizedlight,
    ]{javascript}{./code/2dcontext.js}
\end{tcolorbox}

There  is  also   the  possibility  to  draw  rectangles,   circles  and  much
more. All  of those  elements  can  be styled  easily  via  properties of  the
context  environment. All  the  functionality  is documented  on  the  Mozilla
Network~\cite{moz:2dcontext}.

After having  acquired the rendering  context, something  can be drawn  on the
canvas once.   For the  creation of  a moving  image, those  draws have  to be
re-issued over and  over again. There are various  possibilities in JavaScript
to accomplish this, but only one is actually high-performance and recommended.

Instead of simply drawing to the canvas over and over again, it would be ideal
to only  do that  before a  new frame is  pulled from  the framebuffer  by the
display. JavaScript provides a interface to register a callback that is called
before  a  new frame  is  released. This  callback  will  be called  with  the
same  frequency  as  the  display  refresh rate,  which  nowadays  usually  is
\SI{60}{\hertz}.  To make sure that a callback will always be executed, it has
to  be registered  again  after a  callback has  been  issued. The example  in
Listing~\ref{lst:gui:glcallback} shows  how this is done.   This callback will
not affect the  rest of the DOM. This allows JavaScript  to handle the redraws
of the DOM  with high speed while  the callback will render a  fluent graph of
the data onto just one of the DOM elements.

\begin{tcolorbox}[
        skin=alpenlisting,
        title={
            \refstepcounter{listing}
            \textbf{Listing \thelisting:} Usage of the requestAnimationFrame callback in JavaScript
            \label{lst:gui:glcallback}
            \addcontentsline{lol}{listing}{\protect\numberline{\thelisting}Usage of \code{requestAnimationFrame Callback}}
        }
    ]
    \inputminted[
        linenos,
        numbersep=4pt,
        style=solarizedlight,
    ]{javascript}{./code/glcallback.js}
\end{tcolorbox}

%>>>
% ==============================================================================
%
%                           W I N D O W   T A B L E
%
% ==============================================================================
\section{FFT Windowing Parameters} % <<< -------------------------------------- %
\label{sec:app:gui:fft_params}
% ---------------------------------------------------------------------------- %

\begin{centering}
    \tabcaption[FFT Windowing Parameters]{%
        FFT windowing parameters, taken from~\cite{gui:meyer}%
    }
    \label{tab:fft_window_params}
    \begin{tabular}{l>{$}c<{$}ScS}
        \toprule
         \parbox[c]{17mm}{Window}                                       &
        {\parbox[c]{26mm}{Scaling Factor for Quasi-Periodical Signals}} &
        {\parbox[c]{24mm}{Attenuation of Largest Side Lobe (\si{\dB})}} &
        {\parbox[c]{22mm}{Number of Lines per Bundle}                 } &
        {\parbox[c]{22mm}{Maximum Error in \mbox{Amplitude} (\si{\dB})}      } \\
        \midrule
        Rectangle     & 1        & 13 & 1 -- 2 & -3.8 \\
        Hanning       & 1/0.5000 & 31 & 3 -- 4 & -1.5 \\
        Hamming       & 1/0.5400 & 41 & 3 -- 4 & -1.6 \\
        Blackman      & 1/0.4200 & 58 & 5 -- 6 & -1.1 \\
        Bartlett      & 1/0.5000 & 26 & 3 -- 4 & -1.9 \\
        Kaiser-Bessel & 1/0.4021 & 67 & 7 -- 8 & -1.0 \\
        Flat-Top      & 1/0.2155 & 67 & 9 -- 10 & 0    \\
        \bottomrule
    \end{tabular}
\end{centering}

%>>>
%^^A vim: foldenable foldcolumn=4 foldmethod=marker foldmarker=<<<,>>>
