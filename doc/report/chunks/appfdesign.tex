% ==============================================================================
%
%                          F I L T E R   D E S I G N
%
% ==============================================================================
\chapter{Filter Design} % ---------------------------------------------------- %
\label{ch:app:fdesign}
% ---------------------------------------------------------------------------- %

This  chapter contains  some additional  information about  the filter  design
process.

% ==============================================================================
%
%                        D E C 6 2 5   V A R I A N T S
%
% ==============================================================================
\section{Decimation of 625: Variants} % <<< ---------------------------------- %
\label{sec:dec625_variants}
% ---------------------------------------------------------------------------- %

\begin{figure}
    \centering
    \tikzsetnextfilename{dec625VariantsFreqResponse}
\newcommand*\freqzFileDECA{images/fdesign/dec625--125-5.csv}
\newcommand*\freqzFileDECB{images/fdesign/dec625--25-5-5.csv}
\pgfplotstableread[col sep=comma]{\freqzFileDECA}\freqzTableDECA
\pgfplotstableread[col sep=comma]{\freqzFileDECB}\freqzTableDECB
\begin{tikzpicture}[
    trim axis left,
    trim axis right,
]
     \pgfplotsset{every axis/.style={
            height=45mm,
            width=\textwidth,
            grid=none,
            x unit=\times\,\pi\,\si{\radian}/\si{\sample},
            ylabel=Magnitude,
            xlabel=Normalized Frequency,
            xmin=0,
            xmax=3e-3,
            %xmin=0,
            %xmax=1000000,
            %ymax=5,
            %ymin=-120,
            legend columns=2,
            every axis legend/.append style={
                at={(0.5,-0.30)},  % if attached to bottom plot
                anchor=north,
                cells={anchor=west},
            },
        },
    }
    \begin{axis}[
            title={Decimation of $R=625$: Variants},
            at = {(0,0)},
            y unit=\si{dB},
            ymax=5,
            ymin=-180,
            y filter/.code={\pgfmathparse{20*log10(\pgfmathresult))}},
            x filter/.code={\pgfmathparse{\pgfmathresult / 3.141592654}},
        ]
        \addplot[thick,q1,-] table[x=W, y=abs(H)] \freqzTableDECA;
        \addplot[thick,q5,-] table[x=W, y=abs(H)] \freqzTableDECB;
    \end{axis}
    \begin{axis}[
            at = {(0,-62mm)},
            x filter/.code={\pgfmathparse{\pgfmathresult / 3.141592654}},
            %xtick={0,0.3e6,0.4e6,1e6},
        ]
        \addplot[thick,q1,-] table[x=W, y=abs(H)] \freqzTableDECA;
        \addplot[thick,q5,-] table[x=W, y=abs(H)] \freqzTableDECB;
    \end{axis}
    \begin{axis}[
            at = {(0,-124mm)},
            ylabel=Phase,
            y unit=\si{\degree},
            ymin=-2800,
            ymax=0,
            y filter/.code={\pgfmathparse{180*\pgfmathresult/3.141592654}},
            unbounded coords=discard,
            point meta=explicit,
            filter point/.code={%
                \pgfmathparse{\pgfkeysvalueof{/data point/meta}==0}
                \ifpgfmathfloatcomparison
                    \pgfkeyssetvalue{/data point/y}{nan}%
                \fi
            }
        ]
        \addplot[thick,-,q1,] table[x=W,y=angle(H),meta=abs(H)] \freqzTableDECA;
        \addplot[thick,-,q5] table[x=W,y=angle(H),meta=abs(H)] \freqzTableDECB;
        \legend{%
            $125 \rightarrow \text{CFIR} \rightarrow 5$,
            $ 25 \rightarrow \text{CFIR} \rightarrow 5 \rightarrow 5$%
        };
    \end{axis}
\end{tikzpicture}

    \caption[Decimation Chain Variants for Rate of 625]{%
        Semilog  (top)  and   linear  plot  (middle)  for   two  variants  for
        implementing  a  decimation   chain  for  a  rate   change  factor  of
        $R=625$. Both choices  show almost the  same behavior with  regards to
        magnitude. The  bottom  plot  shows  the phase  response  of  the  two
        filters; here, too, the behavior  is very similar. \emph{Note:}  These
        plots show  the frequency responses  of the entire filter  cascade for
        the two respective variants.%
    }
    \label{fig:dec625_variants}
\end{figure}

%>>>
% ==============================================================================
%
%                         R E S O U R C E   U S A G E
%
% ==============================================================================
\section{Resource Usage for FIR Filters on the FPGA} % <<< ------------------- %
\label{sec:fir_filter_resouce_usage}
% ---------------------------------------------------------------------------- %

This   section   contains  the   experimental   results   of  how   many   DSP
slices  a  given   FIR  filter  needs  when  implemented   with  Xilinx's  FIR
Compiler. These  figures form  the basis  to determine  reasonable bounds  for
the  FIR  filter  \code{5steep} (see  Figure~  \ref{fig:fdesign:chain_concept}
on    page~\pageref{fig:fdesign:chain_concept}). Figure~\ref{fig:usage_report}
depicts      the       results      of      the       measurements,      while
Table~\ref{tab:usage_report:config}  contains   the  configuration  parameters
which were used for the FIR compiler core.

As can be  seen in the plot,  DSP slice usage rises roughly  linearly at these
high sampling  rates. When using  only a  single filter,  a filter  of roughly
\num{760} coefficients is  the maximum possible size. Because  the STEMlab has
two channels, and  because other filters are required as  well, an upper limit
for \code{5steep} of  \num{250} coefficients is set based  on these results. A
smaller  filter  is  also  acceptable  as long  as  it  fulfills  the  general
requirements.

TODO: source FIR compiler documentation

\begin{table}
    \centering
    \caption[FIR Compiler Parameters]{%
        The parameters used  to configure the FIR compiler core  for the usage
        measurements from Figure~\ref{fig:usage_report}%
    }
    \label{tab:usage_report:config}
    \begin{tabular}{lr}
        \toprule
        Parameter                   & Value          \\
        \midrule
        Clock Frequency             & \SI{125}{\MHz} \\
        Decimation Rate             & \num{5}        \\
        Input Data Width            & \SI{24}{\bit}  \\
        Input Fractional Bits       & \num{7}        \\
        Output Data Width           & \SI{32}{\bit}  \\
        Coefficient Fractional Bits & \num{17}       \\
        \bottomrule
    \end{tabular}
\end{table}

\begin{figure}
    \centering
    \tikzsetnextfilename{DSPSlicesUsageReport}
\newcommand*\usageReportFile{images/fdesign/usageReport.csv}
\pgfplotstableread[col sep=comma]{\usageReportFile}\usageReportTable
\begin{tikzpicture}[
    trim axis left,
    trim axis right,
]
     \pgfplotsset{every axis/.style={
            height=45mm,
            width=\textwidth,
            grid=none,
            %x unit=\times\,\pi\,\si{\radian}/\si{\sample},
            %y unit=\si{dB},
            ylabel=DSP Slice Count,
            xlabel=Filter Length,
            %y filter/.code={\pgfmathparse{20*log10(\pgfmathresult))}},
            %x filter/.code={\pgfmathparse{\pgfmathresult / 3.141592654}},
        },
    }
    \begin{axis}[
            at={(0,0)},
            xmin=0,
            xmax=1200,
            %ymax=5,
            %ymin=-150,
        ]
        \draw[dashed,gray] (0,80) -- (1200,80);
        \addplot[only marks,mark=*,mark options={scale=0.5},q1,-] table[x=filterSize, y=DSPCount] \usageReportTable;
    \end{axis}
\end{tikzpicture}

    \caption[Usage Report FIR Compiler]{%
        Usage  report figures  for DSP  slices using  the Xilinx  FIR compiler
        block. The  configuration  of  the  filter in  terms  of  bith  widths
        is  identical   to  the  actual   configuration  used  in   the  final
        implementation. Unlike  the  implementation,  however, only  a  single
        channel was configured.%
    }
    \label{fig:usage_report}
\end{figure}

%>>>
% ==============================================================================
%
%             F I L T E R   F R E Q U E N C Y   R E S P O N S E S
%
% ==============================================================================
\section{Filter Frequency Responses} % <<< ----------------------------------- %
\label{sec:filter_frequency_responses}
% ---------------------------------------------------------------------------- %

This section contains the frequency responses  of the filters and the cascades
as specified in Section~\ref{ch:filter_design}.
% ==============================================================================
%
%                                 5 S T E E P
%
% ==============================================================================
\subsection{\code{5steep}} % <<< --------------------------------------------- %
\label{sec:filter_frequency_responses:5steep}
% ---------------------------------------------------------------------------- %
\tikzsetnextfilename{5steepFreqResponse}
\newcommand*\freqzFileDECA{images/fdesign/r-005--fp-200--fst-225--ap-200--ast-60.csv}
\pgfplotstableread[col sep=comma]{\freqzFileDECA}\freqzTableDECA
\begin{tikzpicture}[
    trim axis left,
    trim axis right,
    ]
     \pgfplotsset{every axis/.style={
            height=45mm,
            width=\textwidth,
            grid=none,
            x unit=\times\,\pi\,\si{\radian}/\si{\sample},
            y unit=\si{dB},
            ylabel=Magnitude,
            xlabel=Normalized Frequency,
            xmin=0,
            xmax=1,
            ymax=5,
            ymin=-90,
            ytick={0,-20,-40,-60,-80},
            y filter/.code={\pgfmathparse{20*log10(\pgfmathresult))}},
            x filter/.code={\pgfmathparse{\pgfmathresult / 3.141592654}},
        },
    }
    \begin{axis}[
            at = {(0,0)},
        ]
        \addplot[thick,q1,-] table[x=W, y=abs(H)] \freqzTableDECA;
    \end{axis}
\end{tikzpicture}

%>>>
% ==============================================================================
%
%                                  5 F L A T
%
% ==============================================================================
\subsection{\code{5flat}} % <<< ---------------------------------------------- %
\label{sec:filter_frequency_responses:5flat}
% ---------------------------------------------------------------------------- %

\begin{tabular}{lS}
    \toprule
    \multicolumn{2}{c}{\sffamily\scshape\bfseries Filter Characteristics} \\
    \midrule
    Filter Length                   &   62       \\
    Passband Edge                   &    0.2     \\
    \SI{3}{\dB} Point               &    0.23491 \\
    \SI{6}{\dB} Point               &    0.24708 \\
    Stopband Edge                   &    0.3     \\
    Passband Ripple (\si{\dB})      &  0.047745  \\
    Stopband Attenuation (\si{\dB}) & 60.1991    \\
    Transition Width                &    0.1     \\
    Number of DSP Slices            &   22       \\
    \bottomrule
\end{tabular}

\vspace{1em}

\noindent\newcommand*\freqzFileDECB{images/fdesign/r-005--fp-200--fst-300--ap-050--ast-60.csv}
\pgfplotstableread[col sep=comma]{\freqzFileDECB}\freqzTableDECB
\begin{tikzpicture}
     \pgfplotsset{every axis/.style={
            height=60mm,
            width=\textwidth,
            grid=none,
            x unit=\times\,\pi\,\si{\radian}/\si{\sample},
            y unit=\si{dB},
            ylabel=Magnitude,
            xlabel=Normalized Frequency,
            xmin=0,
            xmax=1,
            ymax=5,
            ymin=-100,
            every axis legend/.append style={
                at={(1,-0.30)},  % if attached to bottom plot
                anchor=north east,
                cells={anchor=west},
            },
            y filter/.code={\pgfmathparse{20*log10(\pgfmathresult))}},
            x filter/.code={\pgfmathparse{\pgfmathresult / 3.141592654}},
        },
    }
    \begin{axis}[
            at = {(0,0)},
        ]
        \addplot[thick,q1,-] table[x=W, y=abs(H)] \freqzTableDECB;
    \end{axis}
\end{tikzpicture}

%>>>
% ==============================================================================
%
%                                 2 S T E E P
%
% ==============================================================================
\subsection{\code{2steep}} % <<< --------------------------------------------- %
\label{sec:filter_frequency_responses:2steep}
% ---------------------------------------------------------------------------- %
\tikzsetnextfilename{2steepFreqResponse}
\newcommand*\freqzFileDECC{images/fdesign/r-002--tw-0040--ast-60.csv}
\pgfplotstableread[col sep=comma]{\freqzFileDECC}\freqzTableDECC
\begin{tikzpicture}[
    trim axis left,
    trim axis right,
]
     \pgfplotsset{every axis/.style={
            height=45mm,
            width=\textwidth,
            grid=none,
            x unit=\times\,\pi\,\si{\radian}/\si{\sample},
            y unit=\si{dB},
            ylabel=Magnitude,
            xlabel=Normalized Frequency,
            xmin=0,
            xmax=1,
            ymax=5,
            ymin=-90,
            ytick={0,-20,-40,-60,-80},
            y filter/.code={\pgfmathparse{20*log10(\pgfmathresult))}},
            x filter/.code={\pgfmathparse{\pgfmathresult / 3.141592654}},
        },
    }
    \begin{axis}[
            at = {(0,0)},
        ]
        \addplot[thick,q1,-] table[x=W, y=abs(H)] \freqzTableDECC;
    \end{axis}
\end{tikzpicture}

%>>>
% ==============================================================================
%
%                                  C I C 2 5
%
% ==============================================================================
\subsection{\code{CIC25}} % <<< ---------------------------------------------- %
\label{sec:filter_frequency_responses:cic25}
% ---------------------------------------------------------------------------- %
\newcommand*\freqzFileDECD{images/fdesign/r-025--fp-008--ast-060--dl-1.csv}
\pgfplotstableread[col sep=comma]{\freqzFileDECD}\freqzTableDECD
\begin{tikzpicture}
     \pgfplotsset{every axis/.style={
            height=60mm,
            width=\textwidth,
            grid=none,
            x unit=\times\,\pi\,\si{\radian}/\si{\sample},
            y unit=\si{dB},
            ylabel=Magnitude,
            xlabel=Normalized Frequency,
            xmin=0,
            xmax=1,
            ymax=120,
            ymin=-80,
            y filter/.code={\pgfmathparse{20*log10(\pgfmathresult))}},
            x filter/.code={\pgfmathparse{\pgfmathresult / 3.141592654}},
        },
    }
    \begin{axis}[
            at = {(0,0)},
        ]
        \addplot[thick,q1,-] table[x=W, y=abs(H)] \freqzTableDECD;
    \end{axis}
\end{tikzpicture}

%>>>
% ==============================================================================
%
%                                 C F I R 2 5
%
% ==============================================================================
\subsection{\code{CFIR25}} % <<< --------------------------------------------- %
\label{sec:filter_frequency_responses:cfir25}
% ---------------------------------------------------------------------------- %
\tikzsetnextfilename{cfir25FreqResponse}
\newcommand*\freqzFileDECF{images/fdesign/r-025--fp-0080--fst-0160--ap-050--ast-60--dl-1.csv}
\pgfplotstableread[col sep=comma]{\freqzFileDECF}\freqzTableDECF
\begin{tikzpicture}[
    trim axis left,
    trim axis right,
]
     \pgfplotsset{every axis/.style={
            height=45mm,
            width=\textwidth,
            grid=none,
            x unit=\times\,\pi\,\si{\radian}/\si{\sample},
            y unit=\si{dB},
            ylabel=Magnitude,
            xlabel=Normalized Frequency,
            xmin=0,
            xmax=1,
            ymax=5,
            ymin=-90,
            ytick={0,-20,-40,-60,-80},
            y filter/.code={\pgfmathparse{20*log10(\pgfmathresult))}},
            x filter/.code={\pgfmathparse{\pgfmathresult / 3.141592654}},
        },
    }
    \begin{axis}[
            at = {(0,0)},
        ]
        \addplot[thick,q1,-] table[x=W, y=abs(H)] \freqzTableDECF;
    \end{axis}
\end{tikzpicture}

%>>>
% ==============================================================================
%
%                                 C I C 1 2 5
%
% ==============================================================================
\subsection{\code{CIC125}} % <<< --------------------------------------------- %
\label{sec:filter_frequency_responses:cic125}
% ---------------------------------------------------------------------------- %
\newcommand*\freqzFileDECE{images/fdesign/r-125--fp-002--ast-060--dl-1.csv}
\pgfplotstableread[col sep=comma]{\freqzFileDECE}\freqzTableDECE
\begin{tikzpicture}
     \pgfplotsset{every axis/.style={
            height=60mm,
            width=\textwidth,
            grid=none,
            x unit=\times\,\pi\,\si{\radian}/\si{\sample},
            y unit=\si{dB},
            ylabel=Magnitude,
            xlabel=Normalized Frequency,
            xmin=0,
            xmax=1,
            ymax=200,
            ymin=-50,
            y filter/.code={\pgfmathparse{20*log10(\pgfmathresult))}},
            x filter/.code={\pgfmathparse{\pgfmathresult / 3.141592654}},
        },
    }
    \begin{axis}[
            at = {(0,0)},
        ]
        \addplot[thick,q1,-] table[x=W, y=abs(H)] \freqzTableDECE;
    \end{axis}
\end{tikzpicture}

%>>>
% ==============================================================================
%
%                                C F I R 1 2 5
%
% ==============================================================================
\subsection{\code{CFIR125}} % <<< -------------------------------------------- %
\label{sec:filter_frequency_responses:cfir125}
% ---------------------------------------------------------------------------- %
\tikzsetnextfilename{cfir125FreqResponse}
\newcommand*\freqzFileDECG{images/fdesign/r-125--fp-0016--fst-0024--ap-050--ast-60--dl-1.csv}
\pgfplotstableread[col sep=comma]{\freqzFileDECG}\freqzTableDECG
\begin{tikzpicture}[
    trim axis left,
    trim axis right,
]
     \pgfplotsset{every axis/.style={
            height=45mm,
            width=\textwidth,
            grid=none,
            x unit=\times\,\pi\,\si{\radian}/\si{\sample},
            y unit=\si{dB},
            ylabel=Magnitude,
            xlabel=Normalized Frequency,
            xmin=0,
            xmax=1,
            ymax=5,
            ymin=-90,
            ytick={0,-20,-40,-60,-80},
            y filter/.code={\pgfmathparse{20*log10(\pgfmathresult))}},
            x filter/.code={\pgfmathparse{\pgfmathresult / 3.141592654}},
        },
    }
    \begin{axis}[
            at = {(0,0)},
        ]
        \addplot[thick,q1,-] table[x=W, y=abs(H)] \freqzTableDECG;
    \end{axis}
\end{tikzpicture}

%>>>
% ==============================================================================
%
%                       C H A I N   F O R   R   =   2 5
%
% ==============================================================================
\subsection{Chain for R=25} % <<< -------------------------------------------- %
\label{sec:filter_frequency_responses:chain25}
% ---------------------------------------------------------------------------- %
\newcommand*\freqzFileDECH{images/fdesign/chain25detail.csv}
\newcommand*\freqzFileDECI{images/fdesign/r-025--fp-04000--fst-22500--ap-100--ast-60--dl-1--stages-2.csv}
\pgfplotstableread[col sep=comma]{\freqzFileDECH}\freqzTableDECH
\pgfplotstableread[col sep=comma]{\freqzFileDECI}\freqzTableDECI
\begin{tikzpicture}
     \pgfplotsset{every axis/.style={
            height=60mm,
            width=\textwidth,
            grid=none,
            x unit=\times\,\pi\,\si{\radian}/\si{\sample},
            y unit=\si{dB},
            ylabel=Magnitude,
            xlabel=Normalized Frequency,
            y filter/.code={\pgfmathparse{20*log10(\pgfmathresult))}},
            x filter/.code={\pgfmathparse{\pgfmathresult / 3.141592654}},
        },
    }
    \begin{axis}[
            title={Entire Frequency Range},
            at={(0,0)},
            xmin=0,
            xmax=1,
            ymax=5,
            ymin=-150,
        ]
        \addplot[thick,q1,-] table[x=W, y=abs(H)] \freqzTableDECI;
    \end{axis}
    \begin{axis}[
            title={Passband Detail},
            at = {(0,-70mm)},
            xmin=0,
            xmax=8e-2,
            ymax=5,
            ymin=-80,
        ]
        \addplot[thick,q1,-] table[x=W, y=abs(H)] \freqzTableDECH;
    \end{axis}
\end{tikzpicture}

%>>>
% ==============================================================================
%
%                      C H A I N   F O R   R   =  1 2 5
%
% ==============================================================================
\subsection{Chain for R=125} % <<< ------------------------------------------- %
\label{sec:filter_frequency_responses:chain125}
% ---------------------------------------------------------------------------- %
\newcommand*\freqzFileDECJ{images/fdesign/chain125detail.csv}
\newcommand*\freqzFileDECK{images/fdesign/r-125--fp-00800--fst-00900--ap-250--ast-60--dl-1--stages-3.csv}
\pgfplotstableread[col sep=comma]{\freqzFileDECJ}\freqzTableDECJ
\pgfplotstableread[col sep=comma]{\freqzFileDECK}\freqzTableDECK
\begin{tikzpicture}
     \pgfplotsset{every axis/.style={
            height=60mm,
            width=\textwidth,
            grid=none,
            x unit=\times\,\pi\,\si{\radian}/\si{\sample},
            y unit=\si{dB},
            ylabel=Magnitude,
            xlabel=Normalized Frequency,
            y filter/.code={\pgfmathparse{20*log10(\pgfmathresult))}},
            x filter/.code={\pgfmathparse{\pgfmathresult / 3.141592654}},
        },
    }
    \begin{axis}[
            title={Entire Frequency Range},
            at={(0,0)},
            xmin=0,
            xmax=1,
            ymax=130,
            ymin=-140,
        ]
        \addplot[thick,q1,-] table[x=W, y=abs(H)] \freqzTableDECK;
    \end{axis}
    \begin{axis}[
            title={Passband Detail},
            at = {(0,-70mm)},
            xmin=0,
            xmax=1.6e-2,
            ymax=130,
            ymin=-20,
        ]
        \addplot[thick,q1,-] table[x=W, y=abs(H)] \freqzTableDECJ;
    \end{axis}
\end{tikzpicture}

%>>>
% ==============================================================================
%
%                      C H A I N   F O R   R   =  6 2 5
%
% ==============================================================================
\subsection{Chain for R=625} % <<< ------------------------------------------- %
\label{sec:filter_frequency_responses:chain625}
% ---------------------------------------------------------------------------- %
\newcommand*\freqzFileDECL{images/fdesign/chain625detail.csv}
\newcommand*\freqzFileDECM{images/fdesign/r-625--fp-00160--fst-00180--ap-300--ast-60--dl-1--stages-4.csv}
\pgfplotstableread[col sep=comma]{\freqzFileDECL}\freqzTableDECL
\pgfplotstableread[col sep=comma]{\freqzFileDECM}\freqzTableDECM
\begin{tikzpicture}
     \pgfplotsset{every axis/.style={
            height=60mm,
            width=\textwidth,
            grid=none,
            x unit=\times\,\pi\,\si{\radian}/\si{\sample},
            y unit=\si{dB},
            ylabel=Magnitude,
            xlabel=Normalized Frequency,
            y filter/.code={\pgfmathparse{20*log10(\pgfmathresult))}},
            x filter/.code={\pgfmathparse{\pgfmathresult / 3.141592654}},
        },
    }
    \begin{axis}[
            title={Entire Frequency Range},
            at={(0,0)},
            xmin=0,
            xmax=1,
            ymax=150,
            ymin=-200,
        ]
        \addplot[thick,q1,-] table[x=W, y=abs(H)] \freqzTableDECM;
    \end{axis}
    \begin{axis}[
            title={Passband Detail},
            at = {(0,-70mm)},
            xmin=0,
            xmax=3.2e-3,
            ymax=130,
            ymin=10,
        ]
        \addplot[thick,q1,-] table[x=W, y=abs(H)] \freqzTableDECL;
    \end{axis}
\end{tikzpicture}

%>>>
% ==============================================================================
%
%                     C H A I N   F O R   R   =  1 2 5 0
%
% ==============================================================================
\subsection{Chain for R=1250} % <<< ------------------------------------------ %
\label{sec:filter_frequency_responses:chain1250}
% ---------------------------------------------------------------------------- %
\newcommand*\freqzFileDECN{images/fdesign/chain1250detail.csv}
\newcommand*\freqzFileDECO{images/fdesign/r-1250--fp-00080--fst-00083--ap-250--ast-60--dl-1--stages-3.csv}
\pgfplotstableread[col sep=comma]{\freqzFileDECN}\freqzTableDECN
\pgfplotstableread[col sep=comma]{\freqzFileDECO}\freqzTableDECO
\begin{tikzpicture}
     \pgfplotsset{every axis/.style={
            height=60mm,
            width=\textwidth,
            grid=none,
            x unit=\times\,\pi\,\si{\radian}/\si{\sample},
            y unit=\si{dB},
            ylabel=Magnitude,
            xlabel=Normalized Frequency,
            y filter/.code={\pgfmathparse{20*log10(\pgfmathresult))}},
            x filter/.code={\pgfmathparse{\pgfmathresult / 3.141592654}},
        },
    }
    \begin{axis}[
            title={Entire Frequency Range},
            at={(0,0)},
            xmin=0,
            xmax=1,
            ymax=180,
            ymin=-150,
        ]
        \addplot[thick,q1,-] table[x=W, y=abs(H)] \freqzTableDECO;
    \end{axis}
    \begin{axis}[
            title={Passband Detail},
            at = {(0,-70mm)},
            xmin=0,
            xmax=1.6e-3,
            ymax=180,
            ymin=70,
        ]
        \addplot[thick,q1,-] table[x=W, y=abs(H)] \freqzTableDECN;
    \end{axis}
\end{tikzpicture}

%>>>
% ==============================================================================
%
%                     C H A I N   F O R   R   =  2 5 0 0
%
% ==============================================================================
\subsection{Chain for R=2500} % <<< ------------------------------------------ %
\label{sec:filter_frequency_responses:chain2500}
% ---------------------------------------------------------------------------- %
\tikzsetnextfilename{chain2500FreqResponse}
\newcommand*\freqzFileDECP{images/fdesign/chain2500detail.csv}
\newcommand*\freqzFileDECQ{images/fdesign/r-1250--fp-00080--fst-00160--ap-450--ast-60--dl-1--stages-4.csv}
\pgfplotstableread[col sep=comma]{\freqzFileDECP}\freqzTableDECP
\pgfplotstableread[col sep=comma]{\freqzFileDECQ}\freqzTableDECQ
\begin{tikzpicture}[
    trim axis left,
    trim axis right,
]
     \pgfplotsset{every axis/.style={
            height=45mm,
            width=\textwidth,
            grid=none,
            x unit=\times\,\pi\,\si{\radian}/\si{\sample},
            y unit=\si{dB},
            ylabel=Magnitude,
            xlabel=Normalized Frequency,
            y filter/.code={\pgfmathparse{20*log10(\pgfmathresult))}},
            x filter/.code={\pgfmathparse{\pgfmathresult / 3.141592654}},
        },
    }
    \begin{axis}[
            title={Entire Frequency Range},
            at={(0,0)},
            xmin=0,
            xmax=1,
            ymax=180,
            ymin=-100,
        ]
        \addplot[thin,q1,-] table[x=W, y=abs(H)] \freqzTableDECQ;
    \end{axis}
    \begin{axis}[
            title={Passband Detail},
            at = {(0,-65mm)},
            xmin=0,
            xmax=8e-4,
            ymax=180,
            ymin=70,
        ]
        \addplot[thick,q1,-] table[x=W, y=abs(H)] \freqzTableDECP;
    \end{axis}
\end{tikzpicture}

%>>>
%>>>

%^^A vim: foldenable foldcolumn=4 foldmethod=marker foldmarker=<<<,>>>
