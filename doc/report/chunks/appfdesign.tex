\chapter{Filter Design} % <<< ------------------------------------------------ %
\label{ch:app:fdesign}
% ---------------------------------------------------------------------------- %

This  chapter contains  some additional  information about  the filter  design
process.

\section{Decimation of 625: Variants} % <<< ---------------------------------- %
\label{sec:dec625_variants}
% ---------------------------------------------------------------------------- %

\begin{figure}
    \centering
    \tikzsetnextfilename{dec625VariantsFreqResponse}
\newcommand*\freqzFileDECA{images/fdesign/dec625--125-5.csv}
\newcommand*\freqzFileDECB{images/fdesign/dec625--25-5-5.csv}
\pgfplotstableread[col sep=comma]{\freqzFileDECA}\freqzTableDECA
\pgfplotstableread[col sep=comma]{\freqzFileDECB}\freqzTableDECB
\begin{tikzpicture}[
    trim axis left,
    trim axis right,
]
     \pgfplotsset{every axis/.style={
            height=45mm,
            width=\textwidth,
            grid=none,
            x unit=\times\,\pi\,\si{\radian}/\si{\sample},
            ylabel=Magnitude,
            xlabel=Normalized Frequency,
            xmin=0,
            xmax=3e-3,
            %xmin=0,
            %xmax=1000000,
            %ymax=5,
            %ymin=-120,
            legend columns=2,
            every axis legend/.append style={
                at={(0.5,-0.30)},  % if attached to bottom plot
                anchor=north,
                cells={anchor=west},
            },
        },
    }
    \begin{axis}[
            title={Decimation of $R=625$: Variants},
            at = {(0,0)},
            y unit=\si{dB},
            ymax=5,
            ymin=-180,
            y filter/.code={\pgfmathparse{20*log10(\pgfmathresult))}},
            x filter/.code={\pgfmathparse{\pgfmathresult / 3.141592654}},
        ]
        \addplot[thick,q1,-] table[x=W, y=abs(H)] \freqzTableDECA;
        \addplot[thick,q5,-] table[x=W, y=abs(H)] \freqzTableDECB;
    \end{axis}
    \begin{axis}[
            at = {(0,-62mm)},
            x filter/.code={\pgfmathparse{\pgfmathresult / 3.141592654}},
            %xtick={0,0.3e6,0.4e6,1e6},
        ]
        \addplot[thick,q1,-] table[x=W, y=abs(H)] \freqzTableDECA;
        \addplot[thick,q5,-] table[x=W, y=abs(H)] \freqzTableDECB;
    \end{axis}
    \begin{axis}[
            at = {(0,-124mm)},
            ylabel=Phase,
            y unit=\si{\degree},
            ymin=-2800,
            ymax=0,
            y filter/.code={\pgfmathparse{180*\pgfmathresult/3.141592654}},
            unbounded coords=discard,
            point meta=explicit,
            filter point/.code={%
                \pgfmathparse{\pgfkeysvalueof{/data point/meta}==0}
                \ifpgfmathfloatcomparison
                    \pgfkeyssetvalue{/data point/y}{nan}%
                \fi
            }
        ]
        \addplot[thick,-,q1,] table[x=W,y=angle(H),meta=abs(H)] \freqzTableDECA;
        \addplot[thick,-,q5] table[x=W,y=angle(H),meta=abs(H)] \freqzTableDECB;
        \legend{%
            $125 \rightarrow \text{CFIR} \rightarrow 5$,
            $ 25 \rightarrow \text{CFIR} \rightarrow 5 \rightarrow 5$%
        };
    \end{axis}
\end{tikzpicture}

    \caption[Decimation Chain Variants for Rate of 625]{%
        Semilog  (top)  and   linear  plot  (middle)  for   two  variants  for
        implementing  a  decimation   chain  for  a  rate   change  factor  of
        $R=625$. As is  evident, both  choices show  almost the  same behavior
        with regards to magnitude. The bottom plot shows the phase response of
        the two filters; here, too, the behavior is very similar. \emph{Note:}
        These plots show the frequency  responses of the entire filter cascade
        for the two respective variants.%
    }
    \label{fig:dec625_variants}
\end{figure}

%>>>

\section{Resource Usage for FIR Filters on the FPGA} % <<< ------------------- %
\label{sec:fir_filter_resouce_usage}
% ---------------------------------------------------------------------------- %

This   section   contains  the   experimental   results   of  how   many   DSP
slices  a  given   FIR  filter  needs  when  implemented   with  Xilinx's  FIR
Compiler. These  figures were  used  to determine  reasonable  bounds for  the
FIR  filter  \code{5steep}  (see  Figure~  \ref{fig:fdesign:chain_concept}  on
page~\pageref{fig:fdesign:chain_concept}).
TODO: source

%>>>

\section{Halfband Filters} % <<< --------------------------------------------- %
\label{sec:halfband_filters}
% ---------------------------------------------------------------------------- %


%>>>

\section{Filter Frequency Responses} % <<< ----------------------------------- %
\label{sec:filter_frequency_responses}
% ---------------------------------------------------------------------------- %

This section contains the frequency responses  of the filters and the cascades
as specified in Section~\ref{ch:filter_design}.

\subsection{\code{5steep}} % <<< --------------------------------------------- %
\label{sec:filter_frequency_responses:5steep}
% ---------------------------------------------------------------------------- %
\tikzsetnextfilename{5steepFreqResponse}
\newcommand*\freqzFileDECA{images/fdesign/r-005--fp-200--fst-225--ap-200--ast-60.csv}
\pgfplotstableread[col sep=comma]{\freqzFileDECA}\freqzTableDECA
\begin{tikzpicture}[
    trim axis left,
    trim axis right,
    ]
     \pgfplotsset{every axis/.style={
            height=45mm,
            width=\textwidth,
            grid=none,
            x unit=\times\,\pi\,\si{\radian}/\si{\sample},
            y unit=\si{dB},
            ylabel=Magnitude,
            xlabel=Normalized Frequency,
            xmin=0,
            xmax=1,
            ymax=5,
            ymin=-90,
            ytick={0,-20,-40,-60,-80},
            y filter/.code={\pgfmathparse{20*log10(\pgfmathresult))}},
            x filter/.code={\pgfmathparse{\pgfmathresult / 3.141592654}},
        },
    }
    \begin{axis}[
            at = {(0,0)},
        ]
        \addplot[thick,q1,-] table[x=W, y=abs(H)] \freqzTableDECA;
    \end{axis}
\end{tikzpicture}

%>>>
\subsection{\code{5flat}} % <<< ---------------------------------------------- %
\label{sec:filter_frequency_responses:5flat}
% ---------------------------------------------------------------------------- %
\newcommand*\freqzFileDECB{images/fdesign/r-005--fp-200--fst-300--ap-050--ast-60.csv}
\pgfplotstableread[col sep=comma]{\freqzFileDECB}\freqzTableDECB
\begin{tikzpicture}
     \pgfplotsset{every axis/.style={
            height=60mm,
            width=\textwidth,
            grid=none,
            x unit=\times\,\pi\,\si{\radian}/\si{\sample},
            y unit=\si{dB},
            ylabel=Magnitude,
            xlabel=Normalized Frequency,
            xmin=0,
            xmax=1,
            ymax=5,
            ymin=-100,
            every axis legend/.append style={
                at={(1,-0.30)},  % if attached to bottom plot
                anchor=north east,
                cells={anchor=west},
            },
            y filter/.code={\pgfmathparse{20*log10(\pgfmathresult))}},
            x filter/.code={\pgfmathparse{\pgfmathresult / 3.141592654}},
        },
    }
    \begin{axis}[
            at = {(0,0)},
        ]
        \addplot[thick,q1,-] table[x=W, y=abs(H)] \freqzTableDECB;
    \end{axis}
\end{tikzpicture}

%>>>
\subsection{\code{2steep}} % <<< --------------------------------------------- %
\label{sec:filter_frequency_responses:2steep}
% ---------------------------------------------------------------------------- %
\tikzsetnextfilename{2steepFreqResponse}
\newcommand*\freqzFileDECC{images/fdesign/r-002--tw-0040--ast-60.csv}
\pgfplotstableread[col sep=comma]{\freqzFileDECC}\freqzTableDECC
\begin{tikzpicture}[
    trim axis left,
    trim axis right,
]
     \pgfplotsset{every axis/.style={
            height=45mm,
            width=\textwidth,
            grid=none,
            x unit=\times\,\pi\,\si{\radian}/\si{\sample},
            y unit=\si{dB},
            ylabel=Magnitude,
            xlabel=Normalized Frequency,
            xmin=0,
            xmax=1,
            ymax=5,
            ymin=-90,
            ytick={0,-20,-40,-60,-80},
            y filter/.code={\pgfmathparse{20*log10(\pgfmathresult))}},
            x filter/.code={\pgfmathparse{\pgfmathresult / 3.141592654}},
        },
    }
    \begin{axis}[
            at = {(0,0)},
        ]
        \addplot[thick,q1,-] table[x=W, y=abs(H)] \freqzTableDECC;
    \end{axis}
\end{tikzpicture}

%>>>
\subsection{\code{CIC25}} % <<< ---------------------------------------------- %
\label{sec:filter_frequency_responses:cic25}
% ---------------------------------------------------------------------------- %
\newcommand*\freqzFileDECD{images/fdesign/r-025--fp-008--ast-060--dl-1.csv}
\pgfplotstableread[col sep=comma]{\freqzFileDECD}\freqzTableDECD
\begin{tikzpicture}
     \pgfplotsset{every axis/.style={
            height=60mm,
            width=\textwidth,
            grid=none,
            x unit=\times\,\pi\,\si{\radian}/\si{\sample},
            y unit=\si{dB},
            ylabel=Magnitude,
            xlabel=Normalized Frequency,
            xmin=0,
            xmax=1,
            ymax=120,
            ymin=-80,
            y filter/.code={\pgfmathparse{20*log10(\pgfmathresult))}},
            x filter/.code={\pgfmathparse{\pgfmathresult / 3.141592654}},
        },
    }
    \begin{axis}[
            at = {(0,0)},
        ]
        \addplot[thick,q1,-] table[x=W, y=abs(H)] \freqzTableDECD;
    \end{axis}
\end{tikzpicture}

%>>>
\subsection{\code{CFIR25}} % <<< --------------------------------------------- %
\label{sec:filter_frequency_responses:cfir25}
% ---------------------------------------------------------------------------- %
\tikzsetnextfilename{cfir25FreqResponse}
\newcommand*\freqzFileDECF{images/fdesign/r-025--fp-0080--fst-0160--ap-050--ast-60--dl-1.csv}
\pgfplotstableread[col sep=comma]{\freqzFileDECF}\freqzTableDECF
\begin{tikzpicture}[
    trim axis left,
    trim axis right,
]
     \pgfplotsset{every axis/.style={
            height=45mm,
            width=\textwidth,
            grid=none,
            x unit=\times\,\pi\,\si{\radian}/\si{\sample},
            y unit=\si{dB},
            ylabel=Magnitude,
            xlabel=Normalized Frequency,
            xmin=0,
            xmax=1,
            ymax=5,
            ymin=-90,
            ytick={0,-20,-40,-60,-80},
            y filter/.code={\pgfmathparse{20*log10(\pgfmathresult))}},
            x filter/.code={\pgfmathparse{\pgfmathresult / 3.141592654}},
        },
    }
    \begin{axis}[
            at = {(0,0)},
        ]
        \addplot[thick,q1,-] table[x=W, y=abs(H)] \freqzTableDECF;
    \end{axis}
\end{tikzpicture}

%>>>
\subsection{\code{CIC125}} % <<< --------------------------------------------- %
\label{sec:filter_frequency_responses:cic125}
% ---------------------------------------------------------------------------- %
\newcommand*\freqzFileDECE{images/fdesign/r-125--fp-002--ast-060--dl-1.csv}
\pgfplotstableread[col sep=comma]{\freqzFileDECE}\freqzTableDECE
\begin{tikzpicture}
     \pgfplotsset{every axis/.style={
            height=60mm,
            width=\textwidth,
            grid=none,
            x unit=\times\,\pi\,\si{\radian}/\si{\sample},
            y unit=\si{dB},
            ylabel=Magnitude,
            xlabel=Normalized Frequency,
            xmin=0,
            xmax=1,
            ymax=200,
            ymin=-50,
            y filter/.code={\pgfmathparse{20*log10(\pgfmathresult))}},
            x filter/.code={\pgfmathparse{\pgfmathresult / 3.141592654}},
        },
    }
    \begin{axis}[
            at = {(0,0)},
        ]
        \addplot[thick,q1,-] table[x=W, y=abs(H)] \freqzTableDECE;
    \end{axis}
\end{tikzpicture}

%>>>
\subsection{\code{CFIR125}} % <<< -------------------------------------------- %
\label{sec:filter_frequency_responses:cfir125}
% ---------------------------------------------------------------------------- %
\tikzsetnextfilename{cfir125FreqResponse}
\newcommand*\freqzFileDECG{images/fdesign/r-125--fp-0016--fst-0024--ap-050--ast-60--dl-1.csv}
\pgfplotstableread[col sep=comma]{\freqzFileDECG}\freqzTableDECG
\begin{tikzpicture}[
    trim axis left,
    trim axis right,
]
     \pgfplotsset{every axis/.style={
            height=45mm,
            width=\textwidth,
            grid=none,
            x unit=\times\,\pi\,\si{\radian}/\si{\sample},
            y unit=\si{dB},
            ylabel=Magnitude,
            xlabel=Normalized Frequency,
            xmin=0,
            xmax=1,
            ymax=5,
            ymin=-90,
            ytick={0,-20,-40,-60,-80},
            y filter/.code={\pgfmathparse{20*log10(\pgfmathresult))}},
            x filter/.code={\pgfmathparse{\pgfmathresult / 3.141592654}},
        },
    }
    \begin{axis}[
            at = {(0,0)},
        ]
        \addplot[thick,q1,-] table[x=W, y=abs(H)] \freqzTableDECG;
    \end{axis}
\end{tikzpicture}

%>>>
\subsection{Chain for R=25} % <<< -------------------------------------------- %
\label{sec:filter_frequency_responses:chain25}
% ---------------------------------------------------------------------------- %
\newcommand*\freqzFileDECH{images/fdesign/chain25detail.csv}
\newcommand*\freqzFileDECI{images/fdesign/r-025--fp-04000--fst-22500--ap-100--ast-60--dl-1--stages-2.csv}
\pgfplotstableread[col sep=comma]{\freqzFileDECH}\freqzTableDECH
\pgfplotstableread[col sep=comma]{\freqzFileDECI}\freqzTableDECI
\begin{tikzpicture}
     \pgfplotsset{every axis/.style={
            height=60mm,
            width=\textwidth,
            grid=none,
            x unit=\times\,\pi\,\si{\radian}/\si{\sample},
            y unit=\si{dB},
            ylabel=Magnitude,
            xlabel=Normalized Frequency,
            y filter/.code={\pgfmathparse{20*log10(\pgfmathresult))}},
            x filter/.code={\pgfmathparse{\pgfmathresult / 3.141592654}},
        },
    }
    \begin{axis}[
            title={Entire Frequency Range},
            at={(0,0)},
            xmin=0,
            xmax=1,
            ymax=5,
            ymin=-150,
        ]
        \addplot[thick,q1,-] table[x=W, y=abs(H)] \freqzTableDECI;
    \end{axis}
    \begin{axis}[
            title={Passband Detail},
            at = {(0,-70mm)},
            xmin=0,
            xmax=8e-2,
            ymax=5,
            ymin=-80,
        ]
        \addplot[thick,q1,-] table[x=W, y=abs(H)] \freqzTableDECH;
    \end{axis}
\end{tikzpicture}

%>>>
\subsection{Chain for R=125} % <<< -------------------------------------------- %
\label{sec:filter_frequency_responses:chain125}
% ---------------------------------------------------------------------------- %
\newcommand*\freqzFileDECJ{images/fdesign/chain125detail.csv}
\newcommand*\freqzFileDECK{images/fdesign/r-125--fp-00800--fst-00900--ap-250--ast-60--dl-1--stages-3.csv}
\pgfplotstableread[col sep=comma]{\freqzFileDECJ}\freqzTableDECJ
\pgfplotstableread[col sep=comma]{\freqzFileDECK}\freqzTableDECK
\begin{tikzpicture}
     \pgfplotsset{every axis/.style={
            height=60mm,
            width=\textwidth,
            grid=none,
            x unit=\times\,\pi\,\si{\radian}/\si{\sample},
            y unit=\si{dB},
            ylabel=Magnitude,
            xlabel=Normalized Frequency,
            y filter/.code={\pgfmathparse{20*log10(\pgfmathresult))}},
            x filter/.code={\pgfmathparse{\pgfmathresult / 3.141592654}},
        },
    }
    \begin{axis}[
            title={Entire Frequency Range},
            at={(0,0)},
            xmin=0,
            xmax=1,
            ymax=130,
            ymin=-140,
        ]
        \addplot[thick,q1,-] table[x=W, y=abs(H)] \freqzTableDECK;
    \end{axis}
    \begin{axis}[
            title={Passband Detail},
            at = {(0,-70mm)},
            xmin=0,
            xmax=1.6e-2,
            ymax=130,
            ymin=-20,
        ]
        \addplot[thick,q1,-] table[x=W, y=abs(H)] \freqzTableDECJ;
    \end{axis}
\end{tikzpicture}

%>>>
\subsection{Chain for R=625} % <<< -------------------------------------------- %
\label{sec:filter_frequency_responses:chain625}
% ---------------------------------------------------------------------------- %
\newcommand*\freqzFileDECL{images/fdesign/chain625detail.csv}
\newcommand*\freqzFileDECM{images/fdesign/r-625--fp-00160--fst-00180--ap-300--ast-60--dl-1--stages-4.csv}
\pgfplotstableread[col sep=comma]{\freqzFileDECL}\freqzTableDECL
\pgfplotstableread[col sep=comma]{\freqzFileDECM}\freqzTableDECM
\begin{tikzpicture}
     \pgfplotsset{every axis/.style={
            height=60mm,
            width=\textwidth,
            grid=none,
            x unit=\times\,\pi\,\si{\radian}/\si{\sample},
            y unit=\si{dB},
            ylabel=Magnitude,
            xlabel=Normalized Frequency,
            y filter/.code={\pgfmathparse{20*log10(\pgfmathresult))}},
            x filter/.code={\pgfmathparse{\pgfmathresult / 3.141592654}},
        },
    }
    \begin{axis}[
            title={Entire Frequency Range},
            at={(0,0)},
            xmin=0,
            xmax=1,
            ymax=150,
            ymin=-200,
        ]
        \addplot[thick,q1,-] table[x=W, y=abs(H)] \freqzTableDECM;
    \end{axis}
    \begin{axis}[
            title={Passband Detail},
            at = {(0,-70mm)},
            xmin=0,
            xmax=3.2e-3,
            ymax=130,
            ymin=10,
        ]
        \addplot[thick,q1,-] table[x=W, y=abs(H)] \freqzTableDECL;
    \end{axis}
\end{tikzpicture}

%>>>
\subsection{Chain for R=1250} % <<< -------------------------------------------- %
\label{sec:filter_frequency_responses:chain1250}
% ---------------------------------------------------------------------------- %
\newcommand*\freqzFileDECN{images/fdesign/chain1250detail.csv}
\newcommand*\freqzFileDECO{images/fdesign/r-1250--fp-00080--fst-00083--ap-250--ast-60--dl-1--stages-3.csv}
\pgfplotstableread[col sep=comma]{\freqzFileDECN}\freqzTableDECN
\pgfplotstableread[col sep=comma]{\freqzFileDECO}\freqzTableDECO
\begin{tikzpicture}
     \pgfplotsset{every axis/.style={
            height=60mm,
            width=\textwidth,
            grid=none,
            x unit=\times\,\pi\,\si{\radian}/\si{\sample},
            y unit=\si{dB},
            ylabel=Magnitude,
            xlabel=Normalized Frequency,
            y filter/.code={\pgfmathparse{20*log10(\pgfmathresult))}},
            x filter/.code={\pgfmathparse{\pgfmathresult / 3.141592654}},
        },
    }
    \begin{axis}[
            title={Entire Frequency Range},
            at={(0,0)},
            xmin=0,
            xmax=1,
            ymax=180,
            ymin=-150,
        ]
        \addplot[thick,q1,-] table[x=W, y=abs(H)] \freqzTableDECO;
    \end{axis}
    \begin{axis}[
            title={Passband Detail},
            at = {(0,-70mm)},
            xmin=0,
            xmax=1.6e-3,
            ymax=180,
            ymin=70,
        ]
        \addplot[thick,q1,-] table[x=W, y=abs(H)] \freqzTableDECN;
    \end{axis}
\end{tikzpicture}

%>>>
\subsection{Chain for R=2500} % <<< -------------------------------------------- %
\label{sec:filter_frequency_responses:chain2500}
% ---------------------------------------------------------------------------- %
\tikzsetnextfilename{chain2500FreqResponse}
\newcommand*\freqzFileDECP{images/fdesign/chain2500detail.csv}
\newcommand*\freqzFileDECQ{images/fdesign/r-1250--fp-00080--fst-00160--ap-450--ast-60--dl-1--stages-4.csv}
\pgfplotstableread[col sep=comma]{\freqzFileDECP}\freqzTableDECP
\pgfplotstableread[col sep=comma]{\freqzFileDECQ}\freqzTableDECQ
\begin{tikzpicture}[
    trim axis left,
    trim axis right,
]
     \pgfplotsset{every axis/.style={
            height=45mm,
            width=\textwidth,
            grid=none,
            x unit=\times\,\pi\,\si{\radian}/\si{\sample},
            y unit=\si{dB},
            ylabel=Magnitude,
            xlabel=Normalized Frequency,
            y filter/.code={\pgfmathparse{20*log10(\pgfmathresult))}},
            x filter/.code={\pgfmathparse{\pgfmathresult / 3.141592654}},
        },
    }
    \begin{axis}[
            title={Entire Frequency Range},
            at={(0,0)},
            xmin=0,
            xmax=1,
            ymax=180,
            ymin=-100,
        ]
        \addplot[thin,q1,-] table[x=W, y=abs(H)] \freqzTableDECQ;
    \end{axis}
    \begin{axis}[
            title={Passband Detail},
            at = {(0,-65mm)},
            xmin=0,
            xmax=8e-4,
            ymax=180,
            ymin=70,
        ]
        \addplot[thick,q1,-] table[x=W, y=abs(H)] \freqzTableDECP;
    \end{axis}
\end{tikzpicture}

%>>>


%>>>

%>>>
%^^A vim: foldenable foldcolumn=4 foldmethod=marker foldmarker=<<<,>>>
