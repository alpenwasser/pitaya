% ==============================================================================
%
%                               A B S T R A C T
%
% ==============================================================================
\chapter*{Abstract} % -------------------------------------------------------- %
\label{ch:app:abstract}
% ---------------------------------------------------------------------------- %

The objective  of this  project is  to realize a  system for  measuring analog
signals  in  the kilohertz  to  low  megahertz  range  with a  digital  signal
processing  system  in  real  time,  offering  an  affordable  alternative  to
expensive oscilloscopes  and spectrum  analyzers. The basic components  of the
system are a  Red Pitaya STEMlab for signal acquisition  and processing  and a
personal  computer  or  mobile  device  for  data  visualization  and  further
analysis.

To  allow transmission  over  an  Ethernet connection  and  to improve  signal
quality, the high-rate data stream coming from the STEMlab's analog-to-digital
converter is decimated  to a lower-rate signal using  the integrated FPGA. The
decimated signal is then sent to a client over a network connection and can be
observed with a newly  developed oscilloscope application.  During decimation,
the  signal is  filtered  through  one of  six  filter  cascades to  attenuate
aliasing effects.

The cascades  run on the STETMlab's  FPGA. They are based on  a combination of
FIR  and CIC  filters,  and  decimate the  incoming  \SI{125}{\MHz} signal  to
output frequencies  between \SI{50}{\kHz} and \SI{25}{\MHz},  depending on the
chain. The chains achieve an aliasing  attenuation of \SI{60}{\dB} and exhibit
negligible passband droop. A  signal-to-noise ratio of up  to \SI{85}{\dB} has
been measured.

The oscilloscope application  is based on web technologies,  allowing for easy
deployment  across various  platforms. Lastly,  a  comprehensive toolchain  to
extend and modify  the system is provided. All project  components created for
this project are open source and available under the MIT license.



%%%
%This involves the  development of
%appropriate filters for attenuating aliasing effects, a communication back end
%to transmit  data via Ethernet,  and an oscilloscope application  to visualize
%and analyze the results.
%
%Six  filter  chains  are  implemented  on  the  STEMlab's  FPGA,  allowing  to
%downsample   the  \SI{125}{\MHz}   source   signal   to  frequencies   between
%\SI{50}{\kHz} and \SI{25}{\MHz}.  Using a  combination of CIC and FIR filters,
%a stopband  attenuation of  \SI{60}{\dB} is achieved. RMS  in the  passband is
%linear  to within  \SI{2}{\percent}  except at  \SI{25}{\MHz},  and ripple  is
%kept  to  approximately \SI{1}{\percent}. SNR  in  the  passband lies  between
%\SI{70}{\dB} and \SI{80}{\dB} as calculated by Matlab's SNR algorithms.
%
%The oscilloscope application  is based on web technologies,  allowing for easy
%deployment  across various  platforms. Lastly,  a  comprehensive toolchain  to
%extend and modify  the system is provided. All project  components created for
%this project are open source and available under the MIT license.

\vfill

\paragraph{Key Words:} Red Pitaya, STEMlab,  CIC, FIR, sampling, downsampling,
attenuation   stopband,  passband,   aliasing,  open   source,  MIT   license,
oscilloscope, spectrum analyzer, FFT, JavaScript, digital filters,
