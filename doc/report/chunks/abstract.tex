% ==============================================================================
%
%                               A B S T R A C T
%
% ==============================================================================
\chapter*{Abstract} % -------------------------------------------------------- %
\label{ch:app:abstract}
% ---------------------------------------------------------------------------- %

The objective of this project is to build a system to downsample and visualize
analog signals  using a Red  Pitaya STEMlab. This involves the  development of
appropriate filters for attenuating aliasing effects, a communication back end
to transmit  data via Ethernet,  and an oscilloscope application  to visualize
and analyze the results.

Six  filter  chains  are  implemented  on  the  STEMlab's  FPGA,  allowing  to
downsample   the  \SI{125}{\MHz}   source   signal   to  frequencies   between
\SI{50}{\kHz} and \SI{25}{\MHz}.  Using a  combination of CIC and FIR filters,
a stopband  attenuation of  \SI{60}{\dB} is achieved. RMS  in the  passband is
linear  to within  \SI{2}{\percent}  except at  \SI{25}{\MHz},  and ripple  is
kept  to  approximately \SI{1}{\percent}. SNR  in  the  passband lies  between
\SI{70}{\dB} and \SI{80}{\dB} as calculated by Matlab's SNR algorithms.

The oscilloscope application  is based on web technologies,  allowing for easy
deployment  across various  platforms. Lastly,  a  comprehensive toolchain  to
extend and modify  the system is provided. All project  components created for
this project are open source and available under the MIT license.

\vfill

\paragraph{Key Words:} Red Pitaya, STEMlab,  CIC, FIR, sampling, downsampling,
attenuation   stopband,  passband,   aliasing,  open   source,  MIT   license,
oscilloscope, spectrum analyzer, FFT, JavaScript, digital filters,
