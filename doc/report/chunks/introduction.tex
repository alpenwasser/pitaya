% ==============================================================================
%
%                           I N T R O D U C T I O N
%
% ==============================================================================
\chapter*{Introduction} % ---------------------------------------------------- %
\label{ch:intro}
\addcontentsline{toc}{chapter}{\nameref{ch:intro}}
% ---------------------------------------------------------------------------- %

Electronic measuring equipment has historically tended to be very pricey, with
specialized appliances  sometimes costing  as much as  a middle-class  car, or
even more. While high-performance  solutions are unlikely to  be replaced with
something radically  different in  the foreseeable future,  modern, affordable
FPGAs  are  a  viable  alternative  in many  use  cases  nowadays. By  keeping
performance requirements within reasonable bounds and pairing the FPGA with an
appropriate  front-end, sufficient  performance for  many applications  can be
achieved  at a  very competitive  price point.   Additionally, an  FPGA offers
vastly  superior flexibility  over the  fixed silicon  of conventional  signal
processing chips,  since its hardware  capabilities can be altered  even after
deployment.

A suitable front-end for an FPGA usually comprises dedicated ADCs and DACs and
frequently also analog filters. Along with FPGA chips themselves, such ADC and
DAC chips  have become much  more economical in  recent years, resulting  in a
product which may cost  as little as a few hundred Swiss  Francs and which can
replace an apparatus several times as expensive.

\makeatletter
\renewcommand{\thefigure}{\@arabic\c@figure}
\makeatother
\begin{figure}[ht]
    \centering
    \tikzsetnextfilename{intro-system-overview}
\begin{tikzpicture}[
    rounded corners=1mm,
    node distance=5mm,
    fpgaComponent/.style={
        draw,
        minimum height=12ex,
        fill=q4,
    },
    SoCComponent/.style={
        minimum height=8ex,
        minimum width=6em,
        fill=ct2,
        draw=sqC,
        text=br2,
        font=\bfseries,
    },
]
    \small
    \sffamily

    \node[SoCComponent] at (0,0) (fpga) {FPGA};
    \node[SoCComponent,right=of fpga] (server) {Server};

    \begin{scope}[on background layer]
        \node[
            draw=black,
            fit=(fpga) (server),
            inner sep=2mm,
            align=left,
            text height=11ex,
            minimum height=16ex,
            fill=br1,
            text=black,
            font=\bfseries,
        ] (board) {Board};
    \end{scope}

    \node[
        right=10mm of board,
        minimum width=12ex,
        minimum height=8ex,
        fill=ct2,
        align=center,
        text=br2,
    ] (pcmonitor) {\bfseries Scope\\\\};

    \begin{scope}[on background layer]
        \node[
            draw=black,
            fit=(pcmonitor),
            minimum height=8ex,
            fill=black!60!white,
        ] (pcmonitorframe) {};
    \end{scope}

    \fill[br2] ($(pcmonitorframe.south) + (-1ex,0)$) --+ (0,-1ex) --+ (2ex,-1ex) --+ (2ex,0) -- cycle;
    \draw ($(pcmonitorframe.south) + (-1ex,0)$) --+ (0,-1ex);
    \draw ($(pcmonitorframe.south) + (1ex,0)$)--+ (0,-1ex);
    \draw ($(pcmonitorframe.south) + (1ex,0)$)--+ (0,-1ex);
    \draw[fill=br2] ($(pcmonitorframe.south) + (-6ex,-1ex)$) rectangle
          ($(pcmonitorframe.south) + (+6ex,-3ex)$);

    \draw[thick,-latex] (-18.0mm,0mm) -- ($(board.west) + (-1mm,0mm)$);
    \draw[thick,-latex] ($(board.east) + (1mm,0mm)$) -- ($(pcmonitorframe.west) + (-1mm,0mm)$);
    \draw[thick,-latex] ($(fpga.east) + (0.5mm,0mm)$) -- ($(server.west) + (-0.5mm,0mm)$);

    \begin{axis}[
        anchor=east,
        at={(-20mm,0mm)},
        width=20mm,
        height=10mm,
        axis line style={draw=none},
        tick style={draw=none},
        grid=none,
        ticks = none,
        xmin=0,
        xmax=360,
        ymin=-1.1,
        ymax=1.1,
        samples=500,
    ]
        \addplot[line cap=round,very thick,q1,domain=0:360,-]{sin(x)};
    \end{axis}

    \begin{axis}[
        anchor=south,
        yshift=1mm,
        at={(pcmonitor.south)},
        width=12mm,
        height=6mm,
        axis line style={draw=none},
        tick style={draw=none},
        grid=none,
        ticks = none,
        xmin=0,
        xmax=360,
        ymin=-1.3,
        ymax=1.3,
        samples=500,
    ]
        \addplot[very thick,q4,domain=0:360]{sin(x)};
    \end{axis}

    \node[font=\bfseries\footnotesize,text=black] at (-29mm,-8.95mm) {Signal};
\end{tikzpicture}

    \caption[System Overview]{%
        An overview of the main system components.  An analog signal (left) is
        measured, the data  goes into the server, and is  then transmitted via
        network to a computer running the scope.%
    }
    \label{fig:intro:system_overview}
\end{figure}
\makeatletter
\renewcommand{\thefigure}{\thechapter.\@arabic\c@figure}
\makeatother

This  project aims  to arm  such an  FPGA board  with logic  that can  record,
filter  and store  electrical signals  with  adjustable sampling  rates up  to
\SI{125}{\mega\hertz}.   To  complement  the hardware  subsystem,  a  software
component is provided, consisting of two applications: One runs on an embedded
GNU/Linux on the board itself, while  the other runs on a user's computer. The
application on the  board (the \emph{server}) is  responsible for transmitting
the recorded data over a network connection, while the software running on the
user's  computer  (the oscilloscope,  or  \emph{scope}  for short)  serves  to
visualize and  process the  measurement results. This  concept is  depicted in
Figure~\ref{fig:intro:system_overview}.

The  objective  of  this  project  is   to  provide  a  device  which  enables
students and hobbyists  to analyze signals encompassing the  region from audio
frequencies  to the  low megahertz  range. Compared to  the frequencies  which
modern hardware  can handle  (dozens to hundreds  of megahertz),  the sampling
rates required to process such signals  can be kept within the limits suitable
for an FPGA.

A RedPitaya  STEMlab 125-14 board  is used as  the basis for  the hardware. It
easily offers  sufficient performance  to process the  signals in  the desired
range, having an  ADC which provides a \SI{14}{\bit}  signal at \SI{125}{\MHz}
on two channels.   Indeed, downsampling this signal is  \emph{necessary} if it
is  to be  transmitted over  a  network connection,  since its  data rate  far
exceeds the available bandwidth.

Thus, two primary objectives can be defined:
\begin{enumerate}\tightlist
    \item
        The signal coming out of the ADC must be decimated.
    \item
        The resulting data stream must be visualized for the end user.
\end{enumerate}

Downsampling the  data provided by the  ADC is performed on  the FPGA. Because
downsampling introduces  unwanted frequencies  into the signal's  spectrum, it
needs  to be  processed  by filters  which  attenuate those  components. While
the  STEMlab  offers  some  limited  capability in  this  area  in  its  stock
configuration,  this was  deemed insufficient  and has  been improved  in this
project. Several  filter  chains are  implemented  to  allow decimation  rates
between \num{5}  and \num{2500},  corresponding to output  frequencies between
\SI{25}{\MHz}  and \SI{50}{\kHz},  respectively. The  server application  then
transmits the filtered data stream over Ethernet to a client.

The filter chains offer a  mean signal-to-noise ratio between \SI{63}{\dB} and
\SI{79}{\dB}, depending  on the chosen chain. Additionally,  the \SI{14}{\bit}
signal from  the ADC  is improved by  \SI{1.2}{\bit} to  \SI{1.8}{\bit}, again
depending on the chain.

Cost-effective FPGA  boards like the  STEMlab usually come without  a physical
user interface  with displays and  buttons. This keeps the device  compact and
its cost  low.  Since fast  personal computers  are ubiquitous these  days, it
seems obvious  to exploit that and  run an oscilloscope application  on such a
device. Due to  the fragmentation of  the modern computing world  into various
ecosystems (Windows, GNU/Linux, macOS, Android, \ldots), web technologies form
the foundation of  the scope; the application can be  accessed from any modern
browser. This makes life easier both for the developers and for the end users.

\paragraph{This        document        is        split        into        four
main        parts:} Part~\ref{part:project_report},         starting        on
page~\pageref{part:project_report},     contains    the     primary    project
report. Part~\ref{part:Developer_Guide}  is the  developer  guide, and  begins
on  page~\pageref{part:Developer_Guide}. A short  user  guide  is provided  in
Part~\ref{part:User_Guide}  from  page~\pageref{part:User_Guide}  onwards. The
appendices are located on page~\pageref{ch:app:fdesign} and following.

The   project  report   covers  information   relevant  to   the  design   and
implementation   of   the  final   product. Its   first   chapter  starts   on
page~\ref{ch:theory}  and presents  some  relevant  theoretical background  on
digital signal processing, with an emphasis on digital filtes, and CIC filters
in particular. The  next chapter outlines  the process leading to  the concept
for our product, and compares a  few alternative choices against it, beginning
on page~\pageref{ch:mission}.

The       design      and       implementation       of      the       product
is     detailled     in     Chapters~\ref{ch:filter_design},     \ref{ch:fpga}
and~\ref{ch:graphical_front_end}. Chapter~\ref{ch:filter_design} documents the
filter  design;  both  requirements   and  specifications  are  documented. An
overview  of  the  FPGA  implementation  is  given  in  Chapter~\ref{ch:fpga},
highlighting    some   key    points   which    proved   challenging    during
development. Finally,   the   concepts   and   design   choices   underpinning
the   scope   are   explained   in   Chapter~\ref{ch:graphical_front_end}. The
chapters   begin    on   pages~\pageref{ch:filter_design},   \pageref{ch:fpga}
and~\pageref{ch:graphical_front_end}, respectively.

The  product's  performance  is assessed  in  \emph{\nameref{ch:verification}}
from page~\pageref{ch:verification}  onwards, and Chapter~\ref{ch:conclusions}
contains some  concluding remarks  on the overall  result and  possible future
steps.

The Developer and  User Guides are mostly  self-contained. The Developer Guide
is  intended for  people who  wish to  use  our product,  or parts  of it,  to
implement a system of their own. The  User Guide is intended for end-users who
wish to perform measurements with our product.

%^^A vim: foldenable foldcolumn=4 foldmethod=marker foldmarker=<<<,>>>
