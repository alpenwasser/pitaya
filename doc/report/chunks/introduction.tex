Measuring equipments has ever been very expensive. If one wants precise devices it can go up to hundreds or thousands of Francs. Modern, cheap FPGAs can often, if combined with a proper frontend, replace those expensive dedicated devices.
The frontend consists of dedicated ADCs and DACs and oftentimes some analog filters. Those chips also have become a lot cheaper in recent years aand are thus way more accessible.
Extreme equipment, for example with high sampling rates, in contrary can still not be replaced easily.
This project aims at arming an FPGA board with logic that can record, filter and store electrical signals with adjustable sampling rates up to 125\si{\mega\hertz}. To complement the hardware part a software that runs on an embedded Linux on the integrated ARM core will be coded such that it can transmit the recorded samples over the network.
To read and visualize the samples at the other end of the network, another piece of software will be crafted, that will run on the users computer.
The primary focus lies on enabling students to analyze audio signals. Since audio signals contain very low frequencyies only up to tens of thousands of \si{\hertz} they can be sampled with rather low frequencies and thus making FPGAs an excellent choice.
An FPGA not only shines in price competitiveness but also in flexibility.
This means that the logic is not fixed in silicon and can be adjusted after the product has already been delivered.

For this project a RedPitaya board is used. It is ideal since it features a fast (125\si{\mega\hertz}) 14-bit ADC. This poses a huge amount of data, that is not even required for audio signals. Furthermore it is not realisticly possible to transmit this huge amount of data over the network.

Thus the first primary target of this thesis is to decimate the recorded signals. To avoid aliasing effects that emmerge when decimating a signal appropriate filters have to be designed and impemented that are able to attenuate unwanted signal frequencies.

The second primary target of this thesis is the design and implementation of a software-based oscilloscope. This is a grafical user interface that communicates with the RedPitaya board and visualizes the recored samples.
Traditional measuring equipment always has a built in display that visualizes the data on the device itself. This uses up a lot of space and provides very low flexibility.
Since it can be assumed that every engineur is equipped with a computer, said device should be used to display the signals. This keeps cost and required space down.
The data is then transmitted via the network which will be interfaced by both the users computer and the RedPitaya board.°