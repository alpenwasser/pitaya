Shell commands can be type thusly:
\begin{commandshell}
    user:> if [ -f "${myfile}" ];then echo "${myfile} exists!"
\end{commandshell}

\section{Makefile}
\begin{tcolorbox}[
        breakable,
        title={
            \refstepcounter{listing}
            Listing \thelisting: Makefile Code
            \label{lst:makefile}
            \addcontentsline{lol}{listing}{\protect\numberline{\thelisting}Makefile Code}
        }
    ]
    \inputminted{makefile}{code/Makefile}
\end{tcolorbox}

\section{Verilog}
\begin{tcolorbox}[
        breakable,
        width=1.2\textwidth,
        title={
            \refstepcounter{listing}
            Listing \thelisting: Verilog Code
            \label{lst:makefile}
            \addcontentsline{lol}{listing}{\protect\numberline{\thelisting}Verilog Code}
        }
    ]
\inputminted{verilog}{./code/axi_axis_reader.v}
\end{tcolorbox}

\section{VHDL}
\inputminted{vhdl}{./code/comparator.vhd}

\section{TCL}
\inputminted{tcl}{./code/create_cores.tcl}

\begin{listing}
    \begin{minted}[autogobble]{vhdl}
        entity comparator is
            generic (
                Width : integer := 14
            );
            port (
                AxDI : in unsigned(Width - 1 downto 0);
                BxDI : in unsigned(Width - 1 downto 0);
                GreaterxSO : out std_logic;
                EqualxSO : out std_logic;
                LowerxSO : out std_logic
            );
        end comparator;
    \end{minted}
    \caption{Comparator}
    \label{lst:vhdl:comparator}
\end{listing}

\section{Matlab}
Here, we shall also demonstrate breaking a code file into multiple segments
to comment on its contents.

We start with the header:
\begin{tcolorbox}[
        skin = octoboxfirst,
        title={
            \refstepcounter{listing}
            Listing \thelisting: Matlab Code
            \label{lst:makefile}
            \addcontentsline{lol}{listing}{\protect\numberline{\thelisting}Matlab Code}
        }
    ]
    \inputminted[
        linenos,
        numbersep=4pt,
        style=solarizedlight,
        firstline=1,
        lastline=12,
    ]{matlab}{./code/filterChainDesigns.m}
\end{tcolorbox}
In the next box, we start where we  left off previously, and we pack some more
header information into our listing:
\begin{tcolorbox}[
        skin = octoboxmiddle,
    ]
    \inputminted[
        linenos,
        numbersep=4pt,
        style=solarizedlight,
        firstnumber=last,
        firstline=13,
        lastline=26,
    ]{matlab}{./code/filterChainDesigns.m}
\end{tcolorbox}

Then we describe the target for the FIR filter:
\begin{tcolorbox}[
        skin = octoboxmiddle,
    ]
    \inputminted[
        linenos,
        numbersep=4pt,
        style=solarizedlight,
        firstnumber=29,
        firstline=29,
        lastline=40,
    ]{matlab}{./code/filterChainDesigns.m}
\end{tcolorbox}

After which we start the sript proper by setting up the iteration parameters:
\begin{tcolorbox}[
        skin = octoboxmiddle,
    ]
    \inputminted[
        linenos,
        numbersep=4pt,
        style=solarizedlight,
        firstnumber=43,
        firstline=43,
        lastline=64,
    ]{matlab}{./code/filterChainDesigns.m}
\end{tcolorbox}

We define some data structures to contain the filter objects for further processing:
\begin{tcolorbox}[
        skin = octoboxmiddle,
    ]
    \inputminted[
        linenos,
        numbersep=4pt,
        style=solarizedlight,
        firstnumber=66,
        firstline=66,
        lastline=78,
    ]{matlab}{./code/filterChainDesigns.m}
\end{tcolorbox}

And then we iterate:
\begin{tcolorbox}[
        skin = octoboxlast,
    ]
    \inputminted[
        linenos,
        numbersep=4pt,
        style=solarizedlight,
        firstnumber=80,
        firstline=80,
        %lastline=78,
    ]{matlab}{./code/filterChainDesigns.m}
\end{tcolorbox}

