\documentclass[a4paper,oneside]{alpenthesis/alpenthesis}
\hextrue
\paperfalse
%https://tex.stackexchange.com/a/210456/131649
\renewcommand\partnumberlinebox[2]{#2\hspace{2em}}

\begin{document}
\begin{titlingpage}
    \fullhexpage{q1}{q0}
    \flushright\sffamily

    \vspace*{5em}
    \Huge\bfseries{Red Pitaya}\\[1ex]
    \Large\mdseries{Thesis}\\[3ex]

    \normalsize\mdseries
    
    \vfill
    Raphael Frey\\
    Noah H\"usser\\[3ex]

    \vspace{5em}

    \today\\
    Version 0.0.1
\end{titlingpage}

\frontmatter
\tableofcontents*

\mainmatter
\chapter{Introduction}
\begin{itemize}\firmlist
    \item Rationale (Why?)
    \item What is the general approach to solve this problem?
    \item What has been done so far?
    \item Results of previous work
    \item What are we going to do?
    \item What are the contents of this report?
\end{itemize}

\part{System Analysis}
\chapter{Requirements}
\begin{itemize}\firmlist
    \item Detailed List of Specifications
\end{itemize}

\chapter{Existing Solution}
\section{Previous Work}
\section{Red Pitaya Platform}
General Info about Red Pitaya Project:
\begin{itemize}\firmlist
    \item How is the PITA project structured? (logically, license-wise, philosophically)
    \item Why do we care about this?
\end{itemize}
\subsection{FPGA}
\subsection{Linux}

\chapter{Conclusions}
Decision matrix


\part{User Guide}

\part{Developer Guide}
\chapter{IP Core}
Documentation of our FPGA Project (structure, interfaces, registers \ldots)
\chapter{Linux}
Kernel module, server

\chapter{Tool Chain}
Vivado, Build Box, ARM Linux, TCL, Makefiles, Libs for building server application

\part{Implementation}
\chapter{Data Acquisition System}
\section{FPGA}
\section{Kernel Module}

\chapter{Filters}
\chapter{Graphical Front End}
\chapter{Server}

\part{Theoretical Background}
\chapter{Filters}
FIR, IIR, CIC, Half-band, \ldots
\begin{itemize}
    \item downsampling: Aliasing into passband
    \item FIR, IIR: Pros, Cons, not much detail
    \item CIC, half-band: More detailled
\end{itemize}

\end{document}
