\documentclass[11pt]{article}

\usepackage[a4paper,top=30mm,bottom=40mm,left=30mm,right=30mm]{geometry}
\usepackage[ngerman]{babel}

\author{Raphael Frey, Noah H\"usser}
\title{Statusrapport Thesis}
\date{\today}

\begin{document}
\maketitle
\section{Ausgangslage}

Beim  Testen des  Ergebnisses des  Vorg\"angerprojekts konnten  wir dessen  IP
Cores nicht erfolgreicht in das Red Pitaya Projekt einbinden.  Beim Erforschen
der  Ursache  und Debuggen  sind  wir  in  der  Folge auf  einige  unangenehme
Tatsachen gestossen:

\begin{itemize}
    \item FPGA Codebase ist vonseiten des Red Pitaya-Projekts stark im Umbruch.

    \item  Es  existieren  momentan  zwei Branches: Ein  alter,  der  gem\"ass
    Herstellerangaben  keine Updates  mehr  erh\"alt, und  ein neuer,  welcher
    gem\"ass  Herstellerangaben  ``still  under   heavy  development  and  not
    stable'' ist.

    \item In der Dokumentation, soweit sie denn \"uberhaupt vorhanden ist, ist
    oft nicht klar, ob sie jetzt f\"ur die neue oder die alte Codebase gilt.

    \item Neben der  Codebase selbst ist auch die  Dokumentation der Toolchain
    auf Seiten des FPGA nicht besonders gut.

    \item Die  Dokumentation und Codebase des  Linux-Teils des Pitaya-Projekts
    ist aber einigermassen passabel.
\end{itemize}

Es stellt sich also die Grundsatzfrage: 

\begin{itemize}
    \item Versuchen wir, das Projekt mit der alten FPGA-Codebase zum Laufen zu
    bringen, wobei etwaige  Bugs und fehlende Features von  uns entweder durch
    Fixes oder Workarounds gehandhabt werden m\"ussten, oder
    \item setzen  wir auf die neue  Codebase, mit dem Vorbehalt,  obwohl diese
    gem\"ass Herstellerangaben noch gar nicht reif ist,
    \item oder beschreiten wir einen dritten Weg?
\end{itemize}

\section{Getroffene Massnahmen}
\begin{itemize}
    \item asdf
\end{itemize}

\end{document}
