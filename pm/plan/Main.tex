\RequirePackage{snapshot}
\documentclass[a4paper,oneside]{alpenspecs/alpenspecs}
% <<< Preamble
\hexfalse
\paperfalse
%https://tex.stackexchange.com/a/210456/131649
\renewcommand\partnumberlinebox[2]{#2\hspace{2em}}
\usetikzlibrary{positioning}
\tcbuselibrary{breakable}
\tcbset{shield externalize}
\makeindex
% >>>
\begin{document}
\begin{titlingpage} % <<<
    \fullhexpage{q1}{q0}
    \flushright\sffamily

    \vspace*{5em}
    \Huge\bfseries{Red Pitaya}\\[1ex]
    \Large\mdseries{Specifications}\\[3ex]

    \normalsize\mdseries

    \vfill
    Raphael Frey\\
    Noah H\"usser\\[3ex]

    \vspace{5em}

    \today\\
    Version 0.0.1
\end{titlingpage} % >>>

\frontmatter % <<<
% https://en.wikipedia.org/wiki/Edition_notice
\tableofcontents*
%\clearpage
%\listoffigures*
%\clearpage
%\listoftables*
%\clearpage
% >>>

\mainmatter

\chapter{Work Packages} % <<< ------------------------------------------------ %
\label{ch:wpac}
% ---------------------------------------------------------------------------- %

        %wpac:
        %{Start Date}
        %{End Date}
        %{Hours}
        %{Ancestors}
        %{Descendants}
        %{Assignee}
        %{Description}

\section{Filter Design}
\label{sec:filters}

Contains the  work packages  relating to  filter design  in Matlab  and filter
implementation in the FPGA tool chain.

\subsection{Filter Research}
\label{subsec:filter:research}

\wpac
     {}
     {}
     {}
     {None}
     {}
     {Raphael Frey}
     {%
         Research digital the filter technologies which will be needed for this
         project. In particular, this includes FIR and CIC filters, since Xilinx
         provide predefined blocks for these filters in their toolchain.

         Besides the general knowledge on digital filters, the Xilinx toolchain
         must be researched in order to understand the FIR and CIC filter blocks
         and allow their usage.
     }

\subsection{Matlab Scripts for Filter Design}
\label{subsec:filter:matlab}

\wpac
     {}
     {}
     {}
     {None}
     {}
     {Raphael Frey}
     {%
         Matlab scripts which design various filter chains. These consist of one
         or several dispatcher scripts where the filter chains are specified in
         terms of their frequency band properties, and scripts which actually
         design the filters according to these specifications.

         Specifically, scripts for generating CIC filters and their compensation
         filters, as well as FIR filters are needed. For FIR filters, the resulting
         coefficients are to be saved in files so that they can be loaded by the
         Xilinx tool chain.

         If possible Matlab's parallelism should be exploited to reduce filter
         design times.
     }

\subsection{TCL Scripts for Filter Evaluation}
\label{subsec:filter:tcl}

\wpac
     {}
     {}
     {}
     {None}
     {}
     {Raphael Frey}
     {%
        In order to assess the resource usage of the filter chains which are
        designed by Matlab, the respective filter blocks (CIC, FIR) need to be
        implemented in Vivado and a bitstream must be compiled.

        Since the number of filters generated by package~\ref{sec:filter:matlab}
        can be very large (dozens or even hundreds), this process must be automated
        to be of any use. For this, TCL scripts are used.

        For FIR filters, the scripts load the coefficient files which have been
        generated by Matlab.

        Vivado will  generate usage  reports which  can then  be automatically
        post-processed.
     }


\subsection{Documentation}
\label{subsec:filter:doc}

\wpac
     {}
     {}
     {}
     {None}
     {}
     {Raphael Frey}
     {%
         Documentation  for the  filter design  tool chain. The  documentation
         should be sufficiently detailed so that the tool chain can be used by
         a  person  who wishes  to  compile  a  new bitstream  with  different
         filters.
     }


\section{Documentation}
\label{sec:docs}

\subsection{Disposition}
\wpac
     {}
     {}
     {}
     {None}
     {}
     {Noah H\"usser}
     {%
        A general outline of the thesis document.%
     }

\subsection{Report}
\wpac
     {}
     {}
     {}
     {None}
     {}
     {Raphael Frey, Noah H\"usser}
     {%
        The actual thesis document.%
     }


\section{Firmware}
\label{sec:firmware}

\subsection{Linux}
\label{subsec:fw:linux}

\wpac
     {}
     {}
     {}
     {None}
     {}
     {Noah H\"usser}
     {%
        Load Ubuntu Linux onto the Red Pitaya.%
     }

\subsection{Server Application}
\label{subsec:fw:server}

\subsubsection{Design Decisions}
\label{subsubsec:fw:server:design-decisions}
\wpac
     {}
     {}
     {}
     {None}
     {}
     {Noah H\"usser}
     {%
         Design decisions for server application.
     }

\subsubsection{Server Application}
\label{subsubsec:fw:server:server}
\wpac
     {}
     {}
     {}
     {None}
     {}
     {Noah H\"usser}
     {%
         The server application itself.%
     }

\subsubsection{External Libraries}
\label{subsubsec:fw:server:libs}
\wpac
     {}
     {}
     {}
     {None}
     {}
     {Noah H\"usser}
     {%
         A build infrastructure which allows for the compilation of the
         external libraries which are needed for the server application.

         This includes \code{uWebSockets}, \code{zlib} and \code{OpenSSL}.
     }

\subsubsection{Documentation}
\label{subsubsec:fw:server:docs}
\wpac
     {}
     {}
     {}
     {None}
     {}
     {Raphael Frey, Noah H\"usser}
     {%
         Documentation for the server application build process.
     }

\subsection{FPGA}
\label{subsec:fw:fpga}

\subsubsection{Minimal Working Example}
\label{subsubsec:fw:fpga:mwe}
\wpac
     {}
     {}
     {}
     {None}
     {}
     {Noah H\"usser}
     {%
         Get a minimal working example project running on the Red Pitaya.

         This requires importing the cores from Pavel Denim's project as
         well as compiling the Linux for the Red Pitaya.

         Correct functionality of the Red Pitaya hardware, particularly the
         ADC, is to be verified via Anton Potochnik's frequency counter.
     }

\subsubsection{Port Zynq Logger}
\label{subsubsec:fw:fpga:logger}
\wpac
     {}
     {}
     {}
     {None}
     {}
     {Noah H\"usser}
     {%
         Port the Zynq Logger to the Red Pitaya. The main challenges will be
         the porting of its interface (for which a block \code{axis2datalanes}
         is developed), as well as understanding and properly implementing the
         device tree and the kernel module which is needed for the Linux OS
         to interface with the logger hardware.
     }

\subsubsection{Filter Chains}
\label{subsubsec:fw:fpga:filters}
\wpac
     {}
     {}
     {}
     {None}
     {}
     {Raphael Frey}
     {%
         Implement the filter chains as designed in Matlab on the FPGA.

         This requires correct functionality of the FIR and CIC filter
         blocks.
     }


\section{Validation}
\label{sec:validation}

\subsubsection{Build Process}
\label{subsec:validation:build}
\wpac
     {}
     {}
     {}
     {None}
     {}
     {Raphael Frey}
     {%
         Build the entire project (Linux, bitstream), flash
         result onto the Red Pitaya and verify correct functionality.
     }

% >>>

\backmatter
\end{document}
%^^A vim: foldenable foldcolumn=4 foldmethod=marker foldmarker=<<<,>>>
